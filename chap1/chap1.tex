
\chapter{Pré-Requisitos Conjuntistas}
Durante o texto, precisamos de algumas definições e resultados envolvendo noções básicas sobre conjuntos e funções.

Não é objetivo deste capítulo desenvolver formalmente os princípios da teoria dos conjuntos, mas apenas estabelecer convenções notacionais e fixar algumas definições que serão utilizadas ao longo do texto. Assume-se familiaridade do leitor com funções e com manipulação de conjuntos ao nível básico.
Um curso de pré-cálculo ou um curso básico de álgebra linear apresentam conhecimentos mais do que suficientes para acompanhar a leitura deste texto.

\section{Pares ordenados}
Um par ordenado é um conjunto especial constituído a partir de dois outros conjuntos, $a, b$, e denotado por $(a, b)$.
Sua principal propriedade é que $a, b$ são pares ordenados, vale que $(a, b)=(c, d)$ se, e somente se $a=c$ e $b=d$.

Formalmente, em Teoria dos Conjuntos, um par ordenado é definido como qualquer conjunto que possui esta propriedade.
Um modo popular de o fazer é através do chamado par de Kuratowski, que define o par ordenado $(a, b)$ como o conjunto $\{\{a\}, \{a, b\}\}$.
Definindo-se o par ordenado dessa forma, fica a cargo do leitor interessado verificar que $(a, b)=(c, d)$ se, e somente se $a=c$ e $b=d$.

A definição de Kuratowski pode parecer estranha, ou anti-natural, mas satisfaz o que se espera de um par ordenado.
Por isso mesmo, não é desejável que teoremas sobre Álgebra dependam dela. E não dependerão. Tudo que será utilizado ao longo de todo o texto é a propriedade mencionada no parágrafo anterior.

Se $x$ é um par ordenado, existem únicos $a, b$ tais que $x=(a, b)$.
Tais $a, b$ são chamados de \emph{primeira coordenada} e \{segunda coordenada\}, respectivamente, e, em alguns contextos, será conveniente denotá-los por $\pi_1(x)$ e $\pi_2(x)$.

\section{Famílias e produtos cartesianos}
Funções são objetos matemáticos normalmente pensados como entes $f:X\rightarrow Y$ que recebem uma entrada $x$, em um conjunto de possíveis entradas $x \in X$ e devolvem uma saída $f(x)\in Y$.
O conjunto $X$ é chamado de \emph{domínio} de $f$, e denotado por $\dom f$. O conjunto de todos os valores assumidos por $f$ é a \emph{imagem} de $f$, e denotado por $\ran(f)=\{f(x): x \in X\}$.
Um conjunto que contém a imagem é chamado de um \emph{contradomínio} de $f$.
Na notação $f:X\rightarrow Y$, estamos dizendo que $Y$ é um contradomínio de $f$, de modo que $\ran(f)\subseteq Y$.
Note que o contradomínio de uma função não é único: por exemplo, considerando que o conjunto dos números complexos, $\mathbb C$, possui uma cópia do conjunto dos números reais, $\mathbb R$, dentro de si, a função $f:\mathbb R\rightarrow \mathbb R$ dada por $f(x)=x^2$ tem como contradomínios tanto $\mathbb R$ como $\mathbb C$
Formalmente, a função propriamente dita costuma ser definida como o conjunto dos pares ordenados $\{(x, f(x)): x \in X\}$, mas tais detalhes serão de pouca relevância nesse texto.

Muitas vezes, em Matemática, pensamos no conceito de função com o intuito de representar um conjunto de valores indexados, em que a ideia de ``dispositivo de entrada/saída'' muito presente, por exemplo, em Cálculo Diferencial em integral perde importância. Uma dessas ocasiões é no tratamento de sequências. Formalmente, sequências são funções cujo domínio é o conjunto dos números naturais, mas, em muitos contextos, pensa-se em sequências como uma coleção de objetos enumerados, e não como um dispositivo de entrada/saída.

Nesses contextos, é muito usual trocar a terminologia usual utilizada para tratar funções por outra terminologia, em que a função, mesmo sendo o mesmo objeto matemático que no outro contexto, passa a ser chamada de \emph{família}.

Uma família $a=(a_i: i \in I)=(a_i)_{i \in I}$ é uma função cujo domínio é $I$.
Nessa notação, pensamos em $i$ como sendo uma variável muda, como ocorre com o símbolo $x$ nos parágrafos anteriores.
$I$, que é o domínio da função, costuma ser chamado de conjuntos de índices.
A notação não deixa explícito um contradomínio, como ocorre ao escrever $f:X\rightarrow Y$.
Quando se torna relevante, costumamos escrever sentenças como ``a família $(a_i)_{i \in I}$ assume valores em $Y$'', ou, simplesmente, ``$(a_i)_{i \in I}$ é uma família em $Y$''.
Já $a_i$ é o elemento $a(i)$.

Nesse texto, consideraremos que $(a_i: i \in I)$ e $((i, a(i)): i \in I)$ são o mesmo objeto matemático, não havendo qualquer distinção formal entre eles.
A distinção é meramente notacional, e pode ser intercambiada a qualquer momento sem nenhuma perda de formalismo matemático.

No quadro abaixo, apresentamos uma comparação entre as duas notações.
Enfatizamos novamente que, matematicamente, funções e famílias podem ser vistas como o mesmo objeto.
\begin{table}[h]
    \centering
    \begin{tabular}{lllll}
        \hline
        \textbf{Conceito} & \textbf{Função} & \textbf{Família} \\ \hline
        Mapa & $u:I\rightarrow A$ & $(u_i)_{i \in I}=(u_i: i \in I)$ \\
        Valor & $u(i)$ & $u_i$ \\
        Imagem & $\ran u$ & $\{u_i: i \in I\}$\\
        Intuição & objeto dinâmico & objeto estático \\
        Inputs & domínio $I$ & conjunto de índices $I$ \\
        \hline
    \end{tabular}
    \caption{Comparativo de família e função}
\end{table}

Como exemplos, consideremos sequências infinitas e finitas:

\begin{exemplo}[Sequências]
    Uma sequência é uma família cujo conjunto de índices é $\mathbb N$.
    Compare a intuição que passa as notações:
    \begin{itemize}
    \item considere a sequência $u=(\frac{1}{2^n}))_{n \in \mathbb N}$...
    \item considere a função $u:\mathbb N\rightarrow \mathbb R$ dada por $u(n)=\frac{1}{2^n}$...
    \end{itemize}
\end{exemplo}

\begin{exemplo}[Sequências finitas]
    Se $n\geq 1$, identificamos $n=\{0, 1, \dots, n-1\}$.
    Assim:
    \begin{itemize}
    \item Uma família com $n$ elementos é uma família $(a_i)_{i<n}=(a_i)_{i \in n}=(a_0, \dots, a_{n-1})$.
    \end{itemize}

    Essa notação é bastante funcional no sentido de que dá significado como conjunto aos números naturais, e corresponde à construção usual dos números naturais na Teoria dos Conjuntos.
    Como desvantagem, seus contadores se iniciam no $0$, e não no $1$, o que pode ser pouco intuitivo e não coincidir com a notação da maioria dos textos de matemática, apesar de ser muito adotada em textos mais próximos de Teoria dos Conjuntos.
\end{exemplo}

Agora vamos seguir para a definição de produto cartesiano.
Primeiro, vamos lembrar a definição de produto cartesiano de dois conjuntos.

\begin{definition}[Produto cartesiano de dois conjuntos]
    Sejam $A, B$ conjuntos. Então $A\times B=\{(a, b): a\in A, b \in B\}$ é o \emph{produto cartesiano de $A$ e $B$}.
    Ou seja, o conjunto de todos os pares ordenados $(a, b)$ tais que $a\in A$ e $b\in B$.
\end{definition}

Pares ordenados são conjuntos especiais que carregam duas coordenadas de modo a permitem distinguir a ordem dos elementos.
Sua propriedade principal é a de se $a, b, c, d$ são conjuntos, então $(a, b)=(c, d)$ se, e somente se $a=c$ e $b=d$.
Uma construção usual, chamada de par de Kuratowski, para a qual não é difícil provar que vale essa propriedade, é dada por $(a, b)=\{\{a\}, \{a, b\}\}$. Porém, isso não será importante neste texto.


\begin{definition}[Produto cartesiano de conjuntos]
Seja $(A_i)_{i \in I}$ uma família de conjuntos.
O produto cartesiano de conjuntos é o conjunto $\prod_{i \in I} A_i$ definido como o conjunto de todas as famílias $(a_i: i \in I)$ tais que para cada $i \in I$, $a_i \in A_i$.
\[\prod_{i \in I} A_i=\{(a_i)_{i \in I}: \forall i \in I\, a_i \in A_i\}.\]
\end{definition}


\begin{definition}[Exponenciação de conjuntos]
    Sejam $A, I$ conjuntos.
    O conjunto $A^I$ é o conjunto de todas as funções de $I$ em $A$. Ou seja, $A^I=\{f:I\rightarrow A\}$.
    Note que:

    \[A^I=\prod_{i \in I}A=\{(a_i)_{i \in I}: \forall i \in I\,  a_i\in A\}.\]
    \end{definition}

    Na notação anterior, se $n\geq 1$, então:
    \[A^n=\{(a_i)_{i<n}:\forall i<n\, a_i \in A\}=\{(a_0, \dots, a_{n-1}):a_0, \dots, a_{n-1}\in A\}\approx A\times \dots \times A \,(n \text{ vezes}).\]

    \section{Operações}
Ao trabalharmos com estruturas algébricas necessitaremos da noção de operação, que se define como a seguir.
\begin{definition}[Operações $n$-árias]
    Se $X$ é um conjunto e $n \in \mathbb N$, uma operação $n$-ária em $X$ é uma função $f:X^n\rightarrow X$.
\end{definition}

Operações $2$-árias e $1$-árias são frequentemente chamadas de \emph{binárias} e \emph{unárias}, respectivamente.

Caso $*$ seja uma operação binária, a notação $x*y$ é frequentemente utilizada para denotar $x*y$.

Caso $*$ seja uma operação unária, a notação $*x$ é frequentemente utilizada para denotar $*(x)$.
