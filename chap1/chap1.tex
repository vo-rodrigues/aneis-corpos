
\chapter{Pré-Requisitos Conjuntistas}
Antes de descrevermos grupos, precisamos de algumas definições e resultados básicos envolvendo noções básicas sobre conjuntos e funções. Para isso, utilizaremos a notação usual de conjuntos, como $\mathcal P(X)$ para o conjunto das partes de $X$, $X^n$ para o produto cartesiano de $n$ cópias de $X$, e assim por diante.

Não é objetivo deste texto desenvolver a parte inicial da Teoria dos Conjuntos. Apenas apresentaremos algumas definições, notações e resultados básicos que utilizaremos ao longo do texto.
\section{Operações}

\begin{definition}[Operações $n$-árias]
    Se $X$ é um conjunto e $n \in \mathbb N$, uma operação $n$-ária em $X$ é uma função $f:X^n\rightarrow X$.
\end{definition}

Operações $2$-árias e $1$-árias são frequentemente chamadas de \emph{binárias} e \emph{unárias}, respectivamente.

Caso $*$ seja uma operação binária, a notação $x*y$ é frequentemente utilizada para denotar $x*y$.

Caso $*$ seja uma operação unária, a notação $*x$ é frequentemente utilizada para denotar $*(x)$.
\section{Produtos cartesianos de conjuntos}

Famílias são funções com notação especial. Tal notação é utilizada quando pensamos em uma função como um ``conjunto indexado de valores'' ao invés de um ``dispositivo de entrada/saída''.

Matemáticamente, funções e famílias podem ser vistas como o mesmo objeto.
\begin{table}[h]
    \centering
    \begin{tabular}{lllll}
        \hline
        \textbf{Conceito} & \textbf{Função} & \textbf{Família} \\ \hline
        Mapa & $u:I\rightarrow A$ & $(u_i)_{i \in I}=(u_i: i \in I)$ \\
        Valor & $u(i)$ & $u_i$ \\
        Imagem & $\ran u$ & $\{u_i: i \in I\}$\\
        Intuição & objeto dinâmico & objeto estático \\
        Inputs & domínio $I$ & conjunto de índices $I$ \\
        \hline
    \end{tabular}
    \caption{Comparativo de família e função}
\end{table}

Exemplo: sequências. Uma sequência é uma família cujo conjunto de índices é $\mathbb N$. Compare a intuição que passa as notações:
\begin{itemize}
\item Considere a sequência $u=(\frac{1}{2^n}))_{n \in \mathbb N}$...
\item Considere a função $u:\mathbb N\rightarrow \mathbb R$ dada por $u(n)=\frac{1}{2^n}$...
\end{itemize}

Exemplo: sequências finitas. Se $n\geq 1$, identificamos $n=\{0, 1, \dots, n-1\}$. Assim:
\begin{itemize}
\item Uma família com $n$ elementos é uma família $(a_i)_{i<n}=(a_i)_{i \in n}=(a_0, \dots, a_{n-1})$.
\end{itemize}

\begin{definition}[Produto cartesiano de conjuntos]
Seja $(A_i)_{i \in I}$ uma família de conjuntos. O produto cartesiano de conjuntos é o conjunto $\prod_{i \in I} A_i$ definido como o conjunto de todas as famílias $(a_i: i \in I)$ tais que para cada $i \in I$, $a_i \in A_i$.
$$\prod_{i \in I} A_i=\{(a_i)_{i \in I}: \forall i \in I\, a_i \in A_i\}.$$
\end{definition}


\begin{definition}[Exponenciação de conjuntos]
    Sejam $A, I$ conjuntos. O conjunto $A^I$ é o conjunto de todas as funções de $I$ em $A$. Ou seja, $A^I=\{f:I\rightarrow A\}$. Note que:

    $$A^I=\prod_{i \in I}A=\{(a_i)_{i \in I}: \forall i \in I\,  a_i\in A\}.$$
    \end{definition}

    Na notação anterior, se $n\geq 1$ $$A^n=\{(a_i)_{i<n}:\forall i<n\, a_i \in A\}=\{(a_0, \dots, a_{n-1}):a_0, \dots, a_{n-1}\in A\}\approx A\times \dots \times A \,(n \text{ vezes}).$$
    \subsection{Produtos de anéis}

    \begin{definition}[Produto Direto de dois anéis]
        Sejam $R, S$ anéis. O produto direto de $R$ e $S$ é o conjunto $R\times S$ munido das operações ``ponto à ponto'': dados $a=(a_1, a_2)\in R\times S$ e $b=(b_1, b_2)\in R\times S$, temos:
        $$a+b=(a_1+b_1, a_2+b_2)$$
        $$a\cdot b=(a_1\cdot b_1, a_2\cdot b_2)$$
        $$0=(0_R, 0_S)$$
        $$1=(1_R, 1_S)$$
    \end{definition}
    
    Exemplo: Seja $R=\mathbb Z_3$ e $S=\mathbb Z_4$. Então $(2, 2)\in R\times S$ e $(1, 2)\in R\times S$. Temos:
    $$(2, 2)+(1, 2)=(2+ 1, 2+ 2)=(0, 0)$$
    $$(2, 2)\cdot (2, 2)=(2\cdot 2, 2\cdot 2)=(1, 0)$$
