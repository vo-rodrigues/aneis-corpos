\chapter{Agradecimentos}

O autor agradece às seguintes pessoas que contribuíram com a elaboração deste material:
\begin{itemize}
\item \textbf{Gabriel Alves Andretta} -- Aluno do Bacharelado em Matemática do IME-USP. Foi monitor da disciplina ``Anéis e Corpos'' em 2025 e avisou sobre erros de digitação no texto.
\item \textbf{Matheus Engelberg Teixeira da Silva Ulian} -- Aluno do Bacharelado em Matemática do IME-USP. Apontou diversos erros de digitação no texto.
\item \textbf{Renan Ribeiro Marcelino} -- Aluno do Bacharelado em Ciência da Computação IME-USP. Colaborou com algumas correções de erros de digitação no texto a partir do GitHub.
\item \textbf{Ugo Bruzzo} -- Professor do Departamento de Matemática do IME-USP. Lecionou o primeiro terço dessa disciplina em 2025, e indicou uma porção considerável dos exercícios aqui expostos.
\end{itemize}

\chapter{Prefácio}

Estas notas começaram a ser escritas durante o primeiro semestre de 2025, enquanto lecionava a disciplina MAT0264 - Anéis e Corpos, no Instituto de Matemática e Estatística da Universidade de São Paulo (IME-USP).
No presente estado, elas estão em um formato de rascunho, e não são um material completo, nem revisado. O objetivo é que, ao longo do semestre, as notas sejam revisadas e completadas, de modo a se tornarem um material didático mais completo e acessível aos alunos da disciplina.

É assumido que o estudante já tem algum traquejo ao lidar com números inteiros e aritmética modular, tendo já estudado, formalmente, divisibilidade de inteiros, congruência módulo $n$ e os anéis $\mathbb Z_n$.
Será assumida a existência do anel dos números inteiros.
Ao longo do texto, apresentaremos as construções de todos os outros anéis relevantes, porém alguns outros anéis importantes e conhecidos, como $\mathbb Q$, $\mathbb R$ e $\mathbb C$, com o qual se espera que o estudante já possua alguma familiaridade, serão utilizados em exemplos desde seu início, mesmo antes que construções formais sejam apresentadas.

Ao final de cada seção serão apresentados exercícios. Recomenda-se que o estudante resolva-os para fixar o conteúdo apresentado.