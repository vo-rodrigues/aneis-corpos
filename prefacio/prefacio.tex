\chapter{Prefácio}

Estas notas começaram a ser escritas durante o primeiro semestre de 2025, enquanto lecionada a disciplina MAT0264 - Anéis e Corpos, no Instituto de Matemática e Estatística da Universidade de São Paulo (IME-USP).
No presente estado, elas estão em um formato de rascunho, e não são um material completo, nem revisado. O objetivo é que, ao longo do semestre, as notas sejam revisadas e completadas, de modo a se tornarem um material didático mais completo e acessível aos alunos da disciplina.

É assumido que o estudante já tem algum traquejo ao lidar com números inteiros e aritmética modular, tendo já estudado, formalmente, divisibilidade de inteiros, congruência módulo $n$ e os anéis $\mathbb Z_n$.
Será assumida a existência do anel dos números inteiros, e, ao longo do texto, apresentaremos as contruções de todos os outros anéis relevantes.
Porém, alguns anéis importantes e conhecidos, como $\mathbb Q$, $\mathbb R$ e $\mathbb C$, com o qual espera-se que o estudante já possua alguma familiaridade, serão utilizados em exemplos desde o começo, mesmo antes de que construções formais sejam apresentadas.

Ao final de cada seção, serão apresentados exercícios, que o estudante deve resolver para fixar o conteúdo apresentado.

O autor deste texto agradece ao Professor Ugo Bruzzo, que lecionou o primeiro terço dessa disciplina, e formulou uma porção considerável dos exercícios aqui expostos.