\documentclass{book}
\usepackage[brazil]{babel}
\usepackage{amsthm, amsmath, amssymb}
\usepackage[a4paper]{geometry}
\usepackage{enumitem}
\usepackage{stmaryrd}
\usepackage{tikz-cd}
\usepackage{indentfirst}
\usepackage{hyperref}
\usetikzlibrary{babel}
\usepackage{float}
\usepackage{datetime}
\theoremstyle{definition}

\newtheorem{definition}{Definição}[chapter]
\AtEndEnvironment{definition}{\null\hfill\qedsymbol}%

\newtheorem{prop}[definition]{Proposição}
\newtheorem{corol}[definition]{Corolário}
\newtheorem{exer}[definition]{Exercício}
\newtheorem{lemma}[definition]{Lema}
\newtheorem{theorem}[definition]{Teorema}
\newtheorem{exemplo}[definition]{Exemplo}
\AtEndEnvironment{eexemplo}{\null\hfill\qedsymbol}%


\DeclareMathOperator{\ran}{ran}
\DeclareMathOperator{\supp}{supp}
\DeclareMathOperator{\gr}{gr}
\DeclareMathOperator{\id}{id}
\author{Prof. Vinicius Rodrigues}
\title{Notas da disciplina MAT0264 - Anéis e Corpos}
\date{\today, \currenttime}
\begin{document}
\frontmatter
\begin{titlepage}
    \maketitle
\end{titlepage}
\tableofcontents
\chapter{Prefácio}

Estas notas começaram a ser escritas durante o primeiro semestre de 2025, enquanto lecionada a disciplina MAT0264 - Anéis e Corpos, no Instituto de Matemática e Estatística da Universidade de São Paulo (IME-USP). No presente estado, elas estão em um formato de rascunho, e não são um material completo, nem revisado. O objetivo é que, ao longo do semestre, as notas sejam revisadas e completadas, de modo a se tornarem um material didático mais completo e acessível aos alunos da disciplina.
\mainmatter

\chapter{Pré-Requisitos Conjuntistas}
Antes de descrevermos grupos, precisamos de algumas definições e resultados básicos envolvendo noções básicas sobre conjuntos e funções. Para isso, utilizaremos a notação usual de conjuntos, como $\mathcal P(X)$ para o conjunto das partes de $X$, $X^n$ para o produto cartesiano de $n$ cópias de $X$, e assim por diante.

Não é objetivo deste texto desenvolver a parte inicial da Teoria dos Conjuntos. Apenas apresentaremos algumas definições, notações e resultados básicos que utilizaremos ao longo do texto.
\section{Operações}

\begin{definition}[Operações $n$-árias]
    Se $X$ é um conjunto e $n \in \mathbb N$, uma operação $n$-ária em $X$ é uma função $f:X^n\rightarrow X$.
\end{definition}

Operações $2$-árias e $1$-árias são frequentemente chamadas de \emph{binárias} e \emph{unárias}, respectivamente.

Caso $*$ seja uma operação binária, a notação $x*y$ é frequentemente utilizada para denotar $x*y$.

Caso $*$ seja uma operação unária, a notação $*x$ é frequentemente utilizada para denotar $*(x)$.
\section{Produtos cartesianos de conjuntos}

Famílias são funções com notação especial. Tal notação é utilizada quando pensamos em uma função como um ``conjunto indexado de valores'' ao invés de um ``dispositivo de entrada/saída''.

Matemáticamente, funções e famílias podem ser vistas como o mesmo objeto.
\begin{table}[h]
    \centering
    \begin{tabular}{lllll}
        \hline
        \textbf{Conceito} & \textbf{Função} & \textbf{Família} \\ \hline
        Mapa & $u:I\rightarrow A$ & $(u_i)_{i \in I}=(u_i: i \in I)$ \\
        Valor & $u(i)$ & $u_i$ \\
        Imagem & $\ran u$ & $\{u_i: i \in I\}$\\
        Intuição & objeto dinâmico & objeto estático \\
        Inputs & domínio $I$ & conjunto de índices $I$ \\
        \hline
    \end{tabular}
    \caption{Comparativo de família e função}
\end{table}

Exemplo: sequências. Uma sequência é uma família cujo conjunto de índices é $\mathbb N$. Compare a intuição que passa as notações:
\begin{itemize}
\item Considere a sequência $u=(\frac{1}{2^n}))_{n \in \mathbb N}$...
\item Considere a função $u:\mathbb N\rightarrow \mathbb R$ dada por $u(n)=\frac{1}{2^n}$...
\end{itemize}

Exemplo: sequências finitas. Se $n\geq 1$, identificamos $n=\{0, 1, \dots, n-1\}$. Assim:
\begin{itemize}
\item Uma família com $n$ elementos é uma família $(a_i)_{i<n}=(a_i)_{i \in n}=(a_0, \dots, a_{n-1})$.
\end{itemize}

\begin{definition}[Produto cartesiano de conjuntos]
Seja $(A_i)_{i \in I}$ uma família de conjuntos. O produto cartesiano de conjuntos é o conjunto $\prod_{i \in I} A_i$ definido como o conjunto de todas as famílias $(a_i: i \in I)$ tais que para cada $i \in I$, $a_i \in A_i$.
$$\prod_{i \in I} A_i=\{(a_i)_{i \in I}: \forall i \in I\, a_i \in A_i\}.$$
\end{definition}


\begin{definition}[Exponenciação de conjuntos]
    Sejam $A, I$ conjuntos. O conjunto $A^I$ é o conjunto de todas as funções de $I$ em $A$. Ou seja, $A^I=\{f:I\rightarrow A\}$. Note que:

    $$A^I=\prod_{i \in I}A=\{(a_i)_{i \in I}: \forall i \in I\,  a_i\in A\}.$$
    \end{definition}

    Na notação anterior, se $n\geq 1$ $$A^n=\{(a_i)_{i<n}:\forall i<n\, a_i \in A\}=\{(a_0, \dots, a_{n-1}):a_0, \dots, a_{n-1}\in A\}\approx A\times \dots \times A \,(n \text{ vezes}).$$
    \subsection{Produtos de anéis}

    \begin{definition}[Produto Direto de dois anéis]
        Sejam $R, S$ anéis. O produto direto de $R$ e $S$ é o conjunto $R\times S$ munido das operações ``ponto à ponto'': dados $a=(a_1, a_2)\in R\times S$ e $b=(b_1, b_2)\in R\times S$, temos:
        $$a+b=(a_1+b_1, a_2+b_2)$$
        $$a\cdot b=(a_1\cdot b_1, a_2\cdot b_2)$$
        $$0=(0_R, 0_S)$$
        $$1=(1_R, 1_S)$$
    \end{definition}
    
    Exemplo: Seja $R=\mathbb Z_3$ e $S=\mathbb Z_4$. Então $(2, 2)\in R\times S$ e $(1, 2)\in R\times S$. Temos:
    $$(2, 2)+(1, 2)=(2+ 1, 2+ 2)=(0, 0)$$
    $$(2, 2)\cdot (2, 2)=(2\cdot 2, 2\cdot 2)=(1, 0)$$

\chapter{Noções de Grupos}


\section{Definição e Propriedades Básicas}

Grupos são estruturas matemáticas munidas de uma operação binária com algumas propriedades espeiciais.
O principal objetivo deste texto é servir como texto para um estudo introdutório sobre anéis e corpos, que são estruturas matemáticas que possuem duas operações binárias com propriedades especiais.
Conforme veremos no Capítulo 3, todo anel e todo corpo, com uma dessas operações, forma um grupo.
Assim, é útil, para o estudo de anéis e corpos, o conhecimento de noções básicas sobre grupos.

Apesar das noções de anel e de corpo serem, a nível de definição, noções mais complexas que a de grupo, a noção de grupo, em parte por ser menos restritiva, necessita o desenvolvimento de ferramentas específicas para seu estudo completo.
A área do conhecimento matemático resultante do desenvolvimento dessa teoria é extremamente rica, e chamada de \emph{Teoria dos Grupos}.
Nosso objetivo, por outro lado, é focar no estudo inicial das teorias de anéis e corpos, e, portanto, não mergulharemos nesta importante área.

Assim, não é objetivo deste capítulo apresentar uma introdução ao estudo de grupos, mas sim apenas introduzir as noções e resultados básicos próprios de grupos que são estritamente necessários para os resultados envolvendo anéis e corpos descritos no restante do texto.

\begin{definition}
Um grupo é uma quadrupla $(G,\cdot,e)$, tal que $G$ é um conjunto, $\cdot$ é uma operação binária em $G$ e $0 \in G$, e satisfazem:

\begin{itemize}
    \item (\textbf{Propriedade associativa}) $\forall a, b, c \in G$ $(a \cdot b) \cdot c = a \cdot (b \cdot c)$.
    \item (\textbf{Elemento neutro}) $\forall a \in G$  $e \cdot a = a \cdot e = a$.
    \item (\textbf{Elemento inverso}) $\forall a \in G$ $\exists b \in G$ $a \cdot b = b \cdot a = e$.
\end{itemize}
Se, adicionalmente, a seguinte propriedade é satisfeita, o grupo é chamado de \emph{comutativo}, ou, mais comunmente, \emph{Abeliano}:
\begin{itemize}
    \item (\textbf{Comutatividade}) $\forall a, b \in G\, a \cdot b = b \cdot a$.
\end{itemize}
\end{definition}

Alguns exemplos:

\begin{exemplo}Abaixo, exemplificamos alguns grupos importantes.
    \begin{enumerate}[label=\alph*)]
        \item Com a soma usual e $0$, $\mathbb{Z, Q, R, C}$ são grupos Abelianos.
        \item Com a multiplicação usual, o círculo unitário complexo $\mathbb T=\{x \in \mathbb C: |x|=1\}$ é um grupo Abeliano com elemento neutro $1$.
        De fato, o produto de complexos é comutativo, associativo e tem $1$ como elemento neutro.
        Note que $1\in \mathbb T$ e $0\notin \mathbb T$.
        Se $x \in \mathbb T$, o inverso multiplicativo de $x$ é dado por $\frac{\bar x}{|x|^2}$, onde $\bar x$ denota o conjugado de $x$.
        Como $|\bar x|=|x|=1$, segue que $\mathbb T$ tem todos os inversos de todos seus elementos.
        \item Os inteiros módulo $n$ ($n\geq 1$), dados por $\mathbb Z_n=\{0, \dots, n-1\}$ com a soma dada pela aritmética módulo $n$, são grupos.
        \item Se $X$ é um conjunto qualquer, o conjunto das bijeções de $X$ em $X$ é, com a composição usual de funções e a identidade, um grupo, cuja operação inversa é a inversão usual de funções.
        Tal grupo é denominado \emph{grupo de permutações de $X$}.
        
        Caso $X$ tenha ao menos $3$ elementos, ele não é abeliano: sendo $a, b, c$ três elementos distintos de $X$, sendo $f$ a função que permuta $a, b$ e fixa os demais elementos, $g$ a que permuta $b, c$ temos que $f\circ g(a)=f(a)=b$, mas $g\circ f(a)=g(b)=c$, logo, $f\circ g\neq g\circ f$.
    \end{enumerate}
\end{exemplo}

Algumas observações importantes sobre a notação utilizada no estudo de grupos:
\begin{itemize}
\item Ao discursar sobre grupos, é comum omitir a operação e o elemento neutro, referindo-se apenas ao conjunto $G$, conforme fizemos acima ao mencionar que $\mathbb Z$ é um grupo.
O mais formal, porém muito menos usual, feito principalmente em situações em há chance de confusão, é escrever que, por exemplo, $(\mathbb Z, +, 0)$ é um grupo.
\item Como também ocorre com $\mathbb Z$, caso o grupo seja Abeliano, é comum que sua operação binária seja denotada por $+$ ou outro símbolo similar.
Nesse contexto, o elemento neutro é frequentemente denotado por $0$.
\item Caso o grupo em discurso não seja necessariamente Abeliano, é comum que sua operação binária seja denotada por $\cdot$ ou outro símbolo similar.
Nesse contexto, o elemento neutro é frequentemente denotado por $e$, e a operação é frequentemente omitida, ou seja, $a \cdot b$ é frequentemente escrito como $ab$.

Porém, há grupos Abelianos cujas operações também são denotadas por $\cdot$, como no caso o grupo $\mathbb T$ mencionado acima.
\end{itemize}


Agora iniciaremos a provar algumas propriedades básicas sobre grupos.
\begin{prop}[Unicidade do elemento neutro]\label{prop:group_uniqueNeutral}
    Seja $(G,\cdot,e)$ um grupo.
    Então, o elemento neutro $e$ é único.
    Isto é, se $h \in G$ é tal que $\forall a \in G$ $h \cdot a = a \cdot h = a$, então $h = e$.
\end{prop}
\begin{proof}
    Note que $h=he$, pois $e$ é elemento neutro.
    Por outro lado, $e=he$, pois $h$ é elemento neutro.
    Assim, $h=he=e$.
\end{proof}

\begin{prop}[Unicidade dos inversos]\label{prop:group_uniqueInverse}
    Seja $(G,\cdot,e)$ um grupo.
    Então todo $a \in G$ possui um único elemento inverso.
Isto é $\forall a \in G$ $\exists!\, b \in G$ $a \cdot b = b \cdot a = e$.
\end{prop}
\begin{proof}
    A existência do inverso é garantida pela definição de grupo.
    
    Para provar a unicidade, suponha que $b, c$ são inversos de $a$, ou seja, que $a \cdot b = b \cdot a = e$ e $a \cdot c = c \cdot a = e$.
    Segue que:
    $$b=be=b(ac)=(ba)c=ec=c.$$
\end{proof}

A unicidade dos inversos nos permite definir a notação $a^{-1}$ para o inverso de $a$ em um grupo $(G,\cdot,e)$.
Caso $(G, +, 0)$ seja um grupo Abeliano, a notação $-a$ é frequentemente utilizada para denotar o inverso de $a$, e, nesse caso, $-a$ é chamado de \emph{oposto} de $a$.

Note que assim, ficam definidos operadores unários $(\,)^{-1}:G\rightarrow G$ (ou $-:G\rightarrow G$).
Para o segundo caso, define-se também que $a-b=a+(-b)$.

\begin{prop}[Cancelamento]\label{prop:group_cancel}
    Seja $(G,\cdot,e)$ um grupo e $a,b,c \in G$.
    Se $a \cdot b = a \cdot c$, então $b=c$.
    Analogamente, se $b \cdot a = c \cdot a$, então $b=c$.
\end{prop}
\begin{proof}
    Provaremos a primeira afirmação.
    A segunda é análoga e fica como exercício.
    Suponha que $ba=ca$.
    Segue que $(ba)a^{-1}=(ca)a^{-1}$.
    
    Pela propriedade associativa, $b(aa^{-1})=c(aa^{-1})$.

    Pela definição de inverso, segue que $be=ce$.

    Pela neutralidade de $e$, segue que $b=c$.
\end{proof}

\begin{corol}[Cancelamento II]\label{prop:group_cancelII}
    Seja $(G,\cdot,e)$ um grupo.
    Para todos $a, b \in G$, se $ab=a$, então $b=e$.
Analogamente, se $ba=a$, então $b=e$.
\end{corol}
\begin{proof}
    Para a primeira afirmação, note que $ab=ae$, logo, pela proposição anterior, $b=e$.
    Para a segunda afirmação, note que $ba=ea$, logo, pela proposição anterior, $b=e$.
\end{proof}

\begin{prop}[Regras de sinal]\label{prop:regraSinal}
    Seja $G$ um grupo e $a, b \in G$.
    Então:
    \begin{enumerate}[label=\alph*)]
        \item $((a)^{-1})^{-1}=a$ [na notação aditiva, $-(-a)=a$].
\label{prop:regraSinal_A}
        \item $(ab)^{-1}=b^{-1}a^{-1}$ [na notação aditiva, $-(a+b)=(-b)+(-a)]$.\label{prop:regraSinal_B}
        \item $e^{-1}=e$ [na notação aditiva, $-0=0$].\label{prop:regraSinal_C}
    \end{enumerate}
\end{prop}
\begin{proof}
    \ref{prop:regraSinal_A}: Temos que $(a^{-1})^{-1}a^{-1}=e=aa^{-1}$.
    Cancelando $a^{-1}$, segue.
    
    \ref{prop:regraSinal_B}: Temos que $(ab)^{-1}(ab)=e=(b^{-1}a^{-1})ab$.
    Cancelando $ab$, segue que $(ab)^{-1}=b^{-1}a^{-1}$.
    Analogamente, $(ba)^{-1}=a^{-1}b^{-1}$.

    \ref{prop:regraSinal_C}: Temos que $(e^{-1})e=e=ee$.
    Cancelando $e$ à direita, segue.


\end{proof}

\section{Somatórios}

Nessa seção, formalizaremos a noção de somatório.
É desejável que o leitor já possua familiaridade com alguma notação de somatório, não sendo nosso objetivo fornecer ao leitor um primeiro contato. Aqui apresentaremos a notação e as técnicas de ``substituição de variáveis'' que serão utilizadas.

\begin{definition}[Soma de família finita]\label{def:group_sum}
Seja $G$ um conjunto munido de uma operação $+$ associativa, comutativa e com neutro $0$.
Define-se, recursivamente para $n\geq 0$, o somatório de famílias $(a_i: i \in F)$, onde $F$ é um conjunto de $n$ índices e $a_i \in G$ para todo $i \in F$, como se segue:

\begin{itemize}
    \item \textbf{Notação:} se $a=(a_i)_{i\in F}$ é uma sequência de elementos de $G$, então usamos as notações:
    \[\sum a=\sum(a_i: i\in F)=\sum_{i\in F} a_i.\]
    \item Caso base $n=0$ (soma vazia): só existe uma família com $0$ elementos, que é a família vazia $a=()=\emptyset=(a_i:i\in \emptyset)$.
    Definimos: \[\sum a=\sum_{i \in \emptyset}a_i=0\].
    \item Passo recursivo $n\rightarrow n+1$: considere uma família $(a_i)_{i\in F}$, onde $|F|=n+1$.
    Define-se:
    \[\sum(a_i: i \in F)=\sum(a_i: i \in F\setminus\{j\})+a_j,\]
    onde $j \in I$ é qualquer elemento.
\end{itemize}
\end{definition}
É claro que, para mostrar que a definição acima é consistente, precisamos mostrar que a soma não depende da escolha de $j$.

\begin{lemma}
Qualquer que seja o tamanho (finito) de $F$, $\sum(a_i)_{i\in F}$ está bem definido.
\end{lemma}

\begin{proof}
    Seja $F$ um conjunto finito.
Se $|F|=0$, então $F=\emptyset$, e a soma é $0$.
Se $|F|=1$, então $F=\{j\}$ -- só há uma escolha para $j$, e a soma é $a_j$.
    Se $|F|=n+1$ para $n\geq 1$, tome $j, k \in F$.
    Devemos ver que $\left(\sum_{i\in F\setminus\{j\}} a_i\right)+a_j=\left(\sum_{i\in F\setminus\{k\}} a_i\right)+a_k$.
    Com efeito:

    \[\left(\sum_{i\in F\setminus\{j\}} a_i\right)+a_j=\left(\left(\sum_{i\in F\setminus\{j, k\}} a_i\right)+a_k\right)+a_j=\left(\sum_{i\in F\setminus\{j, k\}} a_i\right)+(a_k+a_j)\]

    \[=\left(\sum_{i\in F\setminus\{j, k\}} a_i\right)+(a_j+a_k)=\left(\left(\sum_{i\in F\setminus\{j, k\}} a_i\right)+a_j\right)+a_k=\left(\sum_{i\in F\setminus\{k\}} a_i\right)+a_k.\]
\end{proof}
Como fazemos no cálculo de integrais, muitas vezes é desejável utilizar técnicas de substituição de variáveis para calcular ou simplificar somatórios.
A proposição abaixo formaliza esta técnica.
\begin{prop}[Mudança de variável em somatório]\label{prop:group_sumVarChange}
    Seja $G$ um conjunto munido de uma operação $+$ associativa, comutativa e com neutro $0$.
Seja $(a_i: i \in I)$ uma família finita em $G$ e $\phi:J\rightarrow I$ uma função bijetora.
Então:

    \[\sum_{i \in I}a_i=\sum_{j \in J}a_{\phi(j)}.\]

\end{prop}
\begin{proof}
Novamente, procedemos por indução no tamanho de $n=|I|$.
A base de tamanho $0$ é trivial, já que ambos os lados da igualdade são $0$.

Para o passo indutivo em que $|I|=|J|=n+1$, considere $\phi:J\rightarrow I$ como no enunciado.
Fixe $k \in J$ qualquer e sejam $I'=I\setminus\{\phi(k)\}, J'=J\setminus\{k\}$ e $\phi'=\phi|_{J'}:J'\rightarrow I'$, que é bijetora.
Como $|J'|=|I'|=n$, por hipótese indutiva temos que $\sum_{j \in J'}a_{\phi(j)}=\sum_{i \in I'}a_i$.
Segue que:

\[\sum_{j \in J}a_{\phi(j)}=\left(\sum_{j \in J'}a_{\phi(j)}\right)+a_{\phi(k)}=\left(\sum_{i \in I'}a_{i}\right)+a_{\phi(k)}=\sum_{j \in I}a_{i}.\]
\end{proof}

Também podemos juntar, sempre que necessário, duas somas disjuntas sob um único sinal de somatório.
\begin{prop}[Concatenação de somatórios]\label{prop:group_sumConcat}
    Seja $I, J$ conjuntos disjuntos. Considere em $G$ famílias $(a_i: i \in I)$ e $(a_i: i \in J)$, e a família $(a_i: i \in I\cup J)$.
    Vale a relação:

    \[\sum_{i \in I}a_i+\sum_{i \in I}a_i=\sum_{i \in I\cup J}a_i\]
\end{prop}
\begin{proof}
    Provaremos por indução no tamanho de $J$.
    Se $J=\emptyset$, temos $I\cup J=J$ e $\sum_{i \in J}a_i=0$, logo, segue a tese.

    Se a proposição vale para todo $J$ de tamanho $n$, suponha que $|J|=n+1$ e seja $J'=J\setminus \{j'\}$, onde $j' \in J$ é qualquer elemento arbitrário.

    Então:
    \[\sum_{i \in I}a_i+\sum_{i \in J}a_i=\left(\sum_{i \in I}a_i+\sum_{i \in J'}a_i\right)+a_{j'}=\sum_{i \in I\cup J'}a_i+a_{j'}=\sum_{i \in I\cup J}a_i.\]
\end{proof}

Antes de enunciar a próxima proposição, precisamos falar sobre notações de índices duplos.
Se $K$ é um conjunto que consiste apenas de pares ordenados $(i, j)$, não é incomum encontrarmos na literatura a notação $\sum_{(i, j)\in K}a_{ij}$.
bem como encontrar a notação $(a_{ij}:(i, j)\in K))$.

Tais notações são usadas no contexto no qual uma para todo $i, j$ com $(i, j)\in K$, está definido um elemento $a_{ij}$.
Para se encaixar no nosso formalismo, $(a_{ij}:i, j \in K)$ denota $(a_{\pi_1(k)\pi_2(k)}: k \in K)$, e $\sum_{(i, j)\in K}a_{ij}$ denota $\sum_{k \in K}a_{\pi_1(k)\pi_2(k)}$, onde $\pi_1$ e $\pi_2$ são as funções coordenadas de pares ordenados.

\begin{prop}[Concatenação de somatórios II]\label{prop:group_sumConcatII}
    Seja $I$ um conjunto não vazio e, para cada $i \in F$, seja $F_i$ um conjunto.
    Considere o conjunto finito $K=\bigcup_{i \in I}\{i\}\times F_i=\{(i, j): i \in I, j \in F_i\}$.

    Para cada $i \in I$, considere uma família de elementos de $G$, $(a_{ij}: j \in F_i)$.
    Considere também as famílias $(a_{ij}: (i, j)\in K)=(a_{ij}: i\in I, j \in F_i)$ e $(\sum_{j \in F_i}: i\in I)$. Vale a relação:

    \[\sum_{i \in I}\left(\sum_{j \in F_i}a_{ij}\right)=\sum_{(i, j)\in K}a_{ij}\]
\end{prop}
\begin{proof}
    Provaremos por indução no tamanho de $I$.
    Se $I=\{i'\}$ é unitário, temos que $K=\{(i', j): j \in F_{i'}\}$.
    Segue que $\phi:F_{i'}\rightarrow K$ dada por $\phi(j)=(i', j)$ é bijetora.
    Assim, pela proposição \ref{prop:group_sumVarChange}, temos que:

    \[\sum_{i \in I}\left(\sum_{j \in F_i}a_{ij}\right)=\left(\sum_{j \in F_{i'}}a_{i'j}\right)=\sum_{(i, j)\in K}a_{ij}\]

    Justificando melhor a última igualdade:
    \[\sum_{j \in F_{i'}}a_{i'j}=\sum_{j \in F_{i'}}a_{\pi_1(\phi(j))\pi_2(\phi(j))}=\sum_{k \in K}a_{\pi_1(k)\pi_2(k)}=\sum_{(i, j)\in J}a_{ij}.\]
    
    Agora provaremos o passo indutivo.

    Suponha que $|I|=n+1$ e que a hipótese vale para todo $I$ de tamanho $n$. Fixe qualquer $i' \in I$ e considere $I'=I\setminus \{i'\}$.

    Seja $K'=\{(i', j): j \in F_{i'}\}$. e $\hat K=\{(i, j): i \in I', j \in F_i\}$.
    Segue que $K$ é a união disjunta de $\hat K$ e $K'$.
    Assim, pela Proposição \ref{prop:group_sumConcat}, pelo caso base, e pelo passo indutivo, segue que:
    \[\sum_{i \in I}\left(\sum_{j \in F_i}a_{ij}\right)=\left(\sum_{i \in I'}\sum_{j \in F_{i}}a_{ij}\right)+\left(\sum_{j \in F_{i'}}a_{i'j}\right)=\sum_{(i, j)\in \hat K}a_{ij}+\sum_{(i, j)\in K'}a_{ij}=\sum_{(i, j)\in K}a_{ij}\]
\end{proof}

Agora provaremos a comutação de somatórios.

\begin{prop}\label{prop:group_sumCommut}
    Sejam $I, J$ conjuntos não vazios e considere uma família $(a_{ij}:(i, j)\in I\times J)=(a_{ij}:i \in I, j \in J)$.

    Para cada $i \in I$, considere a família $(a_{ij}: j \in J)$, e, para cada $j \in J$, considere a família $(a_{ij}: i \in I)$.

    Então:

    \[\sum_{i\in I}\sum_{j \in J}a_{ij}=\sum_{j \in J}\sum_{i \in I}a_{ij}\]
\end{prop}
\begin{proof}
    Seja $K=I\times J$ e $K'=J\times I$.
    Temos que $\phi:I\times J\rightarrow J\times I$ dada por $\phi(i, j)=(j, i)$ é uma bijeção entre $K$ e $K'$. Escrevendo de outra forma, $\phi(k)=(\pi_2(k), \pi_1(k))$. Note ainda que $\phi^2=\id_K$.

    Assim, pela Proposição \ref{prop:group_sumVarChange} e Proposição \ref{prop:group_sumConcatII}, temos que:

    \[\sum_{j\in J}\sum_{j \in J}a_{ij}=\sum_{(i,j)\in I\times J}a_{ij}=\sum_{k\in K}a_{\pi_1(\phi^2(k))\pi_2(\phi^2(k))}\]
    \[=\sum_{k \in K'}a_{\pi_1(\phi(k))\pi_2(\phi(k))}=\sum_{k \in K'}a_{\pi_2(k)\pi_1(k)}=\sum_{(j, i)\in J\times I}a_{ij}=\sum_{j \in J}\sum_{i \in I}a_{ij}.\]

\end{proof}
\section{Exercícios}
\begin{exer}
    Suponha que $a$, $b$ e $c$ sejam elementos de um anel $A$, e que $a$ não é divisor de $0$.
    
    Mostre que se $ab = ac$, então ou $a = 0$ ou $b = c$ (isto é, se $a\neq 0$, podemos cancela-lo).
\end{exer}

\chapter{Anéis e subanéis}
Nesta seção, iniciaremos o estudo dos anéis e de estruturas relacionadas. Apresentaremos as definições dessas estruturas e suas propriedades mais elementares.

\section{A definição de anel}
No Capítulo 2, conhecemos, por alto, a definição de grupo.
Um grupo é um conjunto munido de uma operação binária que satisfaz algumas propriedades.
Ele pode ser Abeliano ou não Abeliano, e, quando é Abeliano, lembra-nos da adição de inteiros.
Porém, estururas como inteiros, racionais e reais não são apenas grupos Abelianos, pois possuem também outra operação binária -- a multiplicação.
Esta operação se relaciona com a soma através das propriedades distributivas.

A noção de anel visa capturar parte desas ideias, de modo a generalizar o estudo das estruturas citadas acima.
\begin{definition}[Anel]
    Um anel é uma $4$-upla $(A, +, \cdot, 0, 1)$ conjunto $A$ com duas operações binárias, adição e multiplicação, denotadas por $+$ e $\cdot$, tais que:
    \begin{itemize}
        \item $(A, +, 0)$ é um grupo abeliano.
        \item (\textbf{Associatividade}) Para todo $a, b \in A$, temos $(a \cdot b)\cdot c = a\cdot(b\cdot c)$.
        \item (\textbf{Elemento identidade}) $\forall a \in A$ $1 \cdot a = a \cdot 1 = a$.
        \item (\textbf{Propriedades distributivas}) Para todos $a, b, c, \in A$, temos:
        \begin{align*}
            a \cdot (b + c) &= a \cdot b + a \cdot c, \text{ e}\\
            (a + b) \cdot c &= a \cdot c + b \cdot c
        \end{align*}
    \end{itemize}
    Se, adicionalmente, a seguinte propriedade é satisfeita, o anel é chamado de \emph{comutativo}.
    \begin{itemize}
        \item (\textbf{Comutatividade}) $\forall a, b \in A$ $a \cdot b = b \cdot a$.
    \end{itemize}
\end{definition}

Algumas observações:
\begin{itemize}
    \item Como em grupos, ao discursar sobre anéis é comum omitir as operações, referindo-se apenas ao conjunto $A$.
    \item Ao discursar sobre anéis, e a exemplo do que foi feito ao enunciar as propriedades distributivas, são utilizadas as convenções usuais sobre precedência de operações envolvidas por parênteses.
    Assim, $a + b \cdot c$ é interpretado como $a + (b \cdot c)$.
    \item Há textos que definem anéis sem incluir o elemento identidade $1$.
    Nestes textos, a definição acima dá nome ao que chamam de \emph{anéis com identidade}, ou \emph{anéis com 1}.
    Nesse curso, não usaremos essa convenção, de modo que \textbf{todos nossos anéis possuem identidade}.
    De modo similar, alguns textos definem anéis como sendo comutativos. Também não adotaremos essa convenção.
    \textbf{Os nossos anéis podem ser não comutativos}.
    \item A definição de anel não exige que $0=1$.
    \item $0$ é chamado de elemento nulo, e $1$ de elemento identidade.
\end{itemize}

\begin{prop}[Propriedade multiplicativa do $0$]
    Seja $A$ um anel.
    Então $\forall a \in A$ $0 \cdot a = a \cdot 0 = 0$.
\end{prop}
\begin{proof}
Provaremos a primeira afirmação.
A segunda é análoga e fica como exercício.

Temos que $0\cdot a=(0+0)\cdot a=0\cdot a +0\cdot a$.
Cancelando, segue que $0=0\cdot a$.
\end{proof}

\begin{prop}[Anel trivial]
    Seja $A={x}$ um conjunto qualquer.
    Defina $x\cdot x=x=x+x=0=1$.
    Então $(A, +, \cdot, 0, 1)$ é um anel.
    Um anel dessa forma é chamado de \emph{anel trivial}.
    
    Além disso, se $A$ é um anel tal que $0=1$, então $A$ é um anel trivial.
\end{prop}
\begin{proof}
    A primeira afirmação (de que $A$ como acima é um anel) fica como exercício.

    Para a segunda afirmação, assuma que $A$ é um anel tal que $0=1$.
    Fixe $a \in A$ qualquer.
    Então $a=a\cdot1=a\cdot0=0$, ou seja, $a=0$.
    Assim, $A$ é o conjunto unitário $\{0\}$, que é um anel trivial.
\end{proof}

Todo anel satisfaz as conhecidas regras de sinais referentes à multiplicação e adição, como:
\begin{prop}[Regras de sinal II]\label{prop:regraSinal2}
    Seja $A$ um anel e $a, b \in A$. Então:
    \begin{enumerate}[label=\alph*)]
        \item $(-a)b=a(-b)=-(ab)$\label{prop:regraSinal2_A}
        \item $(-a)(-b)=ab$.\label{prop:regraSinal2_B}
        \item $(-1)a=-a$.\label{prop:regraSinal2_C}
    \end{enumerate}
\end{prop}
\begin{proof}
    \ref{prop:regraSinal2_A}: Temos que $ab+(-a)b=(-a)b+ab=[-a+a]b=0b=0$. Assim, $(-a)b=-(ab)$. Analogamente, $a(-b)=-(ab)$.

    \ref{prop:regraSinal2_B}: Temos que $(-a)(-b)=-[a(-b)]=-[-(ab)]=ab$ pela regra anterior.

    \ref{prop:regraSinal2_C}: Temos que $(-1)a=-(1a)=-a$.
\end{proof}

\section{Elementos invertíveis}
Um anel, com sua soma, é um grupo Abeliano, e, portanto, possui opostos aditivos. Porém, não necessita possuir opostos multiplicativos. Os elementos de um anel que possuem inversos no anel são os chamados \emph{elementos invertíveis} ou \emph{unidades}.
\begin{definition}[Elemento invertível]
    Seja $A$ um anel.
    Um elemento $a \in A$ é dito \emph{invertível}, ou uma \emph{unidade} se $\exists b \in A$ tal que $a \cdot b = b \cdot a = 1$.
    
    O conjunto de todas das unidades de $A$ é denotado por $A^*$.
\end{definition}

\begin{definition}
    Seja $A$ um anel.
    Então, se $a \in A^*$, existe um \textbf{único} $b \in A$ tal que $a \cdot b = b \cdot a = 1$. Este elemento é denotado por $a^{-1}$, e é chamado de \emph{inverso} de $a$.
\end{definition}

Observação: para que a definição acima faça sentido, é necessário mostrar que se $a$ é unidade, existe um \textbf{único} $b \in A$ tal que $a \cdot b = b \cdot a = 1$.
A existência é garantida pela definição de unidade, e a demonstração da unicidade é análoga à da unicidade do inverso em grupos (Proposição \ref{prop:inverso_unico_grupo}), ficando como exercício.

\begin{prop}
Seja $A$ um anel. Para todos $a, b \in A^*$, temos:
\begin{enumerate}[label=\alph*)]
    \item $ab\in A^U$ e $(ab)^{-1}=b^{-1}a^{-1}$.\label{prop:unidadeProduto_a}
    \item $a^{-1}\in A^U$ e $(a^{-1})^{-1}=a$.\label{prop:unidadeProduto_b}
    \item $1^{-1}=1$.\label{prop:unidadeProduto_c}
\end{enumerate}
Além disso, $A^*$ é, com a restrição da operação de multiplicação do anel, um grupo com identidade $1$. Caso $A$ é um anel comutativo, $A^*$ é um grupo abeliano.
\end{prop}
\begin{proof}
    \ref{prop:unidadeProduto_a}: Sejam $a, b \in A^*$. Pela associatividade, $(ab)(b^{-1}a^{-1})=1=(b^{-1}a^{-1})(ab)$, logo, pela unicidade do inverso, $(ab)^{-1}=b^{-1}a^{-1}$.

    \ref{prop:unidadeProduto_b}: Seja $a \in A^*$. Temos que $a^{-1}a=1=a(a^{-1})$, logo, pela unicidade do inverso, $(a^{-1})^{-1}=a$.

    \ref{prop:unidadeProduto_c}: Note que $1\cdot 1=1=1\cdot 1$, logo, pela unicidade do inverso, $1^{-1}=1$.

    Se $A$ é um anel comutativo, então $A^*$ é um grupo abeliano, pois para todo $a, b \in A^*$, temos que $ab=ba$, logo $(ab)^{-1}=b^{-1}a^{-1}=a^{-1}b^{-1}$.
\end{proof}

\section{Mais estruturas}

Abaixo, segue a definição de anel de divisão e corpo.
A noção de corpo será uma das noções mais importantes deste texto.
\begin{definition}[Anel de divisão]
Um \emph{anel de divisão} é um anel não trivial para o qual todo elemento não nulo é invertível.
Um \emph{corpo} é um anel de divisão comutativo.
\end{definition}

\begin{exer}
    Mostre que um anel $A$ é um anel de divisão se, e somente se $A^*=A\setminus\{0\}$.
\end{exer}
\begin{definition}
    Um domínio de integridade é um anel comutativo não trivial $A$ tal que $\forall a, b \in A$, se $ab=0$, então $a=0$ ou $b=0$.
\end{definition}

\begin{prop}
    Seja $K$ um corpo.
    Então $K$ é um domínio de integridade.
\end{prop}
\begin{proof}
Sabemos que $K$ é um anel comutativo não trivial.
Sejam $a, b \in K$ tais que $ab=0$.
Se $a=0$, então segue a tese.
Caso contrário, como $K$ é um corpo, $a^{-1}$ existe.
Assim, temos que $b=(a^{-1}a)b=a^{-1}(ab)=0$, logo, $b=0$.
\end{proof}

\section{O anel dos números inteiros}
[SUBSEÇÃO EM CONSTRUÇÃO]

Espera-se que o estudante já possua traquejo com o anel dos números inteiros, incluindo contato com a noção formal de divisibilidade, o teorema fundamental da aritmética e a noção de congruência módulo $n$.

Assumiremos, sem demonstração (por fugir do escopo do texto), que existe uma estrutura $\mathbb Z=(\mathbb Z,+,\cdot,0,1, \leq)$ como abaixo:

\begin{definition}[Inteiros, anel ordenado]
$\mathbb Z=(\mathbb Z,+,\cdot,0,1, \leq)$ é uma estrutura tal que $\mathbb Z=(\mathbb Z,+,\cdot,0,1)$ é um \emph{anel ordenado} cujos elementos positivos são bem ordenados.
\end{definition}




\section{Subanéis}
Em Matemática, é comum que as estruturas estudadas possuam uma noção de subestrutura.
Em geral, uma subestrutura de uma estrutura data é um subconjunto desta que seja, de forma natural, uma estrutura da mesma natureza daquela.

Veremos que, quando tratamos de anéis, nem todo subconjunto pode ser visto como uma subestrutura.

\begin{definition}[Subanel]
    Seja $A$ um anel e $B \subseteq A$. Dizemos que $B$ é subanel de $A$ se, e somente se $(B, +|_{B^2}, \cdot|_{B^2}, 0_A, 1_A)$ é um anel, onde $+|_{B^2}:B^2\rightarrow B$ e $\cdot|_{B^2}:B^2\rightarrow B$ são as restrições das operações de $A$ à $B^2$.
\end{definition}

Na definição acima, estamos pedindo que $B$ seja um subconjunto de $A$ que possua as mesmas operações que $A$, e que essas operações sejam restritas a $B$ e satisfaçam todas as cláusulas da definição de anel. Aparentemente, na prática, provar que um dado subconjunto de $A$ é um subanel pode parecer uma tarefa longa. Porém, a seguinte proposição encurta esta tarefa significativamente: 

\begin{definition}[Subanel]
    Seja $A$ um anel e $B\subseteq A$. Então $B$ é um subanel de $A$ se, e somente se:
    \begin{itemize}
        \item $1_A \in B$
        \item Para todos $a, b \in B$, $a-b \in B$.
        \item Para todos $a, b \in B$, $ab\in B$
    \end{itemize}

    Além disso, caso $B$ seja um subanel de $A$, os opostos aditivos de $B$ são os mesmos que os de $A$, ou seja, que $-b \in B$ para todo $B \in B$.
\end{definition}

\begin{proof}
    Primeiro, notemos suponhamos que $B$ seja um subanel de $A$. Então $B$ é fechado por $+, \cdot$ e $1_A\in B$. Resta apenas ver que para todos $a, b \in B$, $a-b \in B$.
    Como $B$ é fechado por soma, basta provar a última afirmação: que para todo $b \in B$, $-b \in B$.
    Fixe $b \in B$. Como $(B, +|_B^2, 0_A)$ é um grupo abeliano, existe $x \in B$ tal que $b+x=0_B$. Então, em $a$, segue que $b+x=x+b=0_A$. Pela unicidade dos opostos em $A$, segue que $-b=x\in B$.

    Reciprocamente, provaremos que se $B$ possui $1_B$ como elemento e é fechado por diferença e por produto, então $B$ é um subanel de $A$. Iniciaremos verificando que $B$ é fechado por soma, por opostos e que tem $0_A$ como elemento.

    Como $1_A$ é elemento de $B$, temos que $0_A=1_A-1_A\in B$. Assim, $B$ possui $0_A$ como elemento. Agora, dado $b \in B$, $0_A-b=-b \in B$, o que mostra que $B$ é fechado por opostos. Finalmente, dados $a, b \in B$, $a-(-b)=a+b\in B$, o que mostra que $B$ é fechado para soma.

    As propriedades associativas, comutativas, distributivas e de identidade valem em $B$, pois valem em $A$ e as operações de $B$ são as mesmas de $A$, restritas. Para finalizar, basta observar que dado $a \in B$, $(-a)\in B$, como já mostrado, e que $a+(-a)=(-a)+a=0_A$, o que mostra que $B$ possui opostos aditivos.
\end{proof}

\begin{exemplo}
$\mathbb N$ não é um subanel de $\mathbb Z$, pois $-1 \notin \mathbb Z$.
Porém, note que $\mathbb N$ tem $1$ e é fechado por soma e produto, o que mostra que na proposição anterior, a expressão $a-b$ não pode ser substituída por $a+b$.
\end{exemplo}

\begin{exemplo}[Subanel trivial]
    Para todo $A$, temos que $A$ é subanel de si mesmo.
\end{exemplo}

\begin{exemplo}
O único subanel de $\mathbb Z$ é $\mathbb Z$: se $B$ é um subanel de $\mathbb Z$, então $0, 1 \in B$.
Por indução, para todo $n\geq 1$ temos que $n \in \mathbb B$: com efeito, $1\in B$, e, se $n \in B$, $n+1\in B$, logo vale o passo indutivo. Finalmente, $-n\in B$ para todo $n\geq 1$. Como $\mathbb Z=\{0\}\cup\{n \in \mathbb Z: n\geq 1\}\cup \{-n \in \mathbb Z: n\geq 1\}$, temos que $B=\mathbb Z$.
\end{exemplo}

Como as operações de um subanel são as mesmas de um anel, um subanel de um anel comutativo é comutativo.

\begin{prop}
    Subanéis de aneis comutativos são comutativos.
\end{prop}
\begin{proof}
Seja $A$ um anel comutativo e $B$ um subanel de $A$. Para todos $a, b \in B$, temos que o produto $a\cdot b$ em $B$ é dado pelo produto (comutativo) $a\cdot b$ em $A$, logo $a\cdot b=b\cdot a$.
\end{proof}

\chapter{Homomorfismos e Ideais}
Em matemática, boa parte das coleções de estruturas estudadas possui uma classe de funções que preservam, em algum sentido, suas propriedades.
O estudo generalizado destas estruturas é o que chamamos de \emph{teoria de categorias}, tema que não será tratado neste texto.
Na classe dos anéis, estas funções são o que chamamos de \emph{homomorfismos}.
\section{Definição de homomorfismo}
Homomorfismos são funções que preservam a estrutura de anéis.
Formalmente:
\begin{definition}
Sejam $A$, $R$ aneis.
Uma função $f:A\rightarrow R$ é um \emph{homomorfismo} se:
\begin{itemize}
    \item $f(a+b)=f(a)+f(b)$ para todo $a, b \in A$.
    \item $f(-a)=-f(a)$ para todo $a \in A$.
    \item $f(0_A)=0_R$
    \item $f(ab)=f(a)f(b)$ para todo $a, b \in A$.
    \item $f(1_A)=1_R$.
\end{itemize}

Caso $f$ seja injetora, dizemos que $f$ é um \emph{monomorfismo}.
Caso $f$ seja sobrejetora, dizemos que $f$ é um \emph{epimorfismo}.
Caso $f$ seja bijetora, dizemos que $f$ é um \emph{isomorfismo}.
\end{definition}

A noção de isomorfismo é extremamente importante na Teoria de Anéis. Muitas vezes, temos dois anéis que ``deveriam ser a mesma coisa'', mas, como objetos matemáticos, não são iguais. A noção de isomorfismo entra em campo para dizer que, mesmo que dois anéis não sejam o mesmo objeto, eles possuem exatamente as mesmas propriedades algébricas e operacionais. Para darmos um exemplo concreto:

\begin{exemplo}
Seja $A=\{0, 1\}$ e $R=\{Z, U\}$, onde $Z, U$ são objetos diferentes, e diferentes de $0, 1$. Defina em $A$ as operações $\cdot$ e $+$ dadas pelas seguintes tabelas:

Em $A$:
\begin{multicols}{2}\centering
    \begin{tabular}{c|cc}
        $+$ & 0 & 1 \\ \hline
        0 & 0 & 1 \\
        1 & 1 & 0 \\
    \end{tabular}

    \begin{tabular}{c|cc}
        $\cdot$ & 0 & 1 \\ \hline
        0 & 0 & 0 \\
        1 & 0 & 1 \\
    \end{tabular}
\end{multicols}

Em $R$:
\begin{multicols}{2}\centering
    \begin{tabular}{c|cc}
        $+$ & Z & U \\ \hline
        Z & Z & U \\
        U & U & Z \\
    \end{tabular}

    \begin{tabular}{c|cc}
        $\cdot$ & Z & U \\ \hline
        Z & Z & Z \\
        U & Z & U \\
    \end{tabular}
\end{multicols}

Intuitivamente, $A$ e $R$ correspondem a duas apresentações de uma mesma estrutura algébrica, porém, como $A\cap R=\emptyset$, estes dois anéis não são o mesmo anel.
Como formalizar este fato?
Ora, há uma relação biunívoca (uma bijeção) entre $A$ e $R$ que preserva suas operações, e ela é dada por $\phi(0)=Z$ e $\phi(1)=U$.
Tal $\phi$ é um isomorfismo.
\end{exemplo}

Para todos os fins que interessam à Álgebra, anéis isomorfos tem exatamente as mesmas propriedades, e, assim, são considerados como sendo, em algum sentido, a mesma estrutura.

A definição de homomorfismo, por possuir várias cláusulas, pode parecer de longa verificação.
A proposição abaixo encurta esta verificação substancialmente.

\begin{prop}
    Sejam $A, R$ anéis e $f:A\rightarrow R$ uma função.
    Então $f$ é um homomorfismo se, e somente se:
    \begin{itemize}
        \item $f(a+b)=f(a)+f(b)$ para todo $a, b \in A$.
        \item $f(ab)=f(a)f(b)$ para todo $a, b \in A$.
        \item $f(1_A)=1_R$.
    \end{itemize}
\end{prop}
\begin{proof}
    Provaremos o lado que não é imediatamente trivial.
    Começaremos mostrando que $f(0_A)=0_R$.
    Temos que $f(0_A)=f(0_A+0_A)=f(0_A)+f(0_A)$, logo, cancelando, $f(0_A)=0_R$.

    Agora, vejamos que $f(-a)=-f(a)$ para todo $a \in A$.
    Temos que $f(a)+f(-a)=f(a+(-a))=f(0_A)=0_R$, logo, $f(-a)=-f(a)$.

    Assim, $f$ é um homomorfismo.
\end{proof}

\section{Propriedades elementares}
\begin{lemma}
    Sejam $f:A\rightarrow R$ e $g:R\rightarrow S$ homomorfismos de anéis.
    Então a composição $g\circ f:A\rightarrow S$ é um homomorfismo de anéis.
\end{lemma}

\begin{proof}
    Sejam $a, b \in A$. Então:
    \begin{itemize}
        \item $g\circ f(a+b)=g(f(a+b))=g(f(a)+f(b))=g(f(a))+g(f(b))=(g\circ f)(a)+(g\circ f)(b)$.
        \item $g\circ f(ab)=g(f(ab))=g(f(a)f(b))=g(f(a))g(f(b))=(g\circ f)(a)(g\circ f)(b)$.
        \item $g\circ f(1_A)=g(f(1_A))=g(1_R)=1_S$.
    \end{itemize}
    Assim, $g\circ f$ é um homomorfismo de anéis.
\end{proof}

\begin{prop}[Propriedades de homomorfismos]
    Seja $f:A\rightarrow R$ um homomorfismo de anéis. Então:
    \begin{enumerate}[label=\alph*)]
        \item Para todo $a \in A^*$, temos $f(a)\in  R^*$ e $f(a^{-1})=f(a)^{-1}$. \label{prop:homomorfismo_a}
        \item A imagem de $f$, $\ran f=\{f(a): a \in A\}$, é um subanel de $R$. Se $A$ é comutativo, $\ran f$ também é.  \label{prop:homomorfismo_b}
        \item Se $f$ é injetora, a imagem de $f$ é um subanel de $R$ isomorfo a $A$. \label{prop:homomorfismo_c}
    \end{enumerate}
\end{prop}
\begin{proof}
\ref{prop:homomorfismo_a} Se $a \in A^*$, então $f(a)f(a^{-1})=f(aa^{-1})=f(1_A)=1_R$ e $f(a^{-1})f(a)=f(aa^{-1})=f(1_A)=1_R$. Assim, $f(a^{-1})=f(a)^{-1}$ e $f(a)\in R^*$.

\ref{prop:homomorfismo_b} Seja $a, b \in \ran f$. Então existem $x, y \in A$ tais que $a=f(x)$ e $b=f(y)$.
Assim, $a-b=f(x)-f(y)=f(x-y)$. Logo, $a-b \in \ran f$.
Similarmente, $ab=f(x)f(y)=f(xy)\in \ran f$, e $1_R=f(1_A)\in \ran f$.

Portanto, $\ran f$ é um subanel de $R$.
Se $A$ é comutativo, $\ran(f)$ também é comutativo, pois dados $a, b \in \ran f$, existem $x, y \in A$ tais que $a=f(x)$ e $b=f(y)$.
Assim, $ab=f(x)f(y)=f(xy)=f(yx)=f(y)f(x)=ba$.

\ref{prop:homomorfismo_c} Se $f$ é injetora, então $f$ é bijetora entre $A$ e $\ran f$. Assim, $f$ é um isomorfismo entre $A$ e $\ran f$, dado que é um homomorfismo.
\end{proof}

A noção de isomorfismo é uma relação de equivalência na classe dos anéis.

\begin{prop}[Propriedades de isomorfismo]
    Sejam $A, R, S$ anéis e $f:A\rightarrow R$ e $g:R\rightarrow S$ isomorfismos de anéis.
    Então:
    \begin{enumerate}[label=\alph*)]
        \item $g\circ f$ é um isomorfismo de anéis. \label{prop:isomorfismo_a}
        \item $f^{-1}:R\rightarrow A$ é um isomorfismo de anéis. \label{prop:isomorfismo_b}
        \item $\id_A:A\rightarrow A$ é um isomorfismo de anéis. \label{prop:isomorfismo_c}
    \end{enumerate}
\end{prop}

\begin{proof}
\ref{prop:isomorfismo_a} A composição de funções bijetoras é bijetora, e a composição de homomorfismos é homomorfismo.
Como um isomorfismo é um homomorfismo bijetor, segue que a composição de dois isomorfismos é um isomorfismo.

\ref{prop:isomorfismo_b} Como $f$ é um isomorfismo, $f$ é bijetora, assim, $f^{-1}:R\rightarrow A$ está bem definida e é bijetora. Verificaremos que $f^{-1}$ é um homomorfismo. Dados $r, s \in R$, sejam $a, b \in A$ tais que $f(a)=r$ e $f(b)=s$. Temos que:
\begin{itemize}
    \item $f^{-1}(r+s)=f^{-1}(f(a)+f(b))=f^{-1}(f(a+b))=a+b=f^{-1}(r)+f^{-1}(s)$.
    \item $f^{-1}(rs)=f^{-1}(f(a)f(b))=f^{-1}(f(ab))=a\cdot b=f^{-1}(r)f^{-1}(s)$.
    \item $f^{-1}(1_R)=f^{-1}(f(1_A))=1_A$.
\end{itemize}

\ref{prop:isomorfismo_c} A função identidade $\id_A$ é claramente bijetora, e é um homomorfismo, pois, para todos $a, b \in A$:
    \begin{itemize}
        \item $\id_A(a+b)=a+b=\id_A(a)+\id_A(b)$.
        \item $\id_A(ab)=ab=\id_A(a)\id_A(b)$.
        \item $\id_A(1_A)=1_A$.
    \end{itemize}
\end{proof}
\chapter{Quocientes e Teoremas do Homomorfismo}

Ao estudar o anel dos números inteiros, normalmente são estudadas as relações de congruência e, subsequentemente, os anéis quocientes $\mathbb Z_n=\mathbb Z/n\mathbb Z$.

Neste capítulo, estudaremos quocientes de anéis de forma generalizada, e suas relações com ideais, relações de congruência e homomorfismos de anéis.

\section{Relações de congruência}
As relações de congruência de anéis são relações que generalizam a noção de ``congruência módulo $n$'' do anel dos inteiros.

\begin{definition}
    Seja $A$ um anel. Uma relação de congruência em $A$ é uma relação de equivalência $\sim$ em $A$ que ``preserva operações''.
    Explicitamente, tal que para todos $a, b, c, d \in A$, se $a\sim b$ e $c\sim d$, então $a+c\sim b+d$ e $ac\sim bd$.
\end{definition}

Todo homomorfismo induz naturalmente uma relação de congruência.
Explicitamente:

\begin{prop}
Seja $f: A\rightarrow R$ um homomorfismo de anéis.
Então $\sim_f=\{(a, b) \in A^2: f(a)=f(b)\}$ é uma relação de congruência em $A$.
De outro modo, a relação $\sim_f$ em $A^2$ dada por $a \sim_f b$ se, e somente se $f(a)=f(b)$, é uma relação de congruência em $A$.
\end{prop}

\begin{proof}
    $\sim_f$ é uma relação reflexiva, pois para todo $a \in A$, $f(a)=f(a)$, logo, $a\sim_f a$.

    $\sim_f$ é simétrica, pois se $a\sim_f b$, então $f(a)=f(b)$, e, portanto, $f(b)=f(a)$, o que implica em $b\sim_f a$.

    $\sim_f$ é transitiva, pois se $a\sim_f b$ e $b\sim_f c$, então $f(a)=f(b)$ e $f(b)=f(c)$, logo, $f(a)=f(c)$, o que implica em $a\sim_f c$.

    $\sim_f$ preserva soma, pois se $a\sim_f b$ e $c\sim_f d$, então $f(a)=f(b)$ e $f(c)=f(d)$, logo, $f(a+c)=f(a)+f(c)=f(b)+f(d)=f(b+d)$, o que implica em $a+c\sim_f b+d$.

    $\sim_f$ preserva produto, pois se $a\sim_f b$ e $c\sim_f d$, então $f(a)=f(b)$ e $f(c)=f(d)$, logo, $f(ac)=f(a)f(c)=f(b)f(d)=f(bd)$, o que implica em $ac\sim_f bd$.
\end{proof}

A proposição abaixo classifica todas as relações de congruência a partir dos ideais de um anel.
\begin{prop}[Relações de congruência vs ideais]
    Seja $A$ um anel, $\mathcal R(A)$ o conjunto de todas as relações de congruência em $A$ e $\mathcal I(A)$ o conjunto de todos os ideais de $A$.
    Então, existe uma bijeção entre $\mathcal R(A)$ e $\mathcal I(A)$ dada por
    $\sim \mapsto I_{\sim}=\{a \in A: a\sim 0\}$,
    cuja inversa se dá por $I\mapsto \sim_I=\{(a, b) \in A^2: a-b \in I\}$.
\end{prop}
\begin{proof}
Primeiro, vejamos que se $\sim$ é uma relação de congruência, então $I_\sim$ é um ideal de $A$.

\begin{itemize}
\item $0 \in I_\sim$, pois $0\sim 0$.
\item Se $a, b \in I_\sim$, então $a\sim 0$ e $b\sim 0$, logo $a+b\sim 0+0=0$, portanto, $a+b \in I_\sim$.
\item Se $x \in A$ e $a \in I_\sim$, então $a\sim 0$ e $x\sim 0$, logo $ax\sim a0=0$ e $xa=0a=0$, portanto, $ax, xa \in I_\sim$.
\end{itemize}

Agora, vejamos que se $I$ é um ideal, então $\sim_I$ é uma relação de congruência. De fato, temos que, para todos $a, b, c, d \in A$:
\begin{itemize}
    \item $a\sim_I a$ pois $a-a=0\in I$.
    \item Se $a\sim_I b$, então $a-b \in I$, logo $(-1)(a-b)=b-a\in I$, e, portanto, $b\sim_I a$.
    \item Se $a\sim_I b$ e $b\sim_I c$, então $a-b \in I$ e $b-c \in I$, logo, $(a-b)+(b-c)=a-c \in I$, portanto, $a\sim_I c$.
    \item Se $a\sim_I b$ e $c\sim_I d$, então $a-b \in I$ e $c-d \in I$, logo, $(a-b)+(c-d)=(a+c)-(b+d)\in I$, portanto, $a+c\sim_I b+d$.
    \item Se $a\sim_I b$ e $c\sim_I d$, então $a-b \in I$ e $c-d \in I$, logo, $(a-b)c=ac-bc\in I$ e $b(c-d)=bc-bd\in I$, logo $(ac-bc)+(bc-bd)=ac-bd\in I$, portanto, $ac\sim_I bd$.
    \end{itemize}

Se $I$ é ideal, $I_{\sim_I}=I$, pois, para todo $a\in A$:

$$a\in I_{\sim_I}\Leftrightarrow a\sim_I 0\Leftrightarrow a-0\in I\Leftrightarrow a\in I.$$

Finalmente, se $\sim$ é relação de congruência, $\sim_{I_\sim}=\sim$, pois, para todos $a, b \in A$:

$$a\sim_{I_\sim} b\Leftrightarrow a-b\in I_\sim \Leftrightarrow a-b\sim 0\Leftrightarrow a\sim b.$$

Justificando a última equivalência: se $a-b\sim 0$, como $b\sim b$, temos que $a-b+b\sim b$, ou seja, que $a\sim b$. Reciprocamente, se $a\sim b$, como $(-b)\sim (-b)$, segue que $a+(-b)\sim b+(-b)$, ou seja, que $a-b\sim 0$.
\end{proof}

\begin{exemplo}
Como vimos, $\mathbb Z$ é um domínio de ideais principais.
Assim, todo ideal de $\mathbb Z$ é da forma $n\mathbb Z$.
Como para todo $n$, $n\mathbb Z=(-n)\mathbb Z$, temos que $\{n\mathbb Z: n\geq 0\}$ é a coleção de todos os ideais de $\mathbb Z$.

Quais são todas as relações de congruência em $\mathbb Z$?
Denotemos por $\sim_n$ a relação $\sim_{n\mathbb Z}$.

Temos que $\sim_0$ corresponde à relação de igualdade, pois $a\sim_0 b$ se, e somente se, $a-b=0$, ou seja, $a=b$.
Note que a relação de igualdade sempre é uma relação de congruência, em qualquer anel.

Se $n\geq 1$, $\sim_n$ corresponde à relação de congruência módulo $n$, pois $a\sim_n b$ se, e somente se, $a-b\in n\mathbb Z$, ou seja, $a-b=kn$ para algum $k\in \mathbb Z$.
\end{exemplo}

\section{Quocientes}

Como feito nos inteiros, podemos, ao invés de trabalhar com relações de congruência, encontrar anéis em que a congruência corresponda exatamente à igualdade.

\begin{definition}
Seja $A$ um anel e $\sim$ uma relação de congruência.

Lembremos que o conjunto das classes de equivalência de $\sim$ é denotado por $A/\sim$, e este corresponde, portanto, à $\{[a]_\sim: a \in A\}$, onde $[a]_\sim=\{b\in A: b\sim a\}$ é a classe de equivalência de $a$ com relação a $\sim$.

Define-se que $[a]_\sim+[b]_\sim=[a+b]_\sim$ e que $[a]_\sim[b]_\sim=[ab]_\sim$.
Com essas operações, $(A/\sim, +, \cdot, [0]_\sim, [1]_\sim)$ é chamado de \emph{anel quociente} de $A$ por $\sim$.

Se $I$ é um ideal define-se $A/I=A/\sim_I$, e este é munido das operações anteriores.
Com essas operações, $A/I=A/{\sim_I}$ como descrito acima é chamado de \emph{anel quociente} de $A$ por $I$.

Define-se o \emph{mapa quociente} de $A$ em $A/I$ se dá por $q:A\longrightarrow A/I$ dada por $q(a)=[a]_{\sim_I}$.
\end{definition}

É claro que precisamos mostrar que as operações acima estão bem definidas e torna estes, de fato, anéis.
\begin{lemma}
    As operações dos anéis quocientes estão bem definidas e os tornam anéis.
    Além disso, o mapa quociente é um epimorfismo (homomorfismo sobrejetor).
\end{lemma}

\begin{proof}
    Como as relações de congruência estão em bijeção com os ideais, podemos tratar de um quociente arbitrário da forma $A/\sim$.
    
    Primeiro, vejamos que as operações estão bem definidas, ou seja, que se $a\sim b$ e $c\sim d$, então $[ac]_\sim=[bd]_\sim$ e $[a+b]_\sim=[b+d]_\sim$.
    
    De fato, como $\sim$ é uma relação de congruência e $a\sim b$ e $c\sim d$, temos que $ac\sim bc$ e $a+c\sim b+d$, logo, $[ac]_\sim=[bc]_\sim$ e $[a+c]_\sim=[b+d]_\sim$.
    Note ainda que como $[a]_\sim=q(a)$ e $q(1_A)=[1_A]_\sim$, assim, segue que, caso $A/\sim$ seja anel, $q$ é homomorfismo sobrejetor.

    Agora devemos ver que $A/\sim$ é um anel. Temos que:

    \begin{itemize}
        \item Comutatividade da soma: $q(a)+q(b)=q(a+b)=q(b+a)=q(b)+q(a)$.
        \item Associatividade da soma: $(q(a)+q(b))+q(c)=q(a+b)+q(c)=q((a+b)+c)=q(a+(b+c))=q(a)+q(b+c)=q(a)+(q(b)+q(c))$.
        \item Neutro da soma: $q(0)+q(a)=q(0+a)=q(a)$.
        \item Opostos: $q(a)+q(-a)=q(a+(-a))=q(0)=0$.
        \item Associatividade do produto:$(q(a)q(b))q(c)=q(ab)q(c)=q((ab)c)=q(a(bc))=q(a)q(bc)=q(a)(q(b)q(c))$.
        \item Neutro do produto: $q(1)q(a)=q(1a)=q(a)$, e $q(a)q(1)=q(a1)=q(a)$.
        \item Distributividade: $q(a)(q(b)+q(c))=q(a)q(b+c)=q(a(b+c))=q(ab+ac)=q(ab)+q(ac)=q(a)q(b)+q(a)q(c)$.
        \item Distributividade II: $(q(a)+q(b))q(c)=q(a+b)q(c)=q((a+b)c)=q(ac+bc)=q(ac)+q(bc)=q(a)q(c)+q(b)q(c)$.
    \end{itemize}
        
\end{proof}

Algumas propriedades particulares do quociente:

\begin{lemma}[Propriedades do quociente]
    Na notação acima:
    \begin{enumerate}[label=\alph*)]
        \item $\ker q = I$. \label{lemma:propriedadesQuociente_a}
        \item $q(a)=a+I=\{a+x: x \in I\}$ para todo $a \in A$. \label{lemma:propriedadesQuociente_b}
        \item Se $A$ é anel comutativo, $A/I$ também é. \label{lemma:propriedadesQuociente_c}
    \end{enumerate}
\end{lemma}

\begin{proof}
    \ref{lemma:propriedadesQuociente_a} Temos que $\ker q=\{a \in A: q(a)=q(0)\}=\{a \in A: a\sim_I 0\}=\{a \in A: a\in I\}=I$.

    \ref{lemma:propriedadesQuociente_b} Temos que $q(a)=[a]_{\sim_I}=\{b \in A: b\sim_I a\}=\{b \in A: b-a\in I\}=\{a+x: x \in I\}$ pois se $b-a \in I$ se, e somente se $a-b=x$ para algum $x \in I$.

    \ref{lemma:propriedadesQuociente_c} Se $A$ é comutativo, então $A/I=\ran q$ também é, pois $q$ é homomorfismo de anéis.
\end{proof}

Em particular, temos:

\begin{corol}
    Todo ideal é o núcleo de algum homomorfismo.
\end{corol}

\section{Teoremas do isomorfismo}
Os teoremas do homomorfismo dizem que certos homomorfismos ``fatoram'' para quocientes.
\begin{theorem}[Teorema do homomorfismo]
    Seja $f:A\rightarrow R$ um homomorfismo de anéis e $J$ um ideal tal que $J\subseteq \ker f$. Então, existe um único homomorfismo de anéis $\bar f:A/J\rightarrow R$ tal que $\bar f\circ q=f$, onde $q:A\rightarrow A/J$ é o mapa quociente canônico dado por $q(a)=a+J$.
    \begin{figure}[h]\centering
        \begin{tikzcd}[column sep=1.6cm,row sep=1.2cm]
            A\arrow[d, "q"']\arrow[r, "f"]& R \\
            A/J \arrow[ur, dashed, "\exists!\bar f"']
        \end{tikzcd}
        \caption{Teorema do homomorfismo.}
    \end{figure}
\end{theorem}
\begin{proof}
    Definimos $\bar f:A/J\rightarrow R$ por $\bar f(a+J)=f(a)$.
    Então, $g$ é bem definido, pois se $a+J=b+J$, então $a-b \in J\subseteq \ker f$, logo, $f(a-b)=0_R$, ou seja, $f(a)=f(b)$.

    Agora, vejamos que $\bar f$ é um homomorfismo de anéis.
    De fato, para todo $a', b' \in A/J$, sendo $a'=a+J$ e $b'=b+J$, temos que:
    \begin{itemize}
        \item $\bar f(a'+b')=\bar f((a+J)+(b+J))=\bar f((a+b)+J)=f(a+b)=f(a)+f(b)=\bar f(a+J)+\bar f(b+J)$.
        \item $\bar f(a'b')=\bar f((a+J)(b+J))=\bar f(ab+J)=f(ab)=f(a)f(b)=\bar f(a+J)\bar f(b+J)$.
        \item $\bar f(1_{A/J})=\bar f(1_A+J)=f(1_A)=1_R$.
    \end{itemize}

    Temos que $\bar f\circ q=f$ por definição de $\bar f$.
    Para a unicidade, se $g:A/J\rightarrow R$ é um homomorfismo tal que $g\circ q=f$, fixe $a'\in A/J$.
    Fixe $a \in A$ tal que $a'=q(a)$.
    Então $g(a')=g(q(a))=f(a)=\bar \bar f(q(a))=\bar f(a')$.
    Assim, $g=\bar f$.
\end{proof}

Como consequência, temos o Primeiro Teorema do Isomorfismo:

\begin{theorem}[Primeiro Teorema do Isomorfismo]
    Seja $f:A\rightarrow R$ um homomorfismo de anéis.
    Então, $A/I$ é isomorfo a $\ran f$.
    Mais especificamente, existe um único homomorfismo $\phi:A/\ker f\rightarrow R$ tal que $q\circ \phi=f$, onde $q$ é o mapa quociente, e este homomorfismo é necessariamente um isomorfismo.
    \begin{figure}[H]\centering
        \begin{tikzcd}[column sep=1.6cm,row sep=1.2cm]
            A\arrow[d, "q"']\arrow[r, "f"]& \ran f\\
            A/\ker f \arrow[ur, dashed, "\exists! \phi"'] & 
        \end{tikzcd}
        \caption{Primeiro Teorema do Isomorfismo.}
    \end{figure}
\end{theorem}

\begin{proof}
Pelo Teorema do Homomorfismo, existe um único homomorfismo $\bar \phi:A/\ker f\rightarrow \ran f$ tal que $\phi\circ q=f$, onde $q:A\rightarrow A/\ker f$ é o mapa quociente canônico dado por $q(a)=a+\ker f$.

Temos que $\phi$ é sobrejetor: dado $b \in \ran f$, existe $b \in A$ tal que $f(a)=b$.
Logo, $b=f(a)=\bar \phi(q(a))$, assim, $b \in \ran \phi$.

Agora vejamos que $\phi$ é injetor.
Suponha que $y \in A/\ker f$ é tal que $\phi(y)=0$.
Como $q$ é sobrejetor, tome $a \in A$ tal que $y=q(a)$.
Assim, $0=\phi(y)=\phi\circ q(a)=f(a)$, logo, $a \in \ker f$.
Como $q:A\rightarrow A/\ker f$ é o mapa quociente e $a \in \ker f$, segue que $y=q(a)=0_{A/\ker f}$.
Logo, $\ker\phi=\{0\}$, ou seja, $\phi$ é injetor.
\end{proof}

Do primeiro Teorema do Isomorfismo, decorre o segundo Teorema do Isomorfismo.
Para enuncia-lo, lembremos que se $B, C$ são subconjuntos de um grupo abeliano $A$, então $B+C=\{b+c: b \in B, c \in C\}$.

\begin{lemma}
Se $A$ é um anel, $B$ um subanel de $A$ e $I$ um ideal de $A$ contido em $B$, então para todo $b\in B$, a classe de equivalência $[b]_I$ é a mesma tomando como ambiente tanto o anel $B$ como o anel $A$.

Assim, $B/I\subseteq A/I$.

Além disso, sendo $q:A\rightarrow A/I$ o mapa quociente e $q':B\rightarrow B/I$ o mapa quociente, temos que $q'=q|B$.
\end{lemma}

\begin{proof}
    Fixe $b$.
    Devemos ver que $\{a \in A: a-b\in I\}=\{a \in B: a-b\in I\}$.

    Assim, basta ver que se $a \in A$ e $a-b \in I$, então $a \in B$.
    Ora, $a=(a-b)+b$.
    Como $a-b \in I\subseteq B$ e $b \in B$, temos que $a \in B$.

    Note que o lado esquerdo da igualdade é $q(b)$ e o direito é $q'(b)$, assim, segue a tese.
\end{proof}

\begin{theorem}[Segundo Teorema do Isomorfismo]
    Sejam $A$ um anel, $B$ um subanel de $A$ e $I$ um ideal de $A$. Então:

    \begin{enumerate}[label=\alph*)]
        \item $I\cap B$ é um ideal de $B$.
        \item $I+B$ é um subanel de $A$.
        \item $\displaystyle \frac{I+B}{I}\cong \frac{B}{I\cap B}$.
    
    \end{enumerate}
\end{theorem}

\begin{proof}
    Primeiro, verifiquemos que $I+B$ é um subanel de $A$.

    Temos que $1=0+1 \in I+B$.

    Se $x, y \in I+B$, então $x=a_1+b_1$ e $y=a_2+b_2$, onde $a_1, a_2 \in I$ e $b_1, b_2 \in B$.
    Segue que $a_1-a_2 \in I$ e $b_1-b_2 \in B$, logo, $x-y=(a_1-a_2)+(b_1-b_2) \in I+B$.

    Além disso, $xy=(a_1+b_1)(a_2+b_2)=a_1a_2+a_1b_2+b_1a_2+b_1b_2$.
    Temos que $a_1a_2\in I$, $a_1b_2\in I$, $b_1a_2\in I$ e $b_1b_2\in B$, logo, $xy\in I+B$.
    Assim, $I+B$ é um subanel de $A$.

    Agora considere o mapa $q:I+B\rightarrow \frac{I+B}{I}$ dado por $q(x)=x+I$. Seja $f=q|_B:B\rightarrow \frac{I+B}{I}$ o homomorfismo restrito de $q$ em $B$.

    Pelo primeiro Teorema do Isomorfismo, $B/\ker f \cong \ran f$.
    Veremos que $\ran f=I+B/I$ e $\ker f=I\cap B$, o que completa a prova.

    Temos que $\ran f = \frac{I+B}{I}$ pelo lema anterior, pois $I\subseteq B\subseteq I+B$.

    Calculemos $\ker f$. Ora, se $x \in B$, temos que $f(x)=0$ se, e somente se, $q(x)=0$ se, e somente se $x\in I$.
    Como $x\in B$, isso é equivalente à $x \in I\cap B$, o que completa a prova.
\end{proof}
Finalmente, temos o Terceiro Teorema do Isomorfismo.
\begin{theorem}[Terceiro Teorema do Isomorfismo]
    Sejam $A$ um anel, $B$ um subanel de $A$ e $I\subseteq J\subseteq B$ ideais.
    Seja $q:A\rightarrow A/I$ a projeção natural.
    Então $J/I=\{q(a):a \in J\}$ é um ideal de $A/I$, e:
    \[
         (B/I)/(J/I) \cong B/J.
    \]
\end{theorem}
\begin{proof}
    Seja $p:B\rightarrow B/J$ o mapa quociente dado por $p(b)=b+J$ para todo $b \in B$.
    Seja $q'=q|B:B\rightarrow B/I$ o mapa quociente para $B/I$.

    Temos que $\ker p=B\cap J=J$ e $I\subseteq J$.
    Assim, pelo Teorema do Homomorfismo, existe $\bar p:B\rightarrow B/J$ homomorfismo tal que $\bar f\circ q'=f$.
    \begin{figure}[H]\centering
        \begin{tikzcd}
            B\arrow[d, "q'"']\arrow[r, "p"]& B/J\\
            A/I \arrow[ur, dashed, "\bar p"] \arrow[d]& \\
            \frac{A/I}{\ker \bar p}\arrow[uur, dashed, "\phi"']&
        \end{tikzcd}
    \end{figure}
    Pelo Primeiro Teorema do Isomorfismo, $(B/J)/\ker \bar f\cong \ran \bar f$.
    Calcularemos $\ran \bar f$ e $\ker \bar f$, o que concluirá a prova.

    Temos que, para $\bar p\circ q'=p$.
    Como $q'$ e $p$ são sobrejetoras, temos:
    \[\ran \bar p=\{p(x): x \in A/I\}=\{\bar p(q(b)):b \in B\}=\{p(b):b \in B\}=\ran p=B/J.\]

    Agora calcularemos $\ker \bar p$.
    Fixe $x \in A/I$.
    Existe $b \in B$ tal que $x=q(b)$.
    Se $x \in \ker \bar p$, então $0=\bar p(x)=\bar p(q(b))=p(b)=b+J$, logo, $b \in J$, e, portanto, $x=q(b)\in J/I$.
    Reciprocamente, se $x \in J/I$, então $x=q(b)$ para algum $b \in J$, logo, $p(b)=0$, e, portanto, $p(x)=\bar p(q'(b))=p(b)=0$.
    Assim, $x \in \ker \bar p$.

    Assim, temos que $\ker \bar p=J/I$, e este último é um ideal, pois núcleos de homomorfismos são ideais.
\end{proof}

No terceiro teorema do isomorfismo, vimos que se $I\subseteq J\subseteq A$, então $J/I$ é um ideal de $A/I$.
Quem são os ideais de um quociente?
O teorema a seguir mostra que todos são dessa forma.

\begin{theorem}[Teorema da correspondência]
    Seja $A$ é um anel e $I$ um ideal de $A$.
    Considere a função $\phi:\{J\subseteq A: I\subseteq J \text{ e } J \text{é ideal de } A\}\rightarrow \{K\subseteq A/I: K \text{é ideal de } A/I\}$ dada por:
    \[\phi(J)=J/I.\]
    Então $\phi$ é uma bijeção entre os ideais de $A$ que contêm $I$ e os ideais de $A/I$.
    Além disso, $\phi$ é um isomorfismo de ordem, ou seja, se $J_1, J_2$ são ideais e $I\subseteq J_1, I\subseteq J_2\subseteq A$, então $\phi(J_1)\subseteq \phi(J_2)$ se, e somente se $J_1\subseteq J_2$.
\end{theorem}

\begin{proof}
    Pelo Terceiro Teorema do Isomorfismo, o contradomínio de $\phi$ está correto.
    Pela definição de $J/I$, é claro que $\phi$ é uma função crescente (se $J_1\subseteq J_2$, então $J_1/I\subseteq J_2/I$).

    Agora, seja $\psi:\{K\subseteq A/I: K \text{é ideal de } A/I\}\rightarrow \{J\subseteq A: I\subseteq J \text{ e } J \text{é ideal de } A\}$ dada por $\psi(K)=q^{-1}[K]$, onde $q:A\rightarrow A/I$ é o mapa quociente dado por $q(a)=a+I$.

    Como $q$ é um homomorfismo e ideais são preservados por imagens inversas de homomorfismos, segue que cada $\psi(K)$ é um ideal de $A$.
    Além disso, $\psi(K)$ contém $I$, já que $\ker q=q^{-1}(0)=I\subseteq \psi(K)$.
    Finalmente, pela definição de pré-imagem, $\psi$ também preserva a ordem.

    Agora veremos que $\phi, \psi$ são isomorfismos inversos, o que completará a prova.

    Dado um ideal $J$ de $A$ que contém $I$, temos que $\psi(\phi(I))=\psi(J/I)=\{a \in A: q(a)\in J/I\}$.
    Afirmamos que esse conjunto é $J$.
    Com efeito, se $a \in J$, temos que $a \in A$ e $q(a)\in J/I$.
    Reciprocamente, se $a \in A$ e $q(a) \in J/I$, existe $b \in J$ tal que $q(a)=q(b)$.
    Assim, $b \in J$ e $a-b \in I\subseteq J$, logo, $a=(a-b)+b\in J$.

    Agora, fixe um ideal $K$ de $A/I$.
    
    Veremos que $\phi(\psi(K))=K$.

    Temos que $\phi(\psi(K))=\phi(q^{-1}[K])=\phi(\{a \in A: q(a)\in K\})=\{q(a): a \in A \text{ e } q(a) \in K\}$.
    É imediato que este último é $K$, o que completa a prova.
\end{proof}
\chapter{Domínios de Integridade}
Neste capítulo, exploraremos com mais detalhes os domínios de integridade e a teoria que nasce deles.

\section{Relações entre corpos e domínios de integridade}
Conforme visto, todo corpo é um domínio de integridade, e a recíproca não é verdadeira (sendo $\mathbb Z$ um contra-exemplo).

A seguir, apresentaremos algumas relações entre corpos e domínios de integridade.

\begin{prop}
Todo domínio de integridade finito é um corpo.
\end{prop}
\begin{proof}
    Seja $R$ um domínio de integridade finito.
    Fixe $a \in R\setminus\{0\}$.
    Veremos que $a$ é invertível.

    Considere $\phi:R\setminus\{0\}\rightarrow R\setminus \{0\}$ dado por $\phi(x)=ax$.

    Como $R$ é um domínio de integridade, para todo $x \in R\setminus \{0\}$, temos $ax \neq 0$, logo, $\phi$ está bem definida.

    $\phi$ é uma função injetora: se $\phi(x)=\phi(y)$, então $ax=ay$.
    Logo, $a(x-y)=0$.
    Como $a\neq 0$ e $R$ é um domínio de integridade, segue que $x-y=0$, ou seja, $x=y$.

    Como $R\setminus\{0\}$ é finito e $\phi:R\setminus\{0\}\rightarrow R\setminus \{0\}$ é injetora, segue que $\phi$ é sobrejetora.
    Em particular, existe $x \in X$ tal que $ax=\phi(x)=1$.
    Logo, $a$ é invertível.
\end{proof}

Portanto, restrito aos anéis finitos, o estudo dos corpos e domínios de integridade colapsa em um único estudo.

Outra relação importante é a que segue:

\begin{prop}
Seja $R$ um anel comutativo e $I$ um ideal próprio de $R$. São equivalentes:

\begin{enumerate}[label=(\roman*)]
    \item $R/I$ é um corpo;
    \item $I$ é maximal.
\end{enumerate}
\end{prop}

\begin{proof}
    Seja $q:R\rightarrow I$ o mapa quociente.

    (i) $\Rightarrow$ (ii): Suponha que $R/I$ é um corpo.
    

    $I$ é um ideal próprio, caso contrário, teríamos que $R/I$ é o anel trivial, que não é um corpo.

    Agora suponha que $J$ é um ideal que contém $I$ propriamente.
    Veremos que $J=R$.
    Seja $a \in J\setminus I$.
    Como $a\notin I$, temos que $q(a)\neq 0$.
    Como $A/I$ é um corpo, existe $b \in R$ tal que $q(a)q(b)=1$.
    Isso implica que existe $x \in I$ tal que $ab+x=1$.
    Como $a \in J$ e $x \in I\subseteq J$, segue que $1=ab+x\in J$, e, portanto, $J=R$.

    (ii) $\Rightarrow$ (i): Suponha que $I$ é maximal. Vejamos que $R/I$ é um corpo.

    Seja $x \in R\setminus I$ não nulo.
    Tome $a \in R$ tal que $q(a)=x$.
    Temos que $a \notin I$.
    Como $I+\langle a\rangle$ é um ideal que contém $I$ propriamente, segue que $I+\langle a\rangle=R$.
    Logo, existe $b \in R$ e $c \in I$ tais que $c+ba=1$.
    Logo, $q(1)=q(c)+q(ba)=0+q(b)q(a)=q(b)x$.
    Portanto, $x$ é invertível.
\end{proof}

Será que podemos caracterizar, de forma análoga, ser um domínio de integridade? A resposta é positiva.

\begin{prop}
    Seja $R$ um anel comutativo e $I$ um ideal próprio de $R$. São equivalentes:
    
    \begin{enumerate}[label=(\roman*)]
        \item $R/I$ é um domínio de integridade.
        \item $I$ é primo.
    \end{enumerate}
\end{prop}

\begin{proof}
    Seja $q:R\rightarrow I$ o mapa quociente.

    (i) $\Rightarrow$ (ii): Suponha que $R/I$ é um domínio de integridade.
    

    $I$ é um ideal próprio, caso contrário, teríamos que $R/I$ é o anel trivial, que não é um domínio de integridade.

    Suponha que $a,b \in R$ tais que $ab \in I$.
    Temos que $q(a)q(b)=q(ab)=0$.
    Como $R/I$ é um domínio de integridade, temos que $q(a)=0$ ou $q(b)=0$, ou seja, que $a \in I$ ou $B \in I$.

    Logo, $I$ é primo.

    (ii) $\Rightarrow$ (i): Suponha que $I$ é primo.
    Vejamos que $R/I$ é um domínio de integridade.

    Sejam $x, y \in R$ tais que $q(x)q(y)=0$.
    Devemos ver que $q(x)=0$ ou $q(y)=0$.
    Como $q(xy)=q(x)q(y)=0$, segue que $xy\in I$.
    Então, $x\in I$ ou $y\in I$, ou seja, $q(x)=0$ ou $q(y)=0$.
\end{proof}

Como consequência, temos:

\begin{corol}
    Seja $R$ um anel comutativo finito e $I$ um ideal de $R$. Então $I$ é primo se, e somente se $I$ é maximal.
\end{corol}
\begin{proof}
    Temos que $R/I$ é finito, e, portanto, é um corpo se, e somente se for um domínio de integridade.
    Portanto:

    \[I \text{ é primo} \Leftrightarrow R/I \text{ é um domínio de integridade} \Leftrightarrow R/I \text{ é um corpo} \Leftrightarrow I \text{ é maximal}\]
\end{proof}

\section{O corpo de frações de um domínio de integridade}

Conforme vimos, nem todo domínio de integridade é um corpo, sendo $\mathbb Z$ é o contra-exemplo mais usual.
Apesar disso, parece que, em algum sentido, $\mathbb Q$ é o ``menor'' corpo que contém $\mathbb Z$.

Uma das construções mais usuals do corpo $\mathbb Q$ utiliza classes de equivalências de pares de elementos de $\mathbb Z$.
Nesta seção, estudaremos esta construção de modo generalizado.

Iniciaremos apresentando uma construção do corpo de frações.

\begin{definition}
    Seja $R$ um domínio de integridade.

    Definamos, em $R\times \{0\}$, a relação de equivalência $\sim$ a seguir:

    \[(a, b) \sim (c, d) \Leftrightarrow ad=bc.\]

    O conjunto das classes de equivalência $R/\sim$ será denotado por $\Frac(R)$.
\end{definition}

Ao longo desta seção, a notação $\sim$ será fixada e utilizada exclusivamente para esse fim.
A ideia é pensar em cada par $(a, b)$ como uma fração $\frac{a}{b}$.
A relação $\sim$ captura a ideia que duas frações $\frac{a}{b}$ e $\frac{c}{d}$ são equivalentes se, e somente se, $ad=bc$.

\begin{lemma}
    Na notação acima, a relação $\sim$ é uma relação de equivalência em $R\times \{0\}$.
\end{lemma}

\begin{proof}
    Seja $(a, b), (c, d), (e, f) \in R\times \{0\}$.
    \begin{itemize}
        \item Temos que $(a, b)\sim (a, b)$ pois $ab=ba$.
        \item Simetria: se $(a, b)\sim (c, d)$, temos que $ad=bc$.
        Logo, $cb=da$, o que nos dá $(c, d)\sim (a, b)$.
        \item Transitividade: suponha que $(a, b)\sim (c, d)$ e $(c, d)\sim (e, f)$.
        Temos que $ad=bc$ e $cf=de$.
        Multiplicando a primeira equação por $f$ e a segunda por $b$, temos que $adf=bcf$ e $bcf=deb$.
        Logo, $adf=deb$.
        Como $d\neq 0$, cancelando, temos que $af=eb$, ou seja, que $(a, b)\sim (e, f)$.
    \end{itemize}
\end{proof}
\section{Exercícios}
\begin{exer}
    Demonstre, com suas próprias palavras, de modo que considere satisfatório, a seguinte afirmação demonstrada no texto: Todo domínio de integridade finito é um corpo.
\end{exer}
\chapter{Produtos de anéis}
Neste capítulo, estudaremos o produto direto de anéis.

\section{Produtos de dois anéis}
Dados anéis $R$ e $S$, é possível dar à $R\times S$ uma estrutura natural de anel.
\begin{definition}[Produto Direto de dois anéis]
    Sejam $R, S$ anéis.
    O produto direto de $R$ e $S$ é o conjunto $R\times S$ munido das operações ``ponto à ponto'': dados $a=(a_1, a_2)\in R\times S$ e $b=(b_1, b_2)\in R\times S$, temos:
    \[a+b=(a_1+b_1, a_2+b_2)\]
    \[a\cdot b=(a_1\cdot b_1, a_2\cdot b_2)\]
    \[0=(0_R, 0_S)\]
    \[1=(1_R, 1_S)\]
\end{definition}

Exemplo: Seja $R=\mathbb Z_3$ e $S=\mathbb Z_4$. Então $(2, 2)\in R\times S$ e $(1, 2)\in R\times S$. Temos:
\[(2, 2)+(1, 2)=(2+ 1, 2+ 2)=(0, 0).\]
\[(2, 2)\cdot (2, 2)=(2\cdot 2, 2\cdot 2)=(1, 0).\]

Com as operações explicitadas, o produto de dois anéis é, de fato, um anel.

Deixaremos a prova deste fato como exercício (ver Exercício~\ref{exer:prod_anel}), já que na seção seguinte provaremos um resultado mais geral.

\section{Produtos de uma família de anéis}

\begin{definition}[Produtos de anéis]
    Seja $(R_i)_{i \in I}$ uma família de anéis, onde cada $R_i$ tem as operações $+_i, \cdot_i$ e constantes $0_i, 1_i$.
    
    O produto (direto) de $(R_i)_{i \in I}$ é o conjunto $\prod_{i \in I} R_i$ munido das operações ``ponto à ponto'': dados $a=(a_i: i \in I), b=(b_i: i \in I)$ em $\prod_{i \in I}R_i$:

    $$a+b=(a_i: i \in I)+(b_i: i \in I)=(a_i+_i b_i: i \in I)=(a_i+_ib_i)_{i \in I}$$
    $$a\cdot b=(a_i: i \in I)\cdot (b_i: i \in I)=(a_i\cdot _i b_i: i \in I)=(a_i\cdot _ib_i)_{i \in I}$$

\end{definition}

\begin{lemma}[O produto de anéis está bem definido]
    Seja $(R_i)_{i \in I}$ uma família de anéis.
    Então seu produto direto $\prod_{i \in I}R_i$ é um anel com $0=(0_i: i \in I)$ e $1=(1_i: i \in I)$.
\end{lemma}

\begin{proof}
    Sejam $a=(a_i: i \in I), b=(b_i: i \in I)$ e $c=(c_i: i \in I)$ em $\prod_{i \in I}R_i$.
    \begin{itemize}
        \item \textbf{Associatividade da soma:} $(a+b)+c=(a_i+_i b_i)_{i \in I}+c=((a_i+_i b_i)+_ic_i)_{i \in I}=(a_i+_i (b_i+_i c_i))_{i \in I}=a+(b+c)$
        \item \textbf{Associatividade do produto:} Análogo.
        \item \textbf{Comutatividade da soma:} $a+b=(a_i+_i b_i)_{i \in I}=(b_i+_i a_i)_{i \in I}=b+a$
        \item \textbf{Neutro da soma:} $a+0=(a_i+_i 0_i)_{i \in I}=(a_i)_{i \in I}=a$
        \item \textbf{Inverso da soma:} Dado $a=(a_i)_{i \in I}$, considere $-a=(-a_i)_{i \in I}$. Então $a+(-a)=(a_i+_i (-a_i))_{i \in I}=(0_i)_{i \in I}=0$.
        \item \textbf{Distributividade:} $a\cdot (b+c)=(a_i\cdot _i (b_i+c_i))_{i \in I}=(a_i\cdot _i b_i+a_i\cdot _i c_i)_{i \in I}=a\cdot b+a\cdot c$.
        \item \textbf{Distributividade II:} $(a+b)\cdot c=((a_i+b_i)\cdot _i c_i)_{i \in I}=(a_i\cdot _i c_i+b_i\cdot _i c_i)_{i \in I}=a\cdot c+b\cdot c$.
        \item \textbf{Neutro do produto:} $a\cdot 1=(a_i\cdot _i 1_i)_{i \in I}=(a_i)_{i \in I}=a$ e $1\cdot a=(1_i\cdot _i a_i)_{i \in I}=(a_i)_{i \in I}=a$.
    \end{itemize}
\end{proof}
\begin{definition}[Os mapas de projeção]
    Seja $(R_i)_{i \in I}$ uma família de anéis e seja $P=\prod_{i \in I}R_i$.
    Para cada $i \in I$, o mapa de projeção $\pi_i:R\rightarrow R_i$ é dado por $\pi_i(a)=a_i$.

    Escrevendo de outra forma, $\pi_i((a_j: j \in I))=a_i$.
\end{definition}

\begin{lemma}[Os mapas de projeção são homomorfismos]
    Seja $(R_i)_{i \in I}$ uma família de anéis e seja $P=\prod_{i \in I}R_i$.
    Para cada $i \in I$, o mapa de projeção $\pi_i:R\rightarrow R_i$ é um homomorfismo de anéis.
\end{lemma}
\begin{proof}
    Sejam $a=(a_j: j \in I), b=(b_j: j \in I)$ em $P$.
    Então:
    \begin{itemize}
        \item $\pi_i(a+b)=\pi_i((a_j+b_j)_{j \in I})=a_i+b_i=\pi_i(a)+\pi_i(b)$
        \item $\pi_i(a\cdot b)=\pi_i((a_j\cdot b_j)_{j \in I})=a_i\cdot b_i=\pi_i(a)\cdot \pi_i(b)$
        \item $\pi_i(1_P)=\pi_i((1_j)_{j \in I})=1_{i}$
    \end{itemize}
\end{proof}

\section{A propriedade universal do produto direto de anéis}
\begin{theorem}[Propriedade universal do produto direto de anéis]
    Seja $(R_i)_{i \in I}$ uma família de anéis e seja $P=\prod_{i \in I}R_i$ seu produto direto.
    Então, para cada anel $S$ e cada família de homomorfismos de anéis $f_i:R_i\rightarrow S$, existe um único homomorfismo de anéis $g:p\rightarrow S$ tal que $\pi_i\circ g=f_i$ para todo $i \in I$.
    \begin{figure}[H]
        \centering
    \begin{tikzcd}[column sep=1.5cm,row sep=1.2cm]
        & S\arrow[ld, "f_i"']\arrow[d, dashed, "\exists! g"]\\
        R_i  & \arrow[l, "\pi_i"]P
    \end{tikzcd}
    \end{figure}

    Além disso, tal propriedade caracteriza o produto direto. Ou seja, para quaisquer que sejam um anel $P'$ e uma família de homomorfismos $(p_i:P'\rightarrow R_i)_{i \in I}$, se para todo anel $S$ e toda família de homomorfismos de anéis $f_i:R_i\rightarrow S$ existir um único homomorfismo de anéis $f:P'\rightarrow S$ tal que $p_i\circ f=f_i$ para todo $i \in I$,
    então existe um único isomorfismo de anéis $\phi: P'\rightarrow P$ tal que $\pi_i\circ \phi=p_i$ para todo $i \in I$.
\end{theorem}

\begin{proof}
    Seja $P=\prod_{i \in I}R_i$ e seja $S$ um anel comutativo. Para cada $i \in I$, considere $f_i:S\rightarrow R_i$ um homomorfismo de anéis.
    Defina $g:S\rightarrow P$ tal que, dado $s \in S$:
    \[g(s)=(f_i(s))_{i \in I}.\]

    Então, para cada $i \in I$, $\pi_i\circ g(s)=\pi_i(f_j(s): j \in I)=f_i(s)$, ou seja, $\pi_i\circ f=f_i$.
    Vejamos que $g$ é homomorfismo de anéis.
    Dados $s, t \in S$, temos:
    \begin{itemize}
        \item $g(s+t)=(f_i(s+t))_{i \in I}=(f_i(s)+f_i(t))_{i \in I}=(f_i(s))_{i \in I}+(f_i(t))_{i \in I}=g(s)+g(t)$.
        \item $g(s\cdot t)=(f_i(s\cdot t))_{i \in I}=(f_i(s)\cdot f_i(t))_{i \in I}=(f_i(s))_{i \in I}\cdot (f_i(t))_{i \in I}=g(s)\cdot g(t)$.
        \item $g(1_S)=(f_i(1_S))_{i \in I}=(1_i)_{i \in I}=1_R$.  
    \end{itemize}

    Vejamos que $g$ é único.
    Se $\bar g:R\rightarrow S$ é um homomorfismo de anéis tal que $\pi_i\circ \bar g=f_i$, fixe $s \in S$.
    Devemos ver que $\bar g(s)=g(s)$.
    Como $\bar g(s) \in P$, escreva $\bar g(s)=(b_i)_{i \in I}$, onde $b_i \in R_i$ para cada $i \in I$. Temos, que, para cada $j \in I$:
    \[b_j=\pi_j((b_i)_{i \in I})=\pi_j\circ \bar g(s)=f_j(s).\]
    Assim, $f_j(s)=b_j$ para todo $j \in I$. Daí, $\bar g(s)=(b_j)_{j \in I}=(f_j(s))_{j \in I}=g(s)$.
    Portanto, $g=\bar g$.

    Agora suponha que $P'$ e $(p_i:P'\rightarrow R_i)_{i \in I}$ são como no enunciado.

    Aplicando a propriedade de $P$ para $(\pi_i: i \in I)$, existe um homomorfismo de anéis $\phi: P'\rightarrow P$ tal que $\pi_i\circ \phi=p_i$ para todo $i \in I$.  

    \begin{figure}[H]
        \centering
    \begin{tikzcd}[column sep=1.5cm,row sep=1.2cm]
        & P'\arrow[ld, "p_i"']\arrow[d, dashed, "\exists! \phi"]\\
        R_i  & \arrow[l, "\pi_i"]P
    \end{tikzcd}
    \end{figure}

    Nosso objetivo é mostrar que $\phi$ é isomorfismo.
    Construiremos uma inversa.
    Como ele é o único homomorfismo tal que $\pi_i\circ \phi=p_i$ para todo $i \in I$, e como todo isomorfismo é homomorfismo, isso conclui a prova.

    Aplicando a propriedade de $P'$ para $(\pi_i: i \in I)$, existe um homomorfismo de anéis $\psi: P'\rightarrow P$ tal que $p_i\circ \psi=p_i$ para todo $i \in I$.

    \begin{figure}[H]
        \centering
    \begin{tikzcd}[column sep=1.5cm,row sep=1.2cm]
        & P\arrow[ld, "\pi_i"']\arrow[d, dashed, "\exists! \psi"]\\
        R_i  & \arrow[l, "p_i"]P'
    \end{tikzcd}
    \end{figure}

    Tanto os mapas $\psi\circ \phi$ quanto a identidade $\id_{P'}:P'\rightarrow P'$ são homomorfismos de anéis que satisfazem o seguinte diagrama comutativo:

    \begin{figure}[H]
        \centering
    \begin{tikzcd}[column sep=1.5cm,row sep=1.2cm]
        & P'\arrow[ld, "p_i"']\arrow[d, "\id_{P'}",  "\psi\circ\phi"']\\
        R_i  & \arrow[l, "p_i"]P'
    \end{tikzcd}
    \end{figure}

    Pois para todo $i \in I$, $p_i\circ \id_{P'}= p_i$ e $p_i\circ \psi\circ\phi=\pi_i\circ \phi=p_i$.
    Como a propriedade de $P'$ diz que existe um \emph{único} homomorfismo que satisfaz esse diagrama, segue que $\psi\circ \phi=\id_{P'}$.

    Analogamente, tanto os mapas $\phi\circ \psi$ quanto a identidade $\id_{P}:P\rightarrow P$ são homomorfismos de anéis que satisfazem o seguinte diagrama:
    \begin{figure}[H]
        \centering
    \begin{tikzcd}[column sep=1.5cm,row sep=1.2cm]
        & P\arrow[ld, "\pi_i"']\arrow[d, "\id_{P}",  "\phi\circ\psi"']\\
        R_i  & \arrow[l, "\pi_i"]P
    \end{tikzcd}
    \end{figure}

    Pois $\pi_i\circ \id_{P}=\pi_i$ e $\pi_i\circ \phi\circ\psi=p_i\circ \psi=\pi$.
    Como a propriedade de $P$ diz que existe um \emph{único} homomorfismo que satisfaz esse diagrama, segue que $\phi\circ \psi=\id_{P}$.
    
    Assim, $\psi$ e $\phi$ são isomorfismos inversos.
    Em particular, $\phi$ é isomorfismo, o que completa a prova.
\end{proof}

\section{Exercícios}

\begin{exer}\label{exer:prod_anel}
    Sejam $A, B$ anéis.
    Prove diretamente que o produto direto $A\times B$ é um anel.
    A seguir, prova que as projeções $\pi_1:A\times B\rightarrow A$ e $\pi_2:A\times B\rightarrow B$ dadas por $\pi_1(a, b)=a$ e $\pi_2(a, b)=b$ são homomorfismos de anéis.
\end{exer}

\begin{exer}
    Na notação do exercício anterior, prove diretamente que $A\times S$, com as projeções $(\pi_1, \pi_2)$ satisfazem a propriedade universal do produto direto, ou seja, mostre que:

    Para cada anel $S$ e cada par de homomorfismos de anéis $h_1:S\rightarrow A$ e $h_2:S\rightarrow B$, existe um único homomorfismo de anéis $g:S\rightarrow A\times B$ tal que $\pi_1\circ g=f_1$ e $\pi_2\circ g=f_2$.
\end{exer}
\begin{exer}
Decida quais dos seguintes conjuntos sao subaneis do anel produto $\mathbb R^{[0, 1]}$, onde $[0, 1]$ é o intervalo fechado dos números reais entre $0$ e $1$.
\begin{enumerate}[label=\alph*)]
    \item O conjunto de todas as funções $f:[0, 1]\rightarrow \mathbb R$ tais que $f(q)=0$ para todo $q\in [0, 1]$.
    \item O conjunto de todas as funções polinomiais $f:[0, 1]\rightarrow \mathbb R$.
    \item O conjunto de todas as funções $f:[0, 1]\rightarrow \mathbb R$ que possuem apenas um número finito de zeros, juntamente com a função zero.
    \item O conjunto de todas as funções $f:[0, 1]\rightarrow \mathbb R$ que possuem um número infinito de zeros.
    \item O conjunto de todas as funções $f:[0, 1]\rightarrow \mathbb R$ tais que $\lim_{x\rightarrow 1}f(x)=0$.
    \item O conjunto de todas as combinações lineares racionais das funções $\sin(nx)$ e $\cos(mx)$, onde $m,n$ são inteiros não negativos.
    \item O conjunto de todas as funções $f:[0, 1]\rightarrow \mathbb R$ tais que $f(q)=0$ para todo $q\in [0, 1]$ e $f(0)=1$.
\end{enumerate}
\end{exer}

\begin{exer}
    Seja $C$ o anel das funções de $\mathbb R$ em $\mathbb R$ com a estrutura de anel produto. Demonstre que $C$ não é um domínio.
\end{exer}

\begin{exer}
    Seja $C$ o anel das funções $f:\mathbb R\rightarrow \mathbb R$ com a soma e multiplicação usuais de funções.
    Para cada $r\in \mathbb R$, seja $M(r)$ o subconjunto de $C$ dado por:
    \[M(r)=\{f\in C: f(r)=0\}.\]
    \begin{enumerate}[label=\alph*)]
        \item Demonstre que $M(r)$ é um ideal maximal.
        \item Dê um exemplo de um ideal próprio e não nulo de $C$ que não seja maximal.
    \end{enumerate}
\end{exer}

\backmatter
\end{document} 