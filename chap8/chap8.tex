\chapter{Divisibilidade em anéis}
Neste capítulo, estudaremos a noção de divisibilidade em anéis.
Tal noção é uma generalização da noção de divisibilidade em $\mathbb Z$.

Trataremos de divisibilidade apenas em anéis comutativos.

\section{Definição de divisibilidade}


\begin{definition}
Seja $R$ um anel comutativo. Definimos a relação de divisibilidade, $|$, em $R$, como se segue:

Para $a, b \in R$, dizemos que $a|b$ ($a$ divide $b$) se existe $c \in R$ tal que $b=ac$.
\end{definition}
Algumas propriedades básicas:
\begin{prop}
    Seja $R$ um anel comutativo.
    Então a relação de divisibilidade $|$ em $R$ é uma pré-ordem, ou seja, é reflexiva e transitiva.
\end{prop}

\begin{proof}Sejam $a, b, c \in R$.
    Temos que $a|a$, pois $a=1\cdot a$.

    Se $a|b$ e $b|c$, existem $e, f \in R$ tais que $b=ae$ e $c=bf$.
    Logo, $c=bf=aef=a(ef)$, o que implica em que $a|c$.
\end{proof}

Divisores de zero geram diversas patologias na teoria da divisibilidade, e estas não serão objeto primário de nosso estudo.
Assim, nos restringiremos aos anéis comutativos que não possuem divisores de zero, ou seja, aos domínios de integridade.
\begin{prop}
    Seja $R$ um anel domínio de integridade. Se $a, b\in R$, são equivalentes:

    \begin{enumerate}
        \item $a|b$ e $b|a$.
        \item Existe $u \in R$ invertível tal que $a=ub$.
    \end{enumerate}

    \begin{proof}
        Primeiro, suponha que $a|b$ e $b|a$.
        Temos que existem $c, d$ com $a=cb$ e $b=da$.
        Substituindo, temos que $b=dcb$. Cancelando, $1=dc$. Assim, $c$ é invertível.

        Reciprocamente, como $u$ é invertível, $a=ub$ e $u^{-1}a=b$, logo, $a|b$ e $b|a$.
    \end{proof}
\end{prop}

Com isso, definimos:

\begin{definition}
Seja $R$ um anel comutativo.
Dizemos que elementos $a, b \in R$ são associados se existe $u \in R$ invertível tal que $a=ub$.
\end{definition}

A relação de ser associado é uma relação de equivalência:

\begin{lemma}
    Seja $R$ um anel comutativo. A relação de ser associado é uma relação de equivalência em $R$.
\end{lemma}
\begin{proof}
    Seja $a, b, c \in R$.
    \begin{itemize}
        \item Reflexividade: $a$ é associado a si mesmo, pois $a=1\cdot a$ e $1 \in R^*$.
        \item Simetria: Se $a$ é associado a $b$, então existe $u$ invertível tal que $a=ub$. Logo, $b=u^{-1}a$, e, portanto, $b$ é associado a $a$.
        \item Transitividade: Se $a$ é associado a $b$ e $b$ é associado a $c$, então existem $u, v$ invertíveis tais que $a=ub$ e $b=vc$.
        Logo, temos que $a=uvc$, e, portanto, $a$ é associado a $c$, já que $uv\in R^*$.
    \end{itemize}
\end{proof}