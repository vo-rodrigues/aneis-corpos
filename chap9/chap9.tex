\chapter{Polinômios}
Nesse capítulo, estudaremos os anéis de polinômios no contexto de anéis comutativos.

Apresentaremos uma construção do anel das séries formais, e a partir deste, extrairemos o anel de polinômios.

Após isso, veremos como se dá sua definição abstrata.
\section{Séries Formais}
Ao estudar Análise Real, Análise Funcional, Funções Analíticas ou mesmo Cálculo Diferencial e Integral, é comum se deparar com somas infinitas. Em tais assuntos, essas somas podem convergir ou divergir, e, mesmo quando convergem, não é sempre que podemos manipular essas somas infinitas como gostaríamos.

Para se falar em convergência de tais objetos, é necessária uma noção de convergência, o que pode ser feito por uma noção de métrica, ou, mais geralmente, por uma noção de topologia.
Tal estudo foge do escopo deste texto.

Apesar disso, em anéis comutativos arbitrários, é possível estudar séries de potência como objetos formais, sem nunca de fato computar somar infinitas, ou falar em convergência. É o que faremos nesta seção.


Se $R$ é um anel comutativo, intuitivamente uma série formal é um objeto que se escreve na forma:
$$a_0+a_1x+a_2x^2+\dots=\sum_{i=0}^\infty a_ix^i$$
onde $a_i\in R$.

Antes de definirmos o que, formalmente, é esse objetivo, vamos enunciar algumas propriedades que gostaríamos que esse objeto tivesse.

\begin{itemize}
    \item \textbf{Igualdade:} é conveniente que, no aspecto formal, a igualdade entre séries formais seja determinada pelos seus coeficientes. Ou seja, que:
    \begin{align*}
      \sum_{i=0}^\infty a_ix^i=\sum_{i=0}^\infty b_ix^i & \Leftrightarrow \forall i\in \mathbb N\, a_i= b_i.
    \end{align*}
    \item \textbf{Soma:} intuitivamente, se valem propriedades associativas, comutativas e distributivas, faz sentido que a soma satisfaça a propriedade a seguir, imaginando que as duas séries à esquerda se juntam e se reordenam de modo a obter a da direita.
    \begin{align*}
        \left(\sum_{i=0}^\infty a_ix^i\right)+\left(\sum_{i=0}^\infty b_ix^i\right)=\left(\sum_{i=0}^\infty (a_i+bi)x^i\right)
    \end{align*}
    \item \textbf{Produto:} intuitivamente, se, no lado esquerdo da igualdade, valerem propriedades distributivas ``infinitas'', como podemos definir produto? Ora, o coeficiente $c_i$ da série produto resultante deveria ser obtido agrupando (por meio de uma soma) os coeficientes $a_jb_k$ com $j+k=i$. Isso é equivalente à igualdade abaixo:
    \begin{align*}
        \left(\sum_{i=0}^\infty a_ix^i\right)\cdot\left(\sum_{i=0}^\infty b_ix^i\right)=\sum_{i=0}^\infty \left(\sum_{j=0}^i a_{j}b_{i-j}\right)x^i
    \end{align*}
    \item Notação: $R\llbracket x \rrbracket$ é o conjunto de todas as séries formais em $R$.
\end{itemize}
Olhando apenas para as regras acima, parece que as letras $x^i$ parecem não ter nenhum papel a não ser o de demarcar a ``$i$-ésima posição.'' Com isso em mente, definimos:
\begin{definition}
Seja $R$ um anel comutativo.
Definiremos $R\llbracket x\rrbracket$ como o conjunto $R^{\mathbb N}$ munido das operações definidas abaixo.
Nesse contexto, escrevemos, para $(a_i: i \in \mathbb N)\in R^{\mathbb N}$:

\[\sum_{i=0}^\infty a_ix^i=(a_i)_{i \in \mathbb N}.\]

Reforçamos que, neste contexto, não há nenhuma soma infinita ocorrendo, e o lado esquerdo é simplesmente definido como uma notação alternativa para o lado direito.
Note ainda que, com essa definição, a propriedade da igualdade acima vale automaticamente.

Se $p\in \mathbb R\llbracket x\rrbracket$, escrevemos $p(x)=p$ para reforçar que estamos lidando com uma série formal, e escrevemos $p(x)=\sum_{i=0}^\infty p_ix^i$. Os elementos $p_i$ são chamados de \emph{coeficientes} de $p(x)$, e $p_0$ é chamado de \emph{coeficiente constante}.

\textbf{Operações}:
Se $p(x), q(x) \in R\llbracket x\rrbracket$, define-se:
\[p(x)+q(x)=\sum_{i=0}^\infty p_ix^i+\sum_{i=0}^\infty q_ix^i=\sum_{i=0}^\infty(p_i+q_i)x^i.\]
\[p(x)\cdot q(x)=\sum_{i=0}^\infty p_ix^i\cdot\sum_{i=0}^\infty q_ix^i=\sum_{i=0}^\infty\left(\sum_{j=0}^i p_{i-j}q_{j}\right)x^i.\]
\[1_{R\llbracket x\rrbracket}=\sum_{i=0}^\infty \delta_{i0}x^i=(1, 0, 0, \dots).\]
\[0_{R\llbracket x\rrbracket}=\sum_{i=0}^\infty 0x^i=(0, 0, 0, \dots).\]
\end{definition}

\begin{lemma}[Séries formais formam anéis]
    Se $R$ é um anel comutativo, então $R\llbracket x \rrbracket$ é um anel comutativo.
\end{lemma}

\begin{proof}
    A operação de soma de $\mathbb R\llbracket x \rrbracket$ é a mesma de $\mathbb R^{\mathbb N}$, que já verificamos satisfazer as propriedades de grupo Abeliano. Assim, $R\llbracket x \rrbracket$ é um grupo abeliano sob a soma.

    Para as demais propriedades, fique $p(x), q(x), r(x) \in R\llbracket x \rrbracket$ e $i \in \mathbb N$.
    \begin{itemize}
        \item \textbf{Distributividade:} O $i$-ésimo coeficiente de $p(x)\cdot (q(x)+r(x))$ é:
        
        \[\sum_{j=0}^i p_{i-j}(q_j+r_j)=\sum_{j=0}^i p_{i-j}q_j+\sum_{j=0}^i p_{i-j}r_j.\]

        O que coincide com o $i$-ésimo coeficiente de $p(x)q(x)+p(x)r(x)$.
        \item \textbf{Elemento Neutro:} Temos que:
        \[p(x)\cdot 1=\sum_{i=0}^\infty\left(\sum_{j=0}^i p_{i-j}\delta_{0j}\right)x^i=\sum_{i=0}^\infty p_ix^i=p(x).\]
        \item \textbf{Comutatividade:} A $i$-ésima coordenada de $p(x)\cdot q(x)$ é $\sum_{j=0}^i p_{i-j}q_j=\sum(p_{i-j}q_j: j \in A_i)$, onde $A_i=\{0, \dots, i\}$. A função $\phi: A_i\rightarrow A_i$ dada por $\phi(j)=i-j$ é bijetora, pois é injetora e $A_i$ é finito. Assim:
        \[\sum_{j=0}^i p_{i-j}q_j=\sum_{j=0}^ip_{i-\phi(j)}q_{\phi(j)}=\sum_{j=0}^ip_{j}q_{i-j}=\sum_{j=0}^iq_{i-j}p_{j}.\]

        E esta é a $i$-ésima coordenada de $q(x)\cdot p(x)$.

        \item \textbf{Associatividade:} Temos que a $i$-ésima coordenada de $(p(x)\cdot q(x))\cdot r(x)$ é dada por:

        \[\pi_i((p(x)\cdot q(x))\cdot r(x))=\sum_{j=0}^i\pi_{i-j}(p(x)\cdot q(x))\cdot q_j=\sum_{j=0}^i\left(\sum_{k=0}^{i-j}p_{i-j-k}q_k\right) q_j\]
        \[=\sum_{j=0}^{i}\sum_{k=0}^{i-j}p_{i-j-k}q_kq_j=\sum \left(p_{i-j-k}q_kr_j:(j, k)\in A\right).\]

        Onde $A=\{(j, k): 0\leq j\leq i, 0\leq k\leq i-j\}$.

        Temos que a $i$-ésima coordenada de $p(x)\cdot (q(x)\cdot r(x))$ é dada por:

        $$\pi_i(p(x)\cdot (q(x)\cdot r(x)))=\sum_{s=0}^ip_{i-s}\pi_s(q(x)\cdot r(x))=\sum_{s=0}^ip_{i-s}\left(\sum_{t=0}^sq_{s-t}r_t\right)$$
        $$=\sum_{s=0}^i\sum_{t=0}^sp_{i-s}q_{s-t}r_t=\sum\left(q_{i-s}q_{s-t}r_t:(s, t)\in B\right)$$
        
        onde $B=\{(s, t): 0\leq t\leq s\leq i\}$. A função $\phi: A\rightarrow B$ dada por $\phi(j, k)=(j+k, j)$ é bijetora: é em $B$, pois $0\leq j\leq j+k\leq j+(i-j)=i$. É injetora, pois se $(j+k, j)=(j'+k', j')$ então $j=j'$ e, cancelando, $k=k'$. Finalmente, é sobrejetora, pois se $0\leq t\leq s\leq i$, sendo $j=t$ e $k=s-t$, temos que $0\leq j\leq i$, $0\leq k=s-t\leq i-t=i-j$ e $j+k=s$. Assim, $\phi$ é bijetora. Portanto:

        $$\sum\left(q_{i-s}q_{s-t}r_t:(s, t)\in B\right)=\sum\left(q_{i-(j+k)}q_{(j+k)-j}r_j:(j, k)\in A\right)$$$$=\sum\left(q_{i-j-k}q_{k}r_j:(j, k)\in A\right).$$
    \end{itemize}
\end{proof}

Note que, ao menos por enquanto, a letra $x$ é apenas parte da notação, e que não faz sentido, por enquanto, ``substituir $x$'' por algo.
\section{Anéis de Polinômios}
Na subseção anterior, introduzimos o anel das séries formais de um anel comutativo dado. Vimos que tal anel é um anel comutativo.

Deste anel, podemos extrair o anel de polinômios.

\begin{definition}
Seja $R$ um anel comutativo e $p(x)\in R\llbracket x \rrbracket$.

Define-se o \emph{suporte} de $p(x)$ por:

\[\supp p(x) =\{i \in I: p_i\neq 0\}.\]

Define-se o \emph{grau} de $p(x)$ por:
\[
\gr(p(x))=\begin{cases}
    \infty & \text{se } \supp p(x)  \text{ é infinito}\\
    -\infty  & \text{se } \supp p(x) =\emptyset \,(\text{se } p(x)=0)\\
    \max \supp p(x)  & \text{caso contrário.}
\end{cases}
\]
O \emph{anel de polinômios com coeficientes em $R$}, denotado por $R[x]$, é o subconjunto de $R\llbracket x \rrbracket$ dado por:
$$R[x]=\{p \in R\llbracket x \rrbracket: \gr(p) < \infty\}.$$

Se $p(x)\in R[x]$ é não nulo, define-se o \emph{coeficiente dominante} de $p(x)$ por $a_{\gr(p(x))}$.
Ou seja, o coeficiente dominante de $p(x)$ é seu coeficiente não nulo de mais alta posição.
\end{definition}

Assim, formalmente, o conjunto dos polinômios foi construído como sendo o conjunto de sequências eventualmente nulas de elementos de $R$.

\begin{lemma}
    Seja $R$ um anel comutativo. O anel de polinômios $R[x]$ é um subanel de $R\llbracket x \rrbracket$. Mais especificamente, dados $p(x), q(x) \in R[x]$:

    \begin{enumerate}[label=\alph*)]
        \item $\gr \left(p(x)q(x)\right)\leq\gr p(x)+\gr q(x)$, e a igualdade vale se $R$ for um domínio de integridade.
        \item $\gr\left(p(x)+q(x)\right)\leq \max\{\gr p(x),\gr(q(x))\}$.
        \item $\gr p(x)=\gr (-p(x))$.

    \end{enumerate}
\end{lemma}

\begin{proof}
    Ambas as afirmações são óbvias se $p(x)=0$ ou $q(x)=0$, então suponhamos que $p(x)$ e $q(x)$ são ambos não nulos.
    Sejam $n, m$ os graus de $p(x)$ e $q(x)$, respectivamente.
    
    Calculemos o coeficiente $n+m$ de $p(x)q(x)$.

    $$\pi_{n+m}(p(x)q(x))=\sum_{j=0}^{n+m}p_{n+m-j}q_j.$$

    Se $0\leq j< m$ temos que $n+m-j>n$, e $p_{n+m-j}=0$. Se $j>m$, temos que $q_j=0$. Assim, o único termo possivelmente não nulo da soma é quando $j=m$, que é $p_nq_m$. Este é não nulo $R$ for um domínio.
    
    Por outro lado, se $l>n+m$ temos que:

    $$\pi_{l}(p(x)q(x))=\sum_{j=0}^{l}p_{l-j}q_j.$$

    Se $0\leq j\leq m$ temos que $l-j>m+n-m=n$, e $p_{l-j}=0$. Se $j>m$, temos que $q_j=0$. Assim, todos os coeficientes da soma são $0$.
    Isso conclui que $\gr(p(x)q(x))\leq n+m$, sendo $n+m$ se $R$ for um domínio de integridade.

    Para a segunda afirmação, se $l>\max\{\gr p(x), \gr q(x)\}$, temos que o $l$-ésimo coeficiente de $p(x)+q(x)$ é $0$, pois este é $p_l+q_l=0+0$.

    A terceira afirmação é imediata.

    Agora, para a afirmação principal, as afirmações itemizadas nos mostram que $R[x]$ é fechado pela soma, produto e diferença de $R\llbracket x \rrbracket$. Finalmente, note que o grau da série $1=(1, 0, 0, \dots)$ é $0$, logo, $1\in R[x]$.
\end{proof}

Agora vamos trabalhar um pouco mais nossa notação.

Ao tratar de polinômios, intuitivamente, estamos tratando de expressões do tipo $a_0+a_1x+\dots+a_nx^n$, onde $n\geq 0$ e $a_i \in R$ para cada $i\leq n$.
De acordo com nosso formalismo até então, apesar de utilizarmos a notação de soma $\sum_{i=0}^\infty a_i x^i$ para tratar de séries formais e polinômios, os símbolos $\sum_{i=0}^\infty$ e $x^i$ são apenas, por ora, parte da notação, e não têm outro significado além deste. Porém, ao pensar em um polinômio $a_0+a_1x+\dots+a_nx^n$, pensamos que cada $a_i$ tem um significado individual, bem como $x$. A seguir, expandiremos a nossa notação a fim de formalizar essa ideia.

Vamos definir o elemento $x \in R[x]$, bem como identificar uma cópia de $R$ dentro de $R[x]$.

\begin{definition}
    Seja $R$ um anel comutativo. Em $R[x]$, seja $x=(0, 1, 0, 0, \dots)$ e, para cada $r \in R$, seja $\hat r=(r, 0, 0, \dots)$.
\end{definition}

Façamos uma pausa para discutir essa notação.

Ao pensar intuitivamente em polinômios, pensa-se em $R$ como um subconjunto de $R[x]$.
Porém, isso é, formalmente, falso: cada elemento de $R[x]$ é uma sequência de elementos de $R$, e, usualmente, uma sequência de elementos de $R$ não é um elemento de $R$.
Logo, no geral, $R\not \subseteq R[x]$.

Porém, conforme discutido acima, intuitivamente um polinômio é um objeto da forma $a_0+a_1x+\dots+a_nx^n$ onde cada $a_i \in R$.
Pondo $n=0$ e $a_0=r$, intuitivamente, $r\in R[x]$, o que sabemos, conforme discutido no parágrafo anterior, ser (provavelmente) falso.
Apesar disso, há uma identificação natural de $r$ em $R[x]$: o elemento $\hat r$, é o elemento de $R[x]$ que carrega o coeficiente $r$ na posição correspondente à potência $0$. Compare a definição formal de $\hat r$ com a ideia intuitiva de que $r$ ``deveria ser obtido'' pondo $n=0$ e $a_i=r$.

Já o elemento $x$ é o elemento de $R[x]$ carrega o coeficiente $1$ na posição correspondente à potência $1$. Intuitivamente, $x$ ``deveria ser'' ele é obtido colocando-se $n=2$, $a_0=0$ e $a_1=1$ na expressão $a_0+\dots+a_nx^n$. Compare esse raciocínio intuitivo com a definição formal de $x$ apresentada.

O lema abaixo mostra que os elementos $\hat r$ de fato agem como uma cópia de $R$ dentro de $R[x]$.
Devido a ele, mais a frente, apesar de não ser verdade, formalmente, que $R$ é um subconjunto de $R[x]$, ``identifica-se'' $a \in R$ com $\hat a$, abandonando-se a notação $\hat r$ em favor de escrever simplesmente $r$, e considerando-se $R$ como um subconjunto de $R[x]$, mesmo não tendo sido construído de modo que isso valha.

Um modo de melhor formalizar esse raciocínio é construir um novo anel $R[x]''$ isomorfo à $R[x]$ em que $R$ é, de fato, um subconjunto de $R[x]$.
Para isso, toma-se um conjunto disjunto de $R$ que bijeta com $R[x]\setminus\{\hat r: r \in R\}$ e tranferimos a estrutura de $R[x]$ para $R\cup A$ preservando as operações de $R$, de modo a torná-los isomorfos.
Não faremos esse trabalho aqui por envolver tecnicalidades que fogem do escopo deste texto e recomendamos ao leitor que, nesse momento, não se preocupe com tais tecnicalidades.

\begin{lemma}
    Na notação anterior, seja $\phi:R\rightarrow R[x]$ dada por $\phi(r)=\hat r$. Então $h$ é um homomorfismo injetor.
\end{lemma}
    
\begin{proof}
    Sejam $r, s \in R$. Então:
    \begin{itemize}
        \item $\phi(r+s)=(r+s, 0, 0, \dots)=\hat r+\hat s=\phi(r)+\phi(s)$.
        \item $\phi(rs)=(rs, 0, 0, \dots)=\hat r\cdot \hat s=\phi(r)\cdot \phi(s)$.
        \item $\phi(1_R)=\hat 1=(1_R, 0, 0, \dots)=1_{R[x]}=1_{R[x]}$.
    \end{itemize}

    A injetividade é óbvia.
\end{proof}

Agora vejamos que o elemento $x$ se comporta conforme esperado.

\begin{lemma}
    Na notação anterior, para todo $r \in R$ e $n, i\geq 0$:

    $$\pi_i(\hat r x^n)(i)=\begin{cases}
        0 & \text{se } i\neq n\\
        r & \text{se } i=n.
    \end{cases}$$
    Ou seja, $\hat r x^n=(0, 0, \dots, r, 0, \dots)$ onde o $r$ está na posição $n$.
\end{lemma}

\begin{proof}
    Fixe $r$ Seguimos por indução. Para $n=0$, temos que $\hat rx^0=\hat r=(r, 0, 0, \dots)$ e para $n=1$ temos que $\hat r x^1=x=(0, r, 0, \dots)$.

    Para o passo $n+1$, onde $n\geq 1$, temos que, sendo $i\geq 1$:

    \[\pi_i(\hat rx^{n+1})=\pi_i((\hat rx^n)\cdot x)=\sum_{j=0}^i\pi_{i-j}(\hat r x^n)\cdot \pi_j(x)=\pi_{i-1}(\hat r x^n).\]

    Assim, se $i=n+1$, temos que a coordenada é $r$, e $0$ caso contrário. Resta apenas verificar que a coordenada $0$ é $0$. Ora, a coordenada $0$ se dá por $\pi_0(\hat r x^n)\pi_0(x)=0$.
\end{proof}

Agora veremos que todo elemento de $R[x]$ se escreve como uma soma de elementos de $R$ e potências de $x$, que é o esperado quando pensamos em polinômios.
\begin{prop}
    Na notação anterior, para todo $p(x)\in R[x]$ e $n\geq \gr p(x)$, existem únicos $r_0, r_1, \dots, r_n\in R$ tais que $p(x)=\sum_{i=0}^n \hat r_ix^i$, e estes são os coeficientes de $p(x)$.
\end{prop}
    
\begin{proof}
    Para a unicidade, note que se $p(x)=\sum_{i=0}^n \hat r_ix^i$, então para cada $i\leq n$, o $i$-ésimo coeficiente do lado direito é o $i$-ésimo coeficiente de $\hat r_ix^i$, que é $r_i$.
    Logo, $r_i$ é o $i$-ésimo coeficiente de $p(x)$.

    Para a existência, note que $p(x)=(p_0, p_1, \dots, p_n, 0, 0, \dots)=(p_0, 0, \dots)+(0, p_1, 0, \dots)+\dots+(0, 0, \dots, p_n)=p_0x^0+p_1x^1+\dots+p_nx^n=\sum_{i=0}^n \hat p_ix^i$.
\end{proof}
Note que, diferente do que ocorre na notação inicial sobre séries formais, a notação $\sum_{i=0}^n \hat r_i x^i$ expressa, de fato, uma soma (finita). Além disso, vale a comparação coeficiente-a-coeficiente.
\begin{corol}
    Na notação anterior, se $r_1, \dots, r_n$ e $s_1, \dots, s_n$ são elementos de $R$, então:

    $\sum_{i=0}^n \hat r_i x^i=\sum_{i=0}^n \hat s_i x^i$ se, e somente se, $r_i=s_i$ para todo $i\leq n$.
\end{corol}
Notação: abandona-se $\hat r$ em favor de $r$, mesmo havendo ambiguidade de notação.
\section{A propriedade universal do Anel de Polinômios}

Se $p(x)$ é um polinômio, esperamos poder "substituir" $x$ por um elemento $r$ de $R$, a fim de obter um elemento de $R$.


A proposição abaixo formaliza e generaliza essa ideia.
\begin{prop}[Propriedade universal do anel de polinômios]
    Seja $R$ um anel comutativo. Então
    $R[x]$ é um anel comutativo que satisfaz a seguinte propriedade:

    Para todo anel $S$, todo homomorfismo $f:R\rightarrow S$ e todo $s \in S$, existe um único homomorfismo $g:R[x]\rightarrow S$ tal que $g\circ \phi=f$ e $g(x)=s$.
\end{prop}
\begin{proof}
    Já vimos que $R[x]$ é um anel comutativo.

    Defina $g:R[x]\rightarrow S$ por $g(p(x))=\sum_{i=0}^n f(r_i)s^i$, onde $p(x)=\sum_{i=0}^n r_ix^i$.
    Note que $g$ é bem definido, pois se $p(x)=\sum_{i=0}^n r_ix^i=\sum_{i=0}^n s_ix^i=q(x)$, então $r_i=s_i$ para todo $i\leq n$.

    $g$ é homomorfismo, pois, dados $p(x)=\sum_{i=0}^n r_ix^i$ e $q(x)=\sum_{i=0}^m s_ix^i$, escrevendo $a_i=0$ para $i>n$ e $b_i=0$ para $i>m$, temos que:
    \begin{align*}
        g(p(x)+q(x))&=g\left(\sum_{i=0}^{\max\{n, m\}}(r_i+s_i)x^i\right)\\
        &=\sum_{i=0}^{\max\{n, m\}}f(r_i+s_i)s^i\\
        &=\sum_{i=0}^{n}f(r_i)s^i+\sum_{i=0}^{m}f(s_i)s^i\\
        &=g(p(x))+g(q(x)).
    \end{align*}
    \begin{align*}
        g(p(x)q(x))&=g\left(\sum_{i=0}^{n+m}\left(\sum_{j=0}^i r_{i-j}s_j\right)x^i\right)\\
        &=\sum_{i=0}^{n+m}\left(\sum_{j=0}^i f(r_{i-j})f(s_j)\right)s^i\\
        &=\sum_{i=0}^{n}f(r_i)s^i\cdot \sum_{j=0}^m f(s_j)s^j\\
        &=g(p(x))g(q(x)).
    \end{align*}
    \begin{align*}
        g(1_{R[x]})
        &=\sum_{i=0}^0 f(1_R)s^i=1_S.
    \end{align*}

    A função $g$ é única, pois se $g'$ é outra tal função, temos que, dado $p(x)=\sum_{i=0}^n r_ix^i$, temos que $g'(p(x))=\sum_{i=0}^nf(r_i)s^i=g(p(x))$.
\end{proof}

\begin{exemplo}
Seja $R$ um anel comutativo e $R[x]$ o anel de polinômios sobre $R$.
Para todo $s \in R$, existe um homomorfismo $\av_s:R[x]\rightarrow R$ tal que $\av_s(r)=r$ para todo $r \in R$ e $\av_s(x)=s$, de modo que para todo $p(x)=\sum_{i=0}^n r_ix^i\in R[x]$, $\av_s(p(x))=\sum_{i=0}^n r_is^i$.

Esse homomorfismo é chamado de \emph{avaliação em $s$}, e escrevemos $p(s)=\av_s(p(x))$.
\end{exemplo}
Uma forma de definir um anel de polinômios sem se referir a nenhuma peculiaridade de alguma construção particular é a partir de alguma propriedade que tem dentro da categoria dos anéis que apenas ele possui (a menos de isomorfismos).

Uma tal forma é a seguinte.
\begin{definition}\label{definition:polinomio_propUniversal}
    Seja $R$ um anel comutativo e $X$ um conjunto.
    Um anel de polinômios sobre $R$ com variáveis em $X$ é uma tripla $(R[X], \phi, \alpha)$, onde $R[X]$ é um anel comutativo e $\phi:R\rightarrow R[X]$ é um homomorfismo e $\alpha:X\rightarrow R[X]$ é uma função que satisfazem a seguinte propriedade:

    Para toda tripla $(S, f, \beta)$ onde $S$ é um anel comutativo, $f:R\rightarrow S$ é um homomorfismo e$\beta:X\rightarrow S$, existe um único homomorfismo $g:R[X]\rightarrow S$ tal que $g\circ \phi=f$ e $g\circ \alpha=\beta$.
    \begin{figure}[H]
        \centering
        \begin{tikzcd}[row sep=1.5cm, column sep=2cm]
            R \arrow[r, "\phi"]\arrow[rd, "f"']  & R[X] \arrow[d, dashed, "\exists!g"] & X \arrow[l, "\alpha"']\arrow[ld, "\beta"]\\
            & S &
        \end{tikzcd}
    \end{figure}
\end{definition}

\begin{prop}\label{prop:polinomio_universalUnicidade}
Seja $R$ um anel comutativo e $X$ um conjunto. Se $(R[X], \phi, \alpha)$ e $(\overline{R[X]}, \bar \phi, \bar \alpha)$ são anéis de polinômios sobre $R$ com variáveis em $X$, então $R[X]$ e $\overline{R[X]}$ são isomorfos por um isomorfismo $g:R[X]\rightarrow \overline{R[X]}$ tal que $g\circ \alpha=\bar \alpha$ e $g\circ \phi=\bar \phi$.
\end{prop}
\begin{proof}
    Aplicando a propriedade universal $R[X]$ para $(\overline{R[X]}, \overline \phi, \overline \alpha)$, existe um homomorfismo $g:R[X]\rightarrow \overline{R[X]}$ tal que $g\circ \phi=\bar \phi$ e $g\circ \alpha=\bar \alpha$.

    \begin{figure}[H]
        \centering
        \begin{tikzcd}[row sep=1.5cm, column sep=2cm]
            R \arrow[r, "\phi"]\arrow[rd, "\bar \phi"']  & R[X] \arrow[d, dashed, "g"] & X \arrow[l, "\alpha"']\arrow[ld, "\bar\alpha"]\\
            & \overline{R[X]} &
        \end{tikzcd}
    \end{figure}
    Aplicando a propriedade universal $\overline{R[X]}$ para $(R[X], \phi, \alpha)$, existe um homomorfismo $h:\overline{R[X]}\rightarrow R[X]$ tal que $h\circ\bar \phi=\phi$ e $h\circ \bar\alpha=\alpha$.
    \begin{figure}[H]
        \centering
        \begin{tikzcd}[row sep=1.5cm, column sep=2cm]
            R \arrow[r, "\bar \phi"]\arrow[rd, " \phi"']  & \overline{R[X]} \arrow[d, dashed, "h"] & X \arrow[l, "\bar \alpha"']\arrow[ld, "\alpha"]\\
            & R[X] &
        \end{tikzcd}
    \end{figure}

    Aplicando a propriedade de $R[X]$ para a tripla $(R[X], \phi, \alpha)$, existe um único homomorfismo $u:R[X]\rightarrow R[X]$ tal que $u\circ \phi=\phi$ e $u\circ \alpha=\alpha$.
    É imediato que $\id_{R[X]}$ satisfaz essas propriedades. O mapa $h\circ g$ também satisfaz, pois $h\circ g\circ \phi=h\circ \bar \phi=\phi$ e $h\circ g\circ \alpha=h\circ \bar \alpha=\alpha$.
    \begin{figure}[H]
        \centering
        \begin{tikzcd}[row sep=1.5cm, column sep=2cm]
            R \arrow[r, "\phi"]\arrow[rd, " \phi"']  & R[X] \arrow[d, dashed, "\id_{R[X]}", "h\circ g"'] & X \arrow[l, "\alpha"']\arrow[ld, "\alpha"]\\
            & R[X] &
        \end{tikzcd}
    \end{figure}    
    Aplicando a propriedade de $\overline{R[X]}$ para a tripla $(\overline{R[X]}, \bar \phi, \bar \alpha)$, existe um único homomorfismo $v:\overline{R[X]}\rightarrow \overline{R[X]}$ tal que $v\circ \bar \phi=\bar \phi$ e $v\circ \bar \alpha=\bar \alpha$.
    É imediato que $\id_{\overline{R[X]}}$ satisfaz essas propriedades. O mapa $g\circ h$ também satisfaz, pois $g\circ h\circ \bar \phi=g\circ \phi=\bar \phi$ e $g\circ h\circ \bar \alpha=g\circ \bar \alpha=\bar \alpha$.

    \begin{figure}[H]
        \centering
        \begin{tikzcd}[row sep=1.5cm, column sep=2cm]
            R \arrow[r, "\bar \phi"]\arrow[rd, " \bar \phi"']  & \overline{R[X]} \arrow[d, dashed, "\id_{\overline{R[X]}}", "g\circ h"'] & X \arrow[l, "\bar \alpha"']\arrow[ld, "\bar \alpha"]\\
            & \overline{R[X]} &
        \end{tikzcd}
    \end{figure}    

    Assim, $g\circ h=\id_{\overline{R[X]}}$ e $h\circ g=\id_{R[X]}$. Portanto, $g$ é um isomorfismo de anéis, e $h$ é o inverso de $g$.
\end{proof}

\begin{corol}
    Seja $R$ um anel comutativo e $X$, $Y$ conjuntos de mesmas cardinalidades.

    Se $(R[X], \phi, \alpha)$ e $(R[Y], \bar \phi, \bar \alpha)$ são anéis de polinômios sobre $R$ com variáveis em $X$ e $Y$, respectivamente, então $R[X]$ e $R[Y]$ são isomorfos.
\end{corol}

\begin{proof}
Seja $\theta:Y\rightarrow X$ uma bijeção. Basta ver que $(R[X], \phi, \alpha\circ \theta)$ é um anel de polinômios sobre $R$ com variáveis em $Y$.

Seja $S$ um anel comutativo e $f:R\rightarrow S$ um homomorfismo, e $\beta:Y\rightarrow S$ uma função.
Temos que $\beta'=\beta\circ \theta^{-1}:X\rightarrow S$ é uma função.
Então existe um único homomorfismo $g: R[X]\rightarrow S$ tal que $g\circ \phi=f$ e $g\circ \alpha=\beta'$.
Desta última, $\alpha\circ \theta=\beta$.

Para ver que $g$ é o único tal homomorfismo, se $h: R[Y]\rightarrow S$ é um homomorfismo tal que $h\circ \phi=f$ e $h\circ \alpha\circ \theta=\beta$, então $h\circ \alpha=\beta'$.
Pela unicidade de $g$, segue que $g=h$.
\end{proof}
\begin{prop}
    Seja $R$ um anel comutativo e $X$ um conjunto. Então $R[x, y]\approx R[x][y]$.
\end{prop}
\begin{proof}
    Seja $X=\{x, y\}$, $\psi:R[x]$ a imersão canônica, $\psi': R[x]\rightarrow R[x][y]$ a imersão, $\phi=\psi'\circ \psi$, $\alpha(x)=x\in R[x][y]$ e $\alpha(y)=y\in R[x][y]$. e $\alpha(x)=\psi'(x)\in R[x]$.
    Então, $(R[x][y], \phi, \alpha)$ é um anel de polinômios sobre $R$ com variáveis em $X$.

    Com efeito, seja $S$ um anel comutativo, $f:R\rightarrow S$ um homomorfismo e $\beta:X\rightarrow S$ uma função.
    Então, existe um único homomorfismo $g:R[x]\rightarrow S$ tal que $g\circ \psi=f$ e $g(x)=\beta(x)$.
    Logo, existe um único homomorfismo $h:R[x][y]\rightarrow S$ tal que $h\circ \psi'=g$ e $h(y)=\beta(y)$.
    Assim, $h\circ \phi=h\circ(\psi'\circ \psi)=(h\circ \psi')\circ \psi=g\circ \psi=f$, $h(\alpha(y))=h(y)=\beta(y)$ e $h(\alpha(x))=h\circ \psi'(x)=g(x)=\beta(x)$.

    Para ver a unicidade de $h$, se $\bar h$ é outra tal função, temos que $\bar h\circ \psi=(\bar h\circ \psi')\circ \psi=f$ e $\bar h\circ \alpha(x)=\bar h\circ \psi'(x)=\beta(x)$, logo, $\bar h\circ \psi'=g$.
    Além disso, $\bar h\circ \alpha=\beta$, logo, $\bar h=h$.
\end{proof}

\begin{corol}
Sejam $R, S$ anéis comutativos com unidade e $f: R\rightarrow S$ um isomorfismo.
Então existe um único isomorfismo $\bar f:R[x]\rightarrow S[x]$ que satisfaz $\bar f(\sum_{i=0}^n r_ix^i)=\sum_{i=0}^nf(a_i)x^i$ para todos $n\geq 0$ e $r_0, \dots, r_n\in R$.
\end{corol}

\begin{proof}
    Identificando $R\subseteq R[x]$ e $S\subseteq S[x]$, considere o único homomorfismo $\bar f:R[x]\rightarrow S[x]$ tal que $\bar f|_R=f$ e $f(x)=x\in S[x]$. Então $\bar f(\sum_{i=0}^n r_ix^i)=\sum_{i=0}^nf(a_i)x^i$.

    Seja $g=f^{-1}$ e seja $\bar g:S[x]\rightarrow R[x]$ o homomorfismo tal que $\bar g|_S=g$ e $\bar g(x)=x\in R[x]$. Segue que $g\circ f|_R=\id_R$ $\bar g\circ \bar f(x)=x$, bem como $\id_{R[x]}$, logo, $\bar g\circ \bar f=\id_{R[x]}$. Analogamente, $f\circ g=\id_{S[x]}$.
    Logo, $\bar f$ é isomorfismo.

    A unicidade de $\bar f$ vem do fato de que este é o único homomorfismo que satisfaz $\bar f|_R=f$ e $\bar f(x)=x$.
\end{proof}

\begin{definition}
    Na notação acima, para $p \in R[x]$, $\bar f(p)$ é chamado de $f(p)$.
\end{definition}
\section{Polinômios de várias variáveis}
Nesta seção, daremos uma construção particular para $R[X]$, onde $X$ é um conjunto arbitrário de indeterminadas e $R$ é um anel comutativo.

Se $x, y, z \in X$, precisamos dar sentido às expressões como $x^2y^3z^4$ e $xy^2+xz^3$. A definição abaixo cuida disso.

\begin{definition}
    Seja $X$ um conjunto qualquer. O conjunto $\mathbb N^{(X)}$ é o conjunto de funções $u:X\rightarrow \mathbb N$ tal que $\supp u=\{x \in X: f(x)\neq 0\}$ é finito.

    Se $u, v \in \mathbb N^{(X)}$, define-se $u*v:X\rightarrow \mathbb N$ dada por $(u+v)(x)=u(x)+v(x)$.
\end{definition}

\begin{lemma}
    A operação $*$ definida acima é tal que $\supp(u*v)= \supp u \cup \supp v$, é associativa e comutativa. Além disso, dado $w \in \mathbb N^{(X)}$, temos que $\{(u, v): u*v=w\}$ é finito.
\end{lemma}
\begin{proof}
    A única afirmação não imediata é a última, que provaremos a seguir.
    Se $u*v=w$, temos que $\supp u, \supp v\subseteq \supp w$, que é finito. Assim, $u, v$ são $0$ fora de $\supp w$, e não podem ter valor maior do que $w(x)$ para $x\in \supp w$.
    Portanto, há no máximo $N=\prod_{x \in \supp w}(w(x)+1)$ escolhas para $u, v$, logo, no máximo $N^2$ escolhas para $(u, v)$.

    Observação: o número $N^2$ acima é uma cota superior para o número de pares $(u, v)$ tal que $u*v=w$, mas, no geral, ele é substancialmente menor do que este.
\end{proof}

Agora definiremos o anel das séries formais de $R$ com variáveis em $X$.
A ideia é identificar cada elemento de $\mathbb N^{(X)}$ com um monômio.
Por exemplo, se $X=\{x, y, z\}$ e $u(x)=2$, $u(y)=3$ e $u(z)=4$, pensamos em $u$ como $x^2y^3z^4=y^3z^4x^2=\dots$.
Assim, se temos ainda que $v(x)=1$, $v(y)=4$ e $v(z)=0$, temos que $u*v=(x^2y^3z^4)*(xy^4z^0)=x^3y^7z^4$, como é esperado.

Desta forma, o que buscamos é, para se obter uma série formal com coeficientes em um anel comutativo $R$ com variáveis em $X$, é associar a cada elemento de $\mathbb N^{(X)}$ (cada monômio), um elemento de $R$.
\begin{definition}
    Seja $R$ um anel comutativo e $X$ um conjunto. O anel das séries formais de $R$ com variáveis em $X$, denotado por $R\llbracket X\rrbracket$, é o conjunto de funções $f:\mathbb N^{(X)}\rightarrow R$. Um elemento $p=(p_u)_{u \in \mathbb N}\in R\llbracket X\rrbracket$ é denotado por:

    \[p=\sum_{u \in \mathbb N^{(X)}}p_u u.\]

    A série nula é dada pela família $0=(0)_{u \in \mathbb N^{(X)}}$.
    O monômio $1$ é a função nula $0:X\rightarrow \mathbb N$. O elemento $1$ de $R\llbracket X\rrbracket$ é a função $1=(\delta_{1u})_{u \in \mathbb N^{(X)}}$, onde $\delta_{1u}$ é o delta de Kronecker, que vale $1$ se $u=1$ e $0$ caso contrário.

    A soma e produto de séries formais se dá por:

    \[\sum_{u \in \mathbb N^{(X)}}p_u u+\sum_{u \in \mathbb N^{(X)}}q_u u=\sum_{u \in \mathbb N^{(X)}}(p_u+q_u)u\]
    \[\sum_{u \in \mathbb N^{(X)}}p_u u\cdot \sum_{u \in \mathbb N^{(X)}}q_u u=\sum_{u \in \mathbb N^{(X)}}\sum_{v*w=u}(p_vq_w)u\]
\end{definition}

\begin{lemma}[Séries formais formam anéis]
    Se $R$ é um anel comutativo e $X$ um conjunto, então $R\llbracket X \rrbracket$ é um anel comutativo.
\end{lemma}

\begin{proof}
    A operação de soma de $\mathbb R\llbracket x \rrbracket$ é a mesma de $\mathbb R^{\mathbb N^{(X)}}$, que já verificamos satisfazer as propriedades de grupo Abeliano. Assim, $R\llbracket x \rrbracket$ é um grupo abeliano sob a soma.

    Para as demais propriedades, fique $p, q, r \in R\llbracket X \rrbracket$.
    \begin{itemize}
        \item \textbf{Distributividade:} Fixe $u \in \mathbb N^{(X)}$. O $u$-ésimo coeficiente de $p\cdot (q+r)$ é:
        
        \[\sum_{v*w=u} p_{v}(q_w+r_w)=\sum_{v*w=u}p_vq_w+\sum_{v*w=u}p_vr_w\]

        O que coincide com o $u$-ésimo coeficiente de $pq+pr$.
        \item \textbf{Elemento Neutro:} Temos que:
        \[p\cdot 1=\sum_{u \in \mathbb N^{(X)}}\left(\sum_{v*w=u} p_{v}\delta_{1w}\right)u=\sum_{u \in \mathbb N^{(X)}}p_{u}u=p\]

        \item \textbf{Comutatividade:} A $u$-ésima coordenada de $pq$ é $\sum_{v*w=u} p_vq_w=\sum(p_vq_w: (v, w)\in A)$, onde $A=\{(v, w)\in \mathbb N^{(X)}\times \mathbb N^{(X)}: v*w=u\}$. A função $\phi: A\rightarrow A$ dada por $\phi(v, w)=(w, v)$ é bijetora, pois é injetora e $A$ é finito. Assim:
        \[\sum(p_vq_w: (v, w)\in A)=\sum(p_wq_v: (v, w)\in A)=\sum(q_vp_w: (v, w)\in A)=\sum_{v*w=u}q_vp_w,\]

        que é a $u$-ésima coordenada de $qp$.
    
        \item \textbf{Associatividade:} Temos que a $u$-ésima coordenada de $((pq)r$ é dada por:

        \[\sum_{v*w=u}\pi_v(pq)r_wu=\sum_{v*w=u}\sum_{s*t=v}p_sq_tr_wu=\sum(p_sq_tr_w: (s, t, v, w)\in A),\]

        onde $A=\{(s, t, v, w): s*t=v, v*w=u\}$. Seja $B=\{(a, b, c): a*b*c=u\}$. A função $\phi:B\rightarrow A$ dada por $\phi(a, b, c)=(a, b, a*b, c)$ é bijetora (verifique), logo:

        \[\sum(p_sq_tr_w: (s, t, v, w)\in A)=\sum(p_aq_br_c: (a, b, c)\in A)=\sum_{a*b*c=u}p_aq_br_c.\]

        Analogamente, o $u$-ésimo coeficiente de $p(qr)$ é dado pela expressão acima.
    \end{itemize}
\end{proof}

\begin{definition}
    Seja $R$ um anel comutativo e $X$ um conjunto. O anel de polinômios com coeficientes em $R$ e variáveis em $X$, denotado por $R[X]$, é o subanel de $R\llbracket X \rrbracket$ formado pelos elementos $p\in R\llbracket X \rrbracket$ que têm suporte finito, onde $\supp p=\{u \in \mathbb N^{(X)} \in X: p_u\neq 0\}$.
\end{definition}

\begin{lemma}
    Se $R$ é um anel comutativo e $X$ um conjunto, então $R[X]$ é um subanel de $R\llbracket X \rrbracket$.
\end{lemma}
\begin{proof}
    A soma e o produto de polinômios são polinômios, pois se $p=\sum_{u \in \mathbb N^{(X)}}p_u u$ e $q=\sum_{u \in \mathbb N^{(X)}}q_u u$, então $\supp(p+q)\subseteq\supp p\cup \supp q$ e $\supp(pq)=\{u \in \mathbb N^{(X)}: u=v*w, v\in \supp p, w\in \supp q\}$, que é finito. Além disso, as séries $0$ e $1$ são polinômios.
\end{proof}
\section{Divisibilidade em anéis de polinômios}
\begin{prop}\label{prop:polinomio_Euclideano}
Seja $R$ é um anel comutativo E $p(x), d(x) \in R[x]$ com $d(x)\neq 0$ e o coeficiente dominante de $d$ invertível. Então existem $q(x), r(x) \in R[x]$ tais que $p(x)=d(x)q(x)+r(x)$, onde $\deg(r(x))<\deg(d(x))$.

Além disso, se $R$ é domínio de integridade, então $q(x), r(x)$ são únicos.
\end{prop}
\begin{proof}
    Existência: fixe $d(x)$ e seja $\deg(d(x))=k\geq 0$.

    Se $p(x)=0$, seja $q(x)=r(x)=0$. Note que $\gr(0)=-\infty<\gr(d(x))$.

    Se $p(x)\neq 0$, procedemos por indução em $m=\gr(p(x))$.

    Para $m=0$: se $0<k$, seja $q(x)=0$ e $r(x)=p(x)$. Então $p(x)=d(x)q(x)+r(x)$ e $\gr(r(x))=0<k=\gr(d(x))$.

    Se $0=k$, então $p(x)=a\in K$, $d(x)=b\in K^*$. Seja $q(x)=ab^{-1}$ e $r(x)=0$. Então $\gr(r(x))=-\infty<0=\gr(d(x))$ e $0<k$.
    Provaremos por indução em $m$ que para todo $p(x)\in R[x]$ de grau $m$, existem $q(x), r(x) \in R[x]$ com $p(x)=q(x)d(x)+r(x)$ e $\gr(r(x))<\gr(d(x))$.

    Agora vamos supor que a hipótese vale para $m$. Provaremos que vale para $m+1$.
    Para $m+1<k$, seja $p(x)=r(x)$ e $q(x)=0$. Então $q(x)d(x)+r(x)=r(x)=p(x)$ e $\gr(r(x))=m+1<k=\gr(d(x))$.

    Se $m+1\geq k$, seja $a$ o coeficiente dominante de $p(x)$ e $b$ o coeficiente dominante de $d(x)$.
    Temos que $p(x)-\frac{a}{b}d(x)x^{m+1-k}$ é um polinômio de grau $\leq n$, logo, pela hipótese indutiva, existem $q(x), r(x)$ tais que $p(x)-\frac{a}{b}d(x)x^{m+1-k}=d(x)q(x)+r(x)$, onde $\deg(r(x))<k$.
    Segue que $p(x)=\left(\frac{a}{b}d(x)x^{m+1-k}+q(x)\right)d(x)+r(x)$, onde $\deg(r(x))<k$.

    Para a unicidade caso $R$ seja domínio, se existem $q_1(x), r_1(x)$ e $q_2(x), r_2(x)$ tais que $p(x)=d(x)q_1(x)+r_1(x)=d(x)q_2(x)+r_2(x)$, temos que $d(x)(q_1(x)-q_2(x))=r_2(x)-r_1(x)$.
    Se $q_1(x)-q_2(x)=0$ segue a tese.

    Caso contrário, temos $\gr(r_2(x)-r_1(x))<\gr(d(x))\leq\gr(d(x)(q_1(x)-q_2(x)))=\gr(r_2(x)-r_1(x))$, o que é uma contradição.
\end{proof}
\begin{corol}
Seja $K$ um corpo. Então $K[x]$ é um domínio Euclideano.
\end{corol}
\begin{proof}
Já vimos que se $K$ é um corpo, então $K[x]$ é um domínio de integridade.

Para completar a prova, note que para todos $p(x), q(x) \in K[x]\setminus \{0\}$, $\gr(p(x)q(x))=\gr(p(x))+\gr(q(x))\geq \gr(p(x))$.
\end{proof}

\begin{corol}
Se $R$ é um anel comutativo, $a \in R$ e $p(x)\in R[x]$, então existe $q(x)\in R[x]$ tal que $p(x)=q(x)(x-a)+p(a)$.
\end{corol}
\begin{proof}
    Como $1 \in R[x]$, existem $q(x), r(x) \in R[x]$ tais que $p(x)=q(x)(x-a)+r(x)$ e $\deg(r(x))\leq 0$. Assim, $r(x)=b$ para algum $b \in R$, logo, $p(x)=q(x)(x-a)+b$. Como a avaliação em $a$ é homomorfismo, note que $p(a)=b$. Assim, $p(x)=q(x)(x-a)+p(a)$.
\end{proof}
\begin{prop}
    Seja $R$ um anel comutativo e $p(x)\in R[x]$.
    Temos que $(x-a)|p(x)$ se, e somente se, $p(a)=0$.
\end{prop}
\begin{proof}
    Se $p(a)=0$, segue do corolário anterior que $(x-a)|p(x)$.

    Reciprocamente, se $(x-a)|p(x)$, escreva $p(x)=q'(x)(x-a)$. Segue que $p(a)=q'(a)(a-a)=0$.
\end{proof}

\begin{corol}
Seja $R$ um domínio de integridade e $a \in R$. Então $x-a\in R[x]$ é primo (e, portanto, irredutível).
\end{corol}
\begin{proof}
Suponha que $(x-a)|p(x)q(x)$. Então $p(a)q(a)=0$, logo, $p(a)=0$ ou $q(a)=0$. Assim, $(x-a)|p(x)$ ou $(x-a)|q(x)$.
\end{proof}

\begin{prop}
Seja $R$ um domínio de integridade. Os únicos elementos invertíveis de $R[x]$ são os elementos invertíveis de $R$.
\end{prop}
\begin{proof}
    Se $a \in R$ é invertível, então $aa^{-1}=1$ também em $R[x]$.
    Se $a \in R[x]$ e existe $b$ tal que $ab=1$, temos que $\deg(a)+\deg(b)=0$, logo, $\deg (a)=\deg b=0$ e, assim, $a, b \in R$, e, portanto, $a, b \in R^*$.
\end{proof}
\section{Raízes de polinômios}
\begin{definition}
Seja $R$ um domínio de integridade e $p(x)\in R[x]$ e $a \in R$. Dizemos que $a$ é uma raiz de $p(x)$ se $p(a)=0$.
\end{definition}

\begin{prop}
Seja $R$ um domínio de integridade e $p(x)\in R[x]\setminus \{0\}$. Seja $m=\gr (p(x))$. Então $R[x]$ tem no máximo $m$ raízes.
\end{prop}
\begin{proof}
    Provaremos por indução em $m$.
    Caso $m=0$, escreva $p(a)=b\neq 0$. Temos que para todo $a \in R$ $p(a)=b\neq 0$, logo, $p(x)$ tem $0$ raízes.

    Suponha que a afirmação vale para $m$. Provaremos que vale para $m+1$.
    
    Seja $p(x)$ de grau $m+1$. Caso $p(x)$ não possua raízes, a prova está concluída. Caso contrário, seja $a$ uma raiz de $p(x)$. Então $(x-a)|p(x)$, logo, existe $q(x) \in R[x]$ tal que $p(x)=q(x)(x-a)$

    Como $\gr(p(x))=\gr(q(x))+1$, temos por hipótese indutiva que $q(x)$ tem no máximo $m$ raízes. Observe que se $b\neq a$ é raiz de $p(x)$, então $(b-a)q(b)=0$, logo, $b$ é raiz de $q(x)$. Assim, $p(x)$ tem no máximo $m+1$ raízes.
\end{proof}

Se $R$ é um domínio de integridade e $K$ é seu corpo de frações, podemos identificar $R$ dentro de $K$ como subanel. Assim, $R[x]\subseteq K[x]$.
\begin{prop}
Seja $R$ um domínio de fatoração única e $F$ o corpo de frações de $R$. Seja $p(x)\in R[x]\subseteq K[x]$ não nulo e $u, v \in D$ com $v \neq 0$ e $1\in \MDC(u, v)$.

Então, se $\frac{u}{v}$ é uma raiz de $p(x)=a_nx^n+\dots+a_1x+a_0$ onde $n=\gr p(x)$, temos que $u|a_0$ e $v|a_n$.
\end{prop}
\begin{proof}
    Temos que $0=\sum_{i=0}^na_i \frac{u^i}{v^i}$. Multiplicando por $v^n$, temos que $0=\sum_{i=0}^n a_i u^iv^{n-i}$.

    Como $p(x)$ tem raízes, $n>0$, $-a_nu^n=\sum_{i=0}^{n-1}a_iu^iv^{n-i}=v\sum_{i=0}^{n-1}a_iu^iv^{n-i-1}$. Como $v|a_n u^n$ e $1\in \MDC(v, u)$ e estamos em um domínio de fatoração única, temos que $v|a_n$.

    Similarmente, $-a_0v^n=\sum_{i=1}^n a_i u^iv^{n-i}=u\sum_{i=1}^n a_i u^{i-1}v^{n-i}$. Como $u|a_0v^n$ e $1\in \MDC(u, v)$ e estamos em um domínio de fatoração única, temos que $u|a_0$.
\end{proof}


\begin{exemplo}
Em $\mathbb Q[x]$, considere $p(x)=2x^3- x^2 - 4 x  + 2$. Caso $p(x)\in \mathbb Z[x]$ possua uma raiz racional, ela deve ser da forma $\frac{a}{b}$ com $a|2$ e $b|2$ com $a, b$ primos entre si. As únicas opções de raízes são $\frac{1}{2}$,$-\frac{1}{2}$, $1$, $-1$, $2$, $-2$.
\end{exemplo}
\section{Funções Polinomiais}
Dado um anel comutativo $R$ e $a_0, \dots, a_n\in R$, é possível definir $f:R\rightarrow R$ dada por $f(x)=\sum_{i=0}^n a_ix^i$.
Em muitos contextos, tais funções são chamadas de polinômios, ou funções polinomiais.
Nesta seção, as chamaremos, para evitar confusão, de função polinomial, e estudaremos sua relação com polinômios.

\begin{definition}
Seja $R$ um anel comutativo.
Uma função polinomial é uma função $f:R\rightarrow R$ da forma $f(s)=\sum_{i=0}^n a_is^i$ $(s \in R)$, com $0\leq i\leq n$ e $a_i\in R$.
\end{definition}

É possível que o leitor tenha indagado, desde o início do capítulo, sobre o motivo de polinômios não terem sido definidos como acima.
O exemplo abaixo responde esta pergunta:

\begin{exemplo}
Considere $R=\mathbb Z^2$, $f, g:\mathbb Z^2\rightarrow \mathbb Z^2$ dadas por $f(x)=x^3+x^2$ e $g(x)=x^3+x$. Então $f(0)=g(0)=0$, e $f(1)=g(1)=0$, logo, $f=g=0$.

Apesar disso, $f, g$ foram expressas com coeficientes distintos. Ou seja, não vale a propriedade de que dois polinômios de mesmo grau são iguais se, e somente se possuem a mesma sequência de coeficientes.
\end{exemplo}

\begin{prop}
Seja $R$ um anel comutativo. O conjunto das funções polinomiais de $R$ forma um subanel do anel produto $R^R$.
Se $R$ é um domínio infinito, então tal anel é isomorfo à $R[x]$.
\end{prop}

\begin{proof}
    Seja $h:R\rightarrow R^R$ a função que associa $a \in R$ ao mapa constante $h(a)\in R^R$ dado por $h(a)(r)=a$ para todo $r \in R$.
    Está claro que $h(1)$ é o mapa constante $1 \in R^R$, que $h(a+b)=h(a)+h(b)$ e $h(ab)=h(a)h(b)$.

    Pela propriedade universal do anel de polinômios, existe um homomorfismo $g:R[x]\rightarrow R^R$ tal que $g|_{R}=h$ e $g(x)=\id_R$.
    
    Assim, sendo $p(x)=\sum_{i=0}^n a_ix^i$, temos que $g(p(x))(s)=\sum_{i=0}^n a_is^i$.

    Está claro que $g$ é sobrejetora no conjunto das funções polinomiais, e, portanto, este forma um anel. Além disso, se $R$ é um domínio infinito e $g(p(x))=0$, então $p(s)=0$ para todo $s \in R$.
    Se $p(x)\neq 0$, o número de raízes de $p(x)$ seria finito. Como $R$ é infinito, $p(x)$ tem infinitas raízes, assim, $p(x)=0$.
\end{proof}
\section{Mais divisibilidade em anéis de polinômios}
    Algumas vezes pode ser complicado decidir se certo polinômio é irredutível.

    Sabemos que $x^2-2\in \mathbb Q[x]$ é irredutível, pois, estudando-se os graus, caso tivesse um divisor não invertível, deveria ter um divisor de grau $1$ -- e, portanto, deveria ter uma raiz em $\mathbb Q$, o que sabemos não existir.

    Da mesma forma, pode-se argumentar que $x^3-2\in \mathbb Q[x]$ é irredutível. Porém, o mesmo argumento não se aplica a $x^4-2\in \mathbb Q[x]$: sabemos que tal polinômio não possui uma raiz racional, porém, isso não implica que ele não possui, por exemplo, um divisor de grau $2$.
    Conforme veremos adiante, tal polinômio é, de fato, irredutível sobre $\mathbb Q$, e mostraremos isso a partir de outro argumento.

    Note ainda que $x^4+1\in \mathbb R[x]$ não possui raiz real, porém, $x^4+1=(x^2+\sqrt 2 x+1)(x^2-\sqrt 2 x+1)$, logo, $x^4+1$ não é irredutível em $\mathbb R[x]$, embora não tenha raiz.

    Para melhor estudar a reducibilidade em polinômios, definiremos a noção de conteúdo. Tal noção é motivada pelo seguinte lema:

    \begin{lemma}\label{lemma:preConteudo}
    Seja $R$ um domínio de integridade, $p \in R[x]\setminus \{0\}$ e $a \in R\setminus \{0\}$. Escreva $p=\sum_{i=0}^n a_i x^i$. Então $a\mid p$ em $R[x]$ se, e somente se $a\mid a_i$ em $R$ para todo $i\leq n$.
    \end{lemma}
    \begin{proof}
        Suponha que $a\mid p$. Existe $q\in R[x]$ tal que $p=aq$. Como $\gr a=0$, temos que $\gr p=\gr q$. Escreva $q=\sum_{i=0}^n b_i x^i$. Para todo $i\leq n$, $a_i=ab_i$, e, assim, $a|a_i$.
        
        Reciprocamente, suponha que para todo $i\leq n$, $a\mid a_i$. Para cada $i\leq n$, existe $b_i \in R$ tal que $a_i=ab_i$. Logo, $p=a\sum_{i=0}^n b_i x^i$, e, portanto, $a\mid p$.
    \end{proof}

    \begin{definition}[Conteúdo de um polinômio]
        Seja $R$ um domínio de MDC's e $p \in R[x]\setminus \{0\}$. Um \emph{conteúdo} de $p$ é um MDC de seus coeficientes. Explicitamente, se $p=\sum_{i=0}^n a_i x^i$ com $a_n\neq 0$, então um \emph{conteúdo} de $p$ é um elemento de $\MDC(a_0, \dots, a_n)$.

        O conjunto de todos os conteúdos de $p$ é denotado por $C(p)$.
        Explicitamente, $C(p)=\MDC(a_0, \dots, a_n)$.

        Se $1 \in C(p)$, dizemos que $p$ é $\emph{primitivo}$.
    \end{definition}

    Façamos uma pausa para fazer algumas observações.
    
    \begin{itemize}
        \item Se escrevermos $p=\sum_{i=0}^n a_i x^i$ com $n>\gr p$ (de modo que $a_i=0$ para $i\in\{\gr p+1, \dots, n\}$), então $\MDC(a_0, \dots, a_n)=C(p)$, pois para quaisquer $a_0, \dots, a_k$, vale $\MDC(a_0, \dots, a_k, 0)=\MDC(a_0, \dots, a_k)$.
        \item Como $C(p)$ é um conjunto de MDCs, se $a, b \in C(p)$, então $a, b$ são associados (em $R$), e, se $c\in R$ é associado à $a$, então $c\in C(p)$.
        \item Em particular, $p$ é primitivo se, e somente se $C(p)=R^*$.
    \end{itemize}

    Em algum sentido, $C(p)$ é o conjunto dos MDCs de $p$ com relação à $R$, conforme sintetiza o lema a seguir.

    \begin{lemma}
        Seja $R$ um domínio de MDC, $p \in R[x]\setminus\{0\}$ e $a\in R$. São equivalentes:
        \begin{enumerate}[label=(\alph*)]
            \item $a\in C(p)$.
            \item $a\mid p$ em $R[x]$ e para todo $b \in R$, se $b\mid p$ então $b\mid a$.
        \end{enumerate}
    \end{lemma}
    \begin{proof}
        $(a)\rightarrow (b)$: como $a$ divide os coeficientes de $p$, temos que $a\mid p$. Se $b\mid p$, então $b$ divide todos os coeficientes de $p$, e, portanto, $a$, que é o MDC destes.

        $(b)\rightarrow (a)$: Como $a$ divide $p$, $a$ divide os coeficientes de $p$. Além disso, se $b\in R$ divide os coeficientes de $p$, então $b\mid p$, e, por hipótese, $b\mid a$.
    \end{proof}


    \begin{lemma}
        Seja $R$ um domínio de MDC e $p \in R[x]\setminus\{0\}$. Se $a\in R$, então $C(ap)=aC(p)$.
    \end{lemma}
    \begin{proof}
        Seja $p=\sum_{i=0}^n a_i x^i$. Então $ap=\sum_{i=0}^n aa_i x^i$.
        Dessa forma:
        
        \begin{equation*}
            C(ap)=\MDC(aa_0, \dots, aa_n)=a\MDC(a_0, \dots, a_n)=aC(p).
        \end{equation*}
    \end{proof}
    
    \begin{corol}
        Seja $R$ um domínio de MDC e $p \in R[x]\setminus\{0\}$. Se $a \in C(p)$, então $a\mid p$, e, se $p'\in R[x]$ é tal que $p=ap'$, então $p'$ é primitivo.
    \end{corol}

    \begin{proof}
        Como já vimos $a\mid p$. Seja $p'$ tal que $p=ap'$. Como $a \in aC(p')$, existe $b \in C(p')$ tal que $ab=a$. Cancelando $a\neq 0$, temos que $b=1\in C(p')$.
    \end{proof}

    \begin{prop}[Lema de Gauss]
        Seja $R$ um domínio de fatoração única e $p, q \in R[x]\setminus\{0\}$. Então $C(pq)=C(p)C(q)$\footnote{Nesta notação, $C(p)C(q)=\{ab: a \in C(p), b \in C(q)\}$.}.
        Em particular, um produto de polinômios primitivos é primitivo.
    \end{prop}

    \begin{proof}
        Sejam $a \in C(p)$, $b \in C(q)$. Basta ver que $ab \in C(pq)$.

        Escreva $p=\sum_{i=0}^n a_ix^i$, $q=\sum_{i=0}^m b_i x^i$. Então $pq=\sum_{i=0}^{n+m}\sum_{j=0}^{i}a_{i-j}b_jx^i$, onde $a_{i}=0$ se $n<i\leq n+m$ e $b_i=0$ se $m<i\leq n+m$.

        Seja, para $i\leq n+m$, $c_i=\sum_{j=0}^{i}a_{i-j}b_j$. Devemos ver que $ab \in \MDC(c_0, \dots, c_{n+m})$.

        Dados $i, j$ com $0\leq i\leq n+m$ e $0\leq j\leq i$, temos que $a|a_{i-j}$ e $b|b_j$, logo, $ab|a_{i-j}b_j$, logo, $ab\mid \left(\sum_{j=0}^{i}a_{i-j}b_j\right)=c_i$.

        Para cada $i\leq m+n$, existem $a_i'$ e $b_i'$ tais que $a_i=aa_i'$ e $b_i=bb'_i$.
        Então, para cada $i\leq m+n$, $c_i=ab\sum_{i=0}^{n+m}a_{i-j}'b_j'$.
        Seja $c_i'=\sum_{i=0}^{n+m}a_{i-j}'b_j'$.

        Segue que $\MDC(c_0, \dots, c_{n+m})=\MDC(abc'_0, \dots,abc_{n+m}')=ab\MDC(c'_0, \dots, c_{n+m}')$. Assim, basta ver que $1 \in \MDC(c'_0, \dots, c_{n+m}')$. Para tanto, basta ver que se $\alpha \in R$ é primo, então $\alpha$ não é divisor comum de $(c'_0, \dots, c'_{n+m})$.
        Fixe $\alpha$.

        Do lema anterior, como $a\sum_{i=0}^n a_i'x^i=p$, segue que $1\in \MDC(a_0, \dots, a_n)$.
        Analogamente, $1 \in \MDC(b_0, \dots, b_m)$.
        Seja $r\leq n$ o menor número que $\alpha\nmid a_i$ e $s\leq m$ o menor número que $\alpha \nmid b_j$.

        Temos que se $p\nmid c'_{r+s}$: com efeito, $c'_{r+s}=\sum(a_{r+s-j}b_j:0\leq j\leq r+s, j\neq s)+a_rb_s$. Temos que $p|\sum(a_{r+s-j}b_j:0\leq j\leq r+s, j\neq s)$, pois divide cada somando, mas $p\nmid a_rb_s$, pois $p$ é primo, deveríamos ter $p|a_r$ ou $p|a_s$. Assim, $p\nmid c'_{r+s}$.
    \end{proof}



    \begin{prop}
        Seja $R$ um DFU, $K$ seu corpo de frações, e $p, q\in R[x]$ com $p$ primitivo. Então $p\mid q$ em $R[x]$ se, e somente se $p\mid q$ em $K[x]$.
    \end{prop}
    \begin{proof}
        É claro que se $p\mid q$ em $R[x]$, então $p\mid q$ em $K[x]$.

        Agora suponha que $p\mid q$ em $K[x]$. Existe $r \in K[x]$ tal que $q=rp$. Existe $b\in R$ tal que $br\in R[x]$. Seja $a \in C(br)$ e $r' \in R[x]$ primitivo tal que $br=ar'$. Então $bq=brp=ar'p$.

        Do Lema de Gauss, fixado $m \in C(q)$, temos $bm\in bC(q)=C(bq)=C(ar'p)=aC(r'p)\ni a$, logo, $a$ é associado à $bm$ em $R$. Assim, existe $u \in R^*$ tal que $a=ubm$, e, portanto, $aq=ubmq=umar'p$. Cancelando $a$, temos que $q=umr'p$, e, portanto, $p|q$ em $R[x]$.
    \end{proof}

    \begin{corol}
        Seja $R$ um DFU, $K$ seu corpo de frações e sejam $p, q \in R[x]$ primitivos.
        Então $p, q$ são associados em $R[x]$ se, e somente se $p, q$ são associados em $K[x]$. 
    \end{corol}

    \begin{prop}
        Seja $R$ um DFU, $K$ seu corpo de frações e seja $p\in R[x]$ primitivo de grau $k\geq 1$.
        Então $p$ é irredutível em $R[x]$ se, e somente se $p$ é irredutível em $K[x]$, e, nesse caso, $p$ é primo em $K[x]$ e $R[x]$.
    \end{prop}
    \begin{proof}
        Primeiro, suponha que $p$ é irredutível em $K[x]$. Como $K[x]$ é um domínio Euclidiano, temos que $p$ é primo em $K[x]$. Afirmamos que $p$ é primo em $R[x]$: se $p\mid ab$, com $a, b \in R[x]$, temos que $p\mid a$ ou $p\mid b$ em $K[x]$, e, portanto, em $R[x]$.

        Reciprocamente, suponha que $p$ é redutível em $K[x]$ e escreva $p=qt$, com $q, t \in K[x]$ de modo que $n=\gr q$ e $m=\gr t$ são ambos $\geq 1$.

        Existem $b, d\in R\setminus \{0\}$ com $bq, dt \in R[x]$. Tome $a \in C(bq)$ e $c \in C(dt)$, e $q', t'$ primitivos com $bq=aq'$ e $dt=ct'$.
        Note que $\gr q'=\gr q=n$ e $\gr t'=\gr t=m$.

        Segue que $bdp=bdqt=acq't'$. Pelo Lema de Gauss, $bd$ e $ac$ são associados, e, portanto, existe $u \in R^*$ tal que $bdu=ac$. Assim, $acp=bdup=acuq't'$. Cancelando $ac$, temos que $p=uq't'$, e, portanto, $uq'$, $t'$ são divisores não invertíveis de  $p$.
    \end{proof}
    \begin{lemma}
    Seja $R$ um DFU. Todo elemento primo de $R$ é um elemento primo de $R[x]$.
    \end{lemma}
    \begin{proof}
        Seja $\alpha \in R$ primo. Então $\alpha$ é não nulo e não invertível em $R[x]$.

        Suponha que $\alpha\mid pq$.
        Se $p=0$ ou $q=0$, como $\alpha\mid 0$, segue a tese.
        
        Então suponha que $p, q \neq 0$. Seja $m\in C(p)$, $m'\in C(q)$. Pelo Lema de Gauss, $mm'\in C(pq)$. Assim, $\alpha\mid mm'$, logo, $\alpha\mid m$ ou $\alpha\mid m'$, logo, $\alpha\mid p$ ou $\alpha\mid q$.

        Portanto, $\alpha$ é primo.
    \end{proof}

    Observação: o lema acima também vale para domínios de integridade. Ver Exercício~\ref{exer:irredutivelPrimo}.
    \begin{prop}
        Seja $R$ um DFU. Então $R[x]$ é um DFU.
    \end{prop}
    \begin{proof}
        Basta ver que todo elemento não nulo, não invertível de $R[x]$ é um produto de fatores primos.

        Dado $p \in R[x]$ não nulo e não invertível de grau $0$ ele é um produto de primos de $R$. Se $\gr p>0$, escreva $p=mp'$, onde $m \in C(p)$ e $p'$ é primitivo. Temos que $m$ é invertível ou um produto de fatores primos de $R$. Logo, basta ver que $p'$ é um produto de fatores primos de $R[x]$.

        Seja $K$ o corpo de frações de $R$.

        Em $K[x]$, escreva $p'=p_0\dots p_k$, com $p_0,\dots, p_k$ primos de $K[x]$. Para cada $i\leq k$, existe $b_i$ tal que $b_i p_i\in R[x]$.
        Fixe $a_i \in C(b_ip_i)$ e $p_i'$ primitivo com $b_ip_i=a_ip_i'$. Temos que $b_0 \dots b_k p'= a_0\dots a_k p_0'\dots p_k'$. Pelo Lema de Gauss, $b_0 \dots b_k$ e $a_0\dots a_k$ são associados. Existe $u \in R^*$  tal que $u b_0 \dots b_k= a_0\dots a_k$. Segue que $p'=up_0'\dots p_k'$, que é um produto de primos de $R[x]$.
    \end{proof}

     \begin{theorem}
        Seja $R$ um DFU e $K$ seu corpo de frações. Seja $p=\sum_{i=0}^n a_ix^i \in R[x]$. Se existe $\alpha \in R$ irredutível tal que $\alpha\nmid a_n$, $\alpha \mid a_i$ para $i<n$ e $\alpha^2\nmid a_0$, então $p$ é irredutível em $K[x]$.
    \end{theorem}

    \begin{proof}
        Se tal $\alpha$ existe, então $\gr p>0$.
        Escreva $p=p'm$, onde $m \in C(p)$ e $p'$ primitivo. Basta ver que $p'$ é irredutível em $R[x]$.
        Escreva $p'=\sum_{i=0}^n a_i' x^i$.
        
        Temos que $a_i=m a_i'$ para todo $i\leq n$. Como $\alpha \nmid m$, temos que  $\alpha\nmid a_n'$, $\alpha\mid a_i'$ para $i<n$ e $\alpha^2 \nmid a_0$.

        Suponha por absurdo que $p'$ é redutível em $R[x]$.
        Então $p'=qt$, onde $\gr q=m>0$ e $\gr r=k>0$  (pois $p'$ é primitivo).

        Escreva $q=\sum_{i=0}^m b_i x^i$ e $t=\sum_{i=0}^k c_i x^i$.

        Temos que $a'_0=b_0c_0$, logo, $\alpha \mid b_0$ ou $\alpha \mid c_0$. Sem perda de generalidade, vamos supor que $\alpha \mid b_0$. Como $\alpha\nmid {a'_0}^2$, temos que $\alpha \nmid c_0$.

        Existe $i\leq n$ tal que $\alpha \nmid b_i$ (caso contrário teríamos que $\alpha$ divide todos os coeficientes de $p'$). Seja $j$ o primeiro inteiro positivo tal que $\alpha\mid b_i$ para $i<j$.

        Temos que $j\leq m<n$, então $\alpha\mid a_j=\sum_{i=0}^j c_{j-i}b_i$. Como $\alpha\mid \sum_{i=0}^{j-1} c_{j-i}b_i$, segue que $\alpha\mid c_0 b_j$.
        
        Porém, $\alpha \nmid b_j$ e $\alpha \nmid c_0$, um absurdo.
    \end{proof}
\section{Exercícios}
\begin{exer}
    Seja $K$ um corpo. Demonstre, conforme o roteiro abaixo, que o ideal $\langle x-a\rangle$ é maximal no domínio $K[x]$.
    \begin{enumerate}[label=\alph*)]
        \item Considere o homomorfismo avaliação de $K[x]$ em $K$ que avalia $p(x)$ em $a$. Mostre que o núcleo desse homomorfismo é $\langle x-a\rangle$.
        \item Demonstre que $K[x]/\langle x-a\rangle$ é isomorfo a $K$.
        \item Conclua que $\langle x-a\rangle$ é maximal.
    \end{enumerate}
\end{exer}
\begin{exer}
    Considere o anel de polinômios $K[x, y]=K[x][y]$ e  $I=\langle y\rangle$.
    \begin{itemize}
        \item Mostre que $I$ não é um ideal maximal exibindo um ideal próprio que o contém.
        \item Prove que $I$ não é maximal estudando o quociente $K[x, y]/I$.
    \end{itemize}
\end{exer}
\begin{exer}
Mostre que $\mathbb R[x]/\langle x^2+1\rangle$ é isomorfo à $\mathbb C$.
\end{exer}
\begin{exer}
    Prove que $\mathbb Z[x]/\langle x^2+1\rangle$ é isomorfo à $\mathbb Z[i]$.
\end{exer}
\begin{exer}\label{exer:polinomio_serieDominio}
    Prove que se $R$ é um domínio de integridade, então $R\llbracket x \rrbracket$ é um domínio de integridade.
\end{exer}
\begin{exer}
Considere $R=\mathbb Z_4^\mathbb N$ com a estrutura de anel produto. Seja $a \in R$ dado por $a(n)=2$ para todo $n \in \mathbb N$. Mostre que o polinômio $ax \in R[x]$ possui infinitas raízes.
\end{exer}

\begin{exer}
    Prove ou dê um contra-exemplo: os únicos polinômios invertíveis de $\mathbb Z_4[x]$ são os polinômios de grau $0$ dados por $1$ e $3$.
\end{exer}

\begin{exer}
    Seja $K$ um corpo. Mostre que se $p(x)\in K[x]$ tem grau $2$ ou $3$, então $p(x)$ é irredutível se, e somente se, não possui raízes em $K$.
\end{exer}

\begin{exer}
    Seja $K$ um corpo. Mostre que $K[x]$ é $K$-espaço vetorial com a soma usual e o produto por escalar dado pela multiplicação usual de polinômios. Mostre que $(1, x, x^2, \dots)$ é uma base de $K[x]$.
\end{exer}

\begin{exer}
    Seja $R$ um anel comutativo. Dado $p=\sum_{i=0}^n a_ix^i\in R[x]$ com $n>0$, define-se a \emph{derivada formal} de $p$ como $p'=\sum_{i=1}^{n} ia_i x^{i-1}$.
    \begin{enumerate}
        \item Mostre que a derivada formal está bem definida.
        \item Mostre que $(p+q)'=p'+q'$ para todos $p, q \in R[x]$.
        \item Mostre que $(cp)'=cp'$ para todo $c \in R$ e $p \in R[x]$.
        \item Mostre que $(pq)'=p'q+pq'$ para todos $p, q \in R[x]$.
    \end{enumerate}
\end{exer}

\begin{exer}
    Seja $R$ um domínio de integridade. Na notação do exercício anterior, mostre que se $p \in R[x]$, $\alpha \in R$ e $p(\alpha)=0$, então $(x-\alpha)^2\mid p$ se, e somente se $p'(\alpha)=0$.
\end{exer}

\begin{exer}
    Seja $K$ um corpo. Mostre que todo $p \in K[x]$ não nulo é primitivo.
\end{exer}

\begin{exer}\label{exer:irredutivelPrimo}
    Seja $R$ um domínio de integridade e $\alpha \in R$.
    \begin{enumerate}
        \item Mostre que se $\alpha$ é irredutível em $R$, então $\alpha$ é irredutível em $R[x]$.
        \item Mostre que se $\alpha$ é primo em $R$, então $\alpha$ é primo em $R[x]$.
    \end{enumerate}
\end{exer}

\begin{exer} Mostre que:
    \begin{enumerate}
        \item $x^2 +x+1$ é irredutível em $\mathbb Z_2[x]$.
        \item $x^2 +1$ é irredutível em $\mathbb Z_3[x]$.
        \item $x^3-9$ é irredutível em $\mathbb Z_{31}[x]$.
        \item $x^3-9$ é redutível em $\mathbb Z_{11}[x]$.
    \end{enumerate}
\end{exer}

\begin{exer} Mostre que:
    \begin{enumerate}
        \item $x^2 +x+1$ é irredutível em $\mathbb Z_2[x]$.
        \item $x^2 +1$ é irredutível em $\mathbb Z_3[x]$.
        \item $x^3-9$ é irredutível em $\mathbb Z_{31}[x]$.
        \item $x^3-9$ é redutível em $\mathbb Z_{11}[x]$.
    \end{enumerate}
\end{exer}

\begin{exer} Verifique se os polinômios abaixo são irredutíveis em $\mathbb Q[x]$:
    \begin{enumerate}
        \item $2x^4 - 8x^2 + 1$.
        \item $x^4 + 3x + 5$.
        \item $3x^4 + 2x^3 + 4x^2 + 5x + 1$.
        \item $10x^{11} + 6x^3 + 6$.
        \item $x^3 - 3n^2x + n^3$, onde $n \in \mathbb Z$.
        \item $2x^4 + 4x^2 - 2$.
        \item $x^3 - 15x^2 + 10x - 84$.
        \item $x^4 + 4x^3 + 6x^2 + 2x + 1$.
    \end{enumerate}
\end{exer}

\begin{exer} Mostre que para todo corpo $F$, o polinômio $f(x) = x4 +x3 +x+1 \in F[x]$ não é irredutível em $F[x]$.
\end{exer}