\chapter{Noções de Grupos}


\section{Definição e Propriedades Básicas}

Grupos são estruturas matemáticas munidas de uma operação binária com algumas propriedades espeiciais.
O principal objetivo deste texto é servir como texto para um estudo introdutório sobre anéis e corpos, que são estruturas matemáticas que possuem duas operações binárias com propriedades especiais.
Conforme veremos no Capítulo 3, todo anel e todo corpo, com uma dessas operações, forma um grupo.
Assim, é útil, para o estudo de anéis e corpos, o conhecimento de noções básicas sobre grupos.

Apesar das noções de anel e de corpo serem, a nível de definição, noções mais complexas que a de grupo, a noção de grupo, em parte por ser menos restritiva, necessita o desenvolvimento de ferramentas específicas para seu estudo completo.
A área do conhecimento matemático resultante do desenvolvimento dessa teoria é extremamente rica, e chamada de \emph{Teoria dos Grupos}.
Nosso objetivo, por outro lado, é focar no estudo inicial das teorias de anéis e corpos, e, portanto, não mergulharemos nesta importante área.

Assim, não é objetivo deste capítulo apresentar uma introdução ao estudo de grupos, mas sim apenas introduzir as noções e resultados básicos próprios de grupos que são estritamente necessários para os resultados envolvendo anéis e corpos descritos no restante do texto.

\begin{definition}
Um grupo é uma quadrupla $(G,\cdot,e)$, tal que $G$ é um conjunto, $\cdot$ é uma operação binária em $G$ e $0 \in G$, e satisfazem:

\begin{itemize}
    \item (\textbf{Propriedade associativa}) $\forall a, b, c \in G$ $(a \cdot b) \cdot c = a \cdot (b \cdot c)$.
    \item (\textbf{Elemento neutro}) $\forall a \in G$  $e \cdot a = a \cdot e = a$.
    \item (\textbf{Elemento inverso}) $\forall a \in G$ $\exists b \in G$ $a \cdot b = b \cdot a = e$.
\end{itemize}
Se, adicionalmente, a seguinte propriedade é satisfeita, o grupo é chamado de \emph{comutativo}, ou, mais comunmente, \emph{Abeliano}:
\begin{itemize}
    \item (\textbf{Comutatividade}) $\forall a, b \in G\, a \cdot b = b \cdot a$.
\end{itemize}
\end{definition}

Alguns exemplos:

\begin{exemplo}Abaixo, exemplificamos alguns grupos importantes.
    \begin{enumerate}[label=\alph*)]
        \item Com a soma usual e $0$, $\mathbb{Z, Q, R, C}$ são grupos Abelianos.
        \item Com a multiplicação usual, o círculo unitário complexo $\mathbb T=\{x \in \mathbb C: |x|=1\}$ é um grupo Abeliano com elemento neutro $1$.
        De fato, o produto de complexos é comutativo, associativo e tem $1$ como elemento neutro.
        Note que $1\in \mathbb T$ e $0\notin \mathbb T$.
        Se $x \in \mathbb T$, o inverso multiplicativo de $x$ é dado por $\frac{\bar x}{|x|^2}$, onde $\bar x$ denota o conjugado de $x$.
        Como $|\bar x|=|x|=1$, segue que $\mathbb T$ tem todos os inversos de todos seus elementos.
        \item Os inteiros módulo $n$ ($n\geq 1$), dados por $\mathbb Z_n=\{0, \dots, n-1\}$ com a soma dada pela aritmética módulo $n$, são grupos.
        \item Se $X$ é um conjunto qualquer, o conjunto das bijeções de $X$ em $X$ é, com a composição usual de funções e a identidade, um grupo, cuja operação inversa é a inversão usual de funções.
        Tal grupo é denominado \emph{grupo de permutações de $X$}.
        
        Caso $X$ tenha ao menos $3$ elementos, ele não é abeliano: sendo $a, b, c$ três elementos distintos de $X$, sendo $f$ a função que permuta $a, b$ e fixa os demais elementos, $g$ a que permuta $b, c$ temos que $f\circ g(a)=f(a)=b$, mas $g\circ f(a)=g(b)=c$, logo, $f\circ g\neq g\circ f$.
    \end{enumerate}
\end{exemplo}

Algumas observações importantes sobre a notação utilizada no estudo de grupos:
\begin{itemize}
\item Ao discursar sobre grupos, é comum omitir a operação e o elemento neutro, referindo-se apenas ao conjunto $G$, conforme fizemos acima ao mencionar que $\mathbb Z$ é um grupo.
O mais formal, porém muito menos usual, feito principalmente em situações em há chance de confusão, é escrever que, por exemplo, $(\mathbb Z, +, 0)$ é um grupo.
\item Como também ocorre com $\mathbb Z$, caso o grupo seja Abeliano, é comum que sua operação binária seja denotada por $+$ ou outro símbolo similar.
Nesse contexto, o elemento neutro é frequentemente denotado por $0$.
\item Caso o grupo em discurso não seja necessariamente Abeliano, é comum que sua operação binária seja denotada por $\cdot$ ou outro símbolo similar.
Nesse contexto, o elemento neutro é frequentemente denotado por $e$, e a operação é frequentemente omitida, ou seja, $a \cdot b$ é frequentemente escrito como $ab$.

Porém, há grupos Abelianos cujas operações também são denotadas por $\cdot$, como no caso o grupo $\mathbb T$ mencionado acima.
\end{itemize}


Agora iniciaremos a provar algumas propriedades básicas sobre grupos.
\begin{prop}[Unicidade do elemento neutro]\label{prop:group_uniqueNeutral}
    Seja $(G,\cdot,e)$ um grupo.
    Então, o elemento neutro $e$ é único.
    Isto é, se $h \in G$ é tal que $\forall a \in G$ $h \cdot a = a \cdot h = a$, então $h = e$.
\end{prop}
\begin{proof}
    Note que $h=he$, pois $e$ é elemento neutro.
    Por outro lado, $e=he$, pois $h$ é elemento neutro.
    Assim, $h=he=e$.
\end{proof}

\begin{prop}[Unicidade dos inversos]\label{prop:group_uniqueInverse}
    Seja $(G,\cdot,e)$ um grupo.
    Então todo $a \in G$ possui um único elemento inverso.
Isto é $\forall a \in G$ $\exists!\, b \in G$ $a \cdot b = b \cdot a = e$.
\end{prop}
\begin{proof}
    A existência do inverso é garantida pela definição de grupo.
    
    Para provar a unicidade, suponha que $b, c$ são inversos de $a$, ou seja, que $a \cdot b = b \cdot a = e$ e $a \cdot c = c \cdot a = e$.
    Segue que:
    $$b=be=b(ac)=(ba)c=ec=c.$$
\end{proof}

A unicidade dos inversos nos permite definir a notação $a^{-1}$ para o inverso de $a$ em um grupo $(G,\cdot,e)$.
Caso $(G, +, 0)$ seja um grupo Abeliano, a notação $-a$ é frequentemente utilizada para denotar o inverso de $a$, e, nesse caso, $-a$ é chamado de \emph{oposto} de $a$.

Note que assim, ficam definidos operadores unários $(\,)^{-1}:G\rightarrow G$ (ou $-:G\rightarrow G$).
Para o segundo caso, define-se também que $a-b=a+(-b)$.

\begin{prop}[Cancelamento]\label{prop:group_cancel}
    Seja $(G,\cdot,e)$ um grupo e $a,b,c \in G$.
    Se $a \cdot b = a \cdot c$, então $b=c$.
    Analogamente, se $b \cdot a = c \cdot a$, então $b=c$.
\end{prop}
\begin{proof}
    Provaremos a primeira afirmação.
    A segunda é análoga e fica como exercício.
    Suponha que $ba=ca$.
    Segue que $(ba)a^{-1}=(ca)a^{-1}$.
    
    Pela propriedade associativa, $b(aa^{-1})=c(aa^{-1})$.

    Pela definição de inverso, segue que $be=ce$.

    Pela neutralidade de $e$, segue que $b=c$.
\end{proof}

\begin{corol}[Cancelamento II]\label{prop:group_cancelII}
    Seja $(G,\cdot,e)$ um grupo.
    Para todos $a, b \in G$, se $ab=a$, então $b=e$.
Analogamente, se $ba=a$, então $b=e$.
\end{corol}
\begin{proof}
    Para a primeira afirmação, note que $ab=ae$, logo, pela proposição anterior, $b=e$.
    Para a segunda afirmação, note que $ba=ea$, logo, pela proposição anterior, $b=e$.
\end{proof}

\begin{prop}[Regras de sinal]\label{prop:regraSinal}
    Seja $G$ um grupo e $a, b \in G$.
    Então:
    \begin{enumerate}[label=\alph*)]
        \item $((a)^{-1})^{-1}=a$ [na notação aditiva, $-(-a)=a$].
\label{prop:regraSinal_A}
        \item $(ab)^{-1}=b^{-1}a^{-1}$ [na notação aditiva, $-(a+b)=(-b)+(-a)]$.\label{prop:regraSinal_B}
        \item $e^{-1}=e$ [na notação aditiva, $-0=0$].\label{prop:regraSinal_C}
    \end{enumerate}
\end{prop}
\begin{proof}
    \ref{prop:regraSinal_A}: Temos que $(a^{-1})^{-1}a^{-1}=e=aa^{-1}$.
    Cancelando $a^{-1}$, segue.
    
    \ref{prop:regraSinal_B}: Temos que $(ab)^{-1}(ab)=e=(b^{-1}a^{-1})ab$.
    Cancelando $ab$, segue que $(ab)^{-1}=b^{-1}a^{-1}$.
    Analogamente, $(ba)^{-1}=a^{-1}b^{-1}$.

    \ref{prop:regraSinal_C}: Temos que $(e^{-1})e=e=ee$.
    Cancelando $e$ à direita, segue.


\end{proof}

\section{Somatórios}

Nessa seção, formalizaremos a noção de somatório.
É desejável que o leitor já possua familiaridade com alguma notação de somatório, não sendo nosso objetivo fornecer ao leitor um primeiro contato. Aqui apresentaremos a notação e as técnicas de ``substituição de variáveis'' que serão utilizadas.

\begin{definition}[Soma de família finita]\label{def:group_sum}
Seja $G$ um conjunto munido de uma operação $+$ associativa, comutativa e com neutro $0$.
Define-se, recursivamente para $n\geq 0$, o somatório de famílias $(a_i: i \in F)$, onde $F$ é um conjunto de $n$ índices e $a_i \in G$ para todo $i \in F$, como se segue:

\begin{itemize}
    \item \textbf{Notação:} se $a=(a_i)_{i\in F}$ é uma sequência de elementos de $G$, então usamos as notações:
    \[\sum a=\sum(a_i: i\in F)=\sum_{i\in F} a_i.\]
    \item Caso base $n=0$ (soma vazia): só existe uma família com $0$ elementos, que é a família vazia $a=()=\emptyset=(a_i:i\in \emptyset)$.
    Definimos: \[\sum a=\sum_{i \in \emptyset}a_i=0\].
    \item Passo recursivo $n\rightarrow n+1$: considere uma família $(a_i)_{i\in F}$, onde $|F|=n+1$.
    Define-se:
    \[\sum(a_i: i \in F)=\sum(a_i: i \in F\setminus\{j\})+a_j,\]
    onde $j \in I$ é qualquer elemento.
\end{itemize}
\end{definition}
É claro que, para mostrar que a definição acima é consistente, precisamos mostrar que a soma não depende da escolha de $j$.

\begin{lemma}
Qualquer que seja o tamanho (finito) de $F$, $\sum(a_i)_{i\in F}$ está bem definido.
\end{lemma}

\begin{proof}
    Seja $F$ um conjunto finito.
Se $|F|=0$, então $F=\emptyset$, e a soma é $0$.
Se $|F|=1$, então $F=\{j\}$ -- só há uma escolha para $j$, e a soma é $a_j$.
    Se $|F|=n+1$ para $n\geq 1$, tome $j, k \in F$.
    Devemos ver que $\left(\sum_{i\in F\setminus\{j\}} a_i\right)+a_j=\left(\sum_{i\in F\setminus\{k\}} a_i\right)+a_k$.
    Com efeito:

    \[\left(\sum_{i\in F\setminus\{j\}} a_i\right)+a_j=\left(\left(\sum_{i\in F\setminus\{j, k\}} a_i\right)+a_k\right)+a_j=\left(\sum_{i\in F\setminus\{j, k\}} a_i\right)+(a_k+a_j)\]

    \[=\left(\sum_{i\in F\setminus\{j, k\}} a_i\right)+(a_j+a_k)=\left(\left(\sum_{i\in F\setminus\{j, k\}} a_i\right)+a_j\right)+a_k=\left(\sum_{i\in F\setminus\{k\}} a_i\right)+a_k.\]
\end{proof}
Como fazemos no cálculo de integrais, muitas vezes é desejável utilizar técnicas de substituição de variáveis para calcular ou simplificar somatórios.
A proposição abaixo formaliza esta técnica.
\begin{prop}[Mudança de variável em somatório]\label{prop:group_sumVarChange}
    Seja $G$ um conjunto munido de uma operação $+$ associativa, comutativa e com neutro $0$.
Seja $(a_i: i \in I)$ uma família finita em $G$ e $\phi:J\rightarrow I$ uma função bijetora.
Então:

    \[\sum_{i \in I}a_i=\sum_{j \in J}a_{\phi(j)}.\]

\end{prop}
\begin{proof}
Novamente, procedemos por indução no tamanho de $n=|I|$.
A base de tamanho $0$ é trivial, já que ambos os lados da igualdade são $0$.

Para o passo indutivo em que $|I|=|J|=n+1$, considere $\phi:J\rightarrow I$ como no enunciado.
Fixe $k \in J$ qualquer e sejam $I'=I\setminus\{\phi(k)\}, J'=J\setminus\{k\}$ e $\phi'=\phi|_{J'}:J'\rightarrow I'$, que é bijetora.
Como $|J'|=|I'|=n$, por hipótese indutiva temos que $\sum_{j \in J'}a_{\phi(j)}=\sum_{i \in I'}a_i$.
Segue que:

\[\sum_{j \in J}a_{\phi(j)}=\left(\sum_{j \in J'}a_{\phi(j)}\right)+a_{\phi(k)}=\left(\sum_{i \in I'}a_{i}\right)+a_{\phi(k)}=\sum_{j \in I}a_{i}.\]
\end{proof}

Também podemos juntar, sempre que necessário, duas somas disjuntas sob um único sinal de somatório.
\begin{prop}[Concatenação de somatórios]\label{prop:group_sumConcat}
    Seja $I, J$ conjuntos disjuntos. Considere em $G$ famílias $(a_i: i \in I)$ e $(a_i: i \in J)$, e a família $(a_i: i \in I\cup J)$.
    Vale a relação:

    \[\sum_{i \in I}a_i+\sum_{i \in I}a_i=\sum_{i \in I\cup J}a_i\]
\end{prop}
\begin{proof}
    Provaremos por indução no tamanho de $J$.
    Se $J=\emptyset$, temos $I\cup J=J$ e $\sum_{i \in J}a_i=0$, logo, segue a tese.

    Se a proposição vale para todo $J$ de tamanho $n$, suponha que $|J|=n+1$ e seja $J'=J\setminus \{j'\}$, onde $j' \in J$ é qualquer elemento arbitrário.

    Então:
    \[\sum_{i \in I}a_i+\sum_{i \in J}a_i=\left(\sum_{i \in I}a_i+\sum_{i \in J'}a_i\right)+a_{j'}=\sum_{i \in I\cup J'}a_i+a_{j'}=\sum_{i \in I\cup J}a_i.\]
\end{proof}

Antes de enunciar a próxima proposição, precisamos falar sobre notações de índices duplos.
Se $K$ é um conjunto que consiste apenas de pares ordenados $(i, j)$, não é incomum encontrarmos na literatura a notação $\sum_{(i, j)\in K}a_{ij}$.
bem como encontrar a notação $(a_{ij}:(i, j)\in K))$.

Tais notações são usadas no contexto no qual uma para todo $i, j$ com $(i, j)\in K$, está definido um elemento $a_{ij}$.
Para se encaixar no nosso formalismo, $(a_{ij}:i, j \in K)$ denota $(a_{\pi_1(k)\pi_2(k)}: k \in K)$, e $\sum_{(i, j)\in K}a_{ij}$ denota $\sum_{k \in K}a_{\pi_1(k)\pi_2(k)}$, onde $\pi_1$ e $\pi_2$ são as funções coordenadas de pares ordenados.

\begin{prop}[Concatenação de somatórios II]\label{prop:group_sumConcatII}
    Seja $I$ um conjunto não vazio e, para cada $i \in F$, seja $F_i$ um conjunto.
    Considere o conjunto finito $K=\bigcup_{i \in I}\{i\}\times F_i=\{(i, j): i \in I, j \in F_i\}$.

    Para cada $i \in I$, considere uma família de elementos de $G$, $(a_{ij}: j \in F_i)$.
    Considere também as famílias $(a_{ij}: (i, j)\in K)=(a_{ij}: i\in I, j \in F_i)$ e $(\sum_{j \in F_i}: i\in I)$. Vale a relação:

    \[\sum_{i \in I}\left(\sum_{j \in F_i}a_{ij}\right)=\sum_{(i, j)\in K}a_{ij}\]
\end{prop}
\begin{proof}
    Provaremos por indução no tamanho de $I$.
    Se $I=\{i'\}$ é unitário, temos que $K=\{(i', j): j \in F_{i'}\}$.
    Segue que $\phi:F_{i'}\rightarrow K$ dada por $\phi(j)=(i', j)$ é bijetora.
    Assim, pela proposição \ref{prop:group_sumVarChange}, temos que:

    \[\sum_{i \in I}\left(\sum_{j \in F_i}a_{ij}\right)=\left(\sum_{j \in F_{i'}}a_{i'j}\right)=\sum_{(i, j)\in K}a_{ij}\]

    Justificando melhor a última igualdade:
    \[\sum_{j \in F_{i'}}a_{i'j}=\sum_{j \in F_{i'}}a_{\pi_1(\phi(j))\pi_2(\phi(j))}=\sum_{k \in K}a_{\pi_1(k)\pi_2(k)}=\sum_{(i, j)\in J}a_{ij}.\]
    
    Agora provaremos o passo indutivo.

    Suponha que $|I|=n+1$ e que a hipótese vale para todo $I$ de tamanho $n$. Fixe qualquer $i' \in I$ e considere $I'=I\setminus \{i'\}$.

    Seja $K'=\{(i', j): j \in F_{i'}\}$. e $\hat K=\{(i, j): i \in I', j \in F_i\}$.
    Segue que $K$ é a união disjunta de $\hat K$ e $K'$.
    Assim, pela Proposição \ref{prop:group_sumConcat}, pelo caso base, e pelo passo indutivo, segue que:
    \[\sum_{i \in I}\left(\sum_{j \in F_i}a_{ij}\right)=\left(\sum_{i \in I'}\sum_{j \in F_{i}}a_{ij}\right)+\left(\sum_{j \in F_{i'}}a_{i'j}\right)=\sum_{(i, j)\in \hat K}a_{ij}+\sum_{(i, j)\in K'}a_{ij}=\sum_{(i, j)\in K}a_{ij}\]
\end{proof}

Agora provaremos a comutação de somatórios.

\begin{prop}\label{prop:group_sumCommut}
    Sejam $I, J$ conjuntos não vazios e considere uma família $(a_{ij}:(i, j)\in I\times J)=(a_{ij}:i \in I, j \in J)$.

    Para cada $i \in I$, considere a família $(a_{ij}: j \in J)$, e, para cada $j \in J$, considere a família $(a_{ij}: i \in I)$.

    Então:

    \[\sum_{i\in I}\sum_{j \in J}a_{ij}=\sum_{j \in J}\sum_{i \in I}a_{ij}\]
\end{prop}
\begin{proof}
    Seja $K=I\times J$ e $K'=J\times I$.
    Temos que $\phi:I\times J\rightarrow J\times I$ dada por $\phi(i, j)=(j, i)$ é uma bijeção entre $K$ e $K'$. Escrevendo de outra forma, $\phi(k)=(\pi_2(k), \pi_1(k))$. Note ainda que $\phi^2=\id_K$.

    Assim, pela Proposição \ref{prop:group_sumVarChange} e Proposição \ref{prop:group_sumConcatII}, temos que:

    \[\sum_{j\in J}\sum_{j \in J}a_{ij}=\sum_{(i,j)\in I\times J}a_{ij}=\sum_{k\in K}a_{\pi_1(\phi^2(k))\pi_2(\phi^2(k))}\]
    \[=\sum_{k \in K'}a_{\pi_1(\phi(k))\pi_2(\phi(k))}=\sum_{k \in K'}a_{\pi_2(k)\pi_1(k)}=\sum_{(j, i)\in J\times I}a_{ij}=\sum_{j \in J}\sum_{i \in I}a_{ij}.\]

\end{proof}
\section{Exercícios}
\begin{exer}
    Suponha que $a$, $b$ e $c$ sejam elementos de um anel $A$, e que $a$ não é divisor de $0$.
    
    Mostre que se $ab = ac$, então ou $a = 0$ ou $b = c$ (isto é, se $a\neq 0$, podemos cancela-lo).
\end{exer}