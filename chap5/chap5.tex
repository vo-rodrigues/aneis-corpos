\chapter{Ideais e Quocientes}
\section{Ideais}
\begin{definition}[Ideal à esquerda]
    Seja $A$ um anel. Um subconjunto $I \subseteq A$ é dito \emph{ideal à esquerda} se:
    \begin{itemize}
        \item $0 \in I$.
        \item Para todos $a, b \in I$, temos $a+b\in I$.
        \item $\forall a \in A$ e $\forall b \in I$, temos $ab \in I$.
    \end{itemize}
\end{definition}

\begin{definition}[Ideal à direita]
    Seja $A$ um anel. Um subconjunto $I \subseteq A$ é dito \emph{ideal à direita} se:
    \begin{itemize}
        \item $0 \in I$.
        \item Para todos $a, b \in I$, temos $a+b\in I$.
        \item $\forall a \in I$ e $\forall b \in A$, temos $ab \in I$.
    \end{itemize}
\end{definition}

\begin{definition}[Ideal]
    Seja $A$ um anel. Um subconjunto $I \subseteq A$ é dito \emph{ideal} se for um ideal à esquerda e um ideal à direita. Ou seja, $I$ é um ideal se:
    \begin{itemize}
        \item $0 \in I$.
        \item Para todos $a, b \in I$, temos $a+b\in I$.
        \item $\forall a \in A$ e $\forall b \in I$, temos $ab \in I$.
        \item $\forall a \in I$ e $\forall b \in A$, temos $ab \in I$.
    \end{itemize}
\end{definition}

\begin{prop}[Ideal trivial]Seja $A$ um anel. Então $\{0\}$ e $A$ são ideais de $A$. Estes ideais são chamados de \emph{ideais principais}
\end{prop}
\begin{proof}
    Exercício.
\end{proof}

Note que se $A$ é um anel comutativo, então $I$ é um ideal à esquerda se, e somente se, $I$ é um ideal à direita. Assim, em anéis comutativos, a noção de ideal é equivalente à de ideal à esquerda ou à de ideal à direita.

\begin{prop}[Interseção de ideais]
    Seja $A$ um anel e $\mathcal F$ uma coleção não vazia de ideais à esquerda de $A$. Então $\bigcap_{I \in \mathcal F}I=\bigcap \mathcal F$ é um ideal de $A$. O mesmo vale para ideais à direita e ideais.
\end{prop}
\begin{proof}
    Provaremos para ideais à esquerda. A prova para ideais à direita é análoga e fica como exercício.

    Seja $I=\bigcap \mathcal F$. Então $0 \in I$, pois $0 \in I$ para todo $I \in \mathcal F$.

    Sejam $a, b \in I$. Então, para todo $I \in \mathcal F$, temos que $a, b \in I$, logo, $a+b\in I$. Assim, $a+b\in \bigcap \mathcal F$.

    Finalmente, seja $a \in A$ e $b \in I$. Então, para todo $I \in \mathcal F$, temos que $b \in I$, logo, $ab\in I$. Assim, $ab\in \bigcap \mathcal F$.
\end{proof}

\begin{prop}[Ideal gerado]
    Seja $A$ um anel comutativo e $B\subseteq A$ um conjunto não vazio. Então, o conjunto $I=\{a_1b_1+\cdots+a_nb_n: n\geq 1, a_i \in A, b_i \in B\}$ é o menor ideal à esquerda $A$ que contém $B$ (ou seja, além de ser um ideal contendo $B$, se $J$ é qualquer ideal contendo $B$, então $I\subseteq J$). O ideal $I$ é chamado de \emph{ideal gerado por $B$}, e denotado por $\langle B \rangle$.
    
    Se $B=\{x_0, \dots, x_n\}$, então abreviamos $\langle B \rangle$ como $\langle x_0, \dots, x_n \rangle$.
\end{prop}
\begin{proof}
    Primeiro, verificaremos que $I$ é um ideal.

    $0 \in I$, pois $0=0b$ para todo $b \in B$.

    Sejam $x, y \in I$. Então existem $n, m\geq 1$, $a_1, \dots, a_n \in A$, $b_1, \dots, b_n \in B$, $c_1, \dots, c_m \in A$ e $d_1, \dots, d_m \in B$ tais que $x=a_1b_1+\cdots+a_nb_n$ e $y=c_1d_1+\cdots+c_md_m$. Assim, $x+y=(a_1b_1+\cdots+a_nb_n)+(c_1d_1+\cdots+c_md_m)=(a_1b_1+\cdots+a_nb_n)+(c_1d_1+\cdots+c_md_m) \in I$.

    Finalmente, seja $a \in A$ e $b \in I$. Então existem $n\geq 1$, $a_1, \dots, a_n \in A$ e $b_1, \dots, b_n \in B$ tais que $b=a_1b_1+\cdots+a_nb_n$. Assim, $ab=(a_1b_1+\cdots+a_nb_n)a=a_1(b_1a)+\cdots+a_n(b_na) \in I$.

    Agora, seja $J$ um ideal de $A$ que contém $B$. Então, como $J$ é um ideal de $A$, temos que $\forall a_i\in A$, $\forall b_i\in B$, temos que $(a_i b_i)\in J$. Logo, $I\subseteq J$. Portanto, $I$ é o menor ideal de $A$ que contém $B$.
\end{proof}

Observação: note que o menor ideal contendo $B=\emptyset$ é o ideal nulo, $\{0\}$.

\begin{definition}[Ideal principal]
    Seja $A$ um anel. Para todo $x \in A$, o conjunto $xA=\{xa:a \in A\}$ é um ideal à direita de $A$. O ideal $xA$ é chamado de \emph{ideal principal à direita gerado por $x$}.
    Analogamente, o conjunto $Ax=\{ax:a \in A\}$ é um ideal à esquerda de $A$, e é chamado de \emph{ideal principal à esquerda gerado por $x$}.
\end{definition}
\begin{proof}
Mostraremos que $xA$ é um ideal à direita. As demais afirmações ficam como exercício.

Note que $0 \in xA$, pois $x0=0$.

Sejam $a, b \in xA$. Então, existem $a_1, a_2 \in A$ tais que $a=xa_1$ e $b=xa_2$. Assim, $a+b=xa_1+xa_2=x(a_1+a_2) \in xA$.

Finalmente, seja $a \in A$ e $b \in xA$. Então, existe $b_1 \in A$ tal que $b=xb_1$. Assim, $ab=(xa)b_1=x(ab_1) \in xA$.
\end{proof}
\begin{definition}[Ideal principal]
    Seja $A$ um anel. Para todo $x \in A$, o conjunto $xA=\{xa:a \in A\}$ é um ideal à esquerda de $A$. O ideal $xA$ é chamado de \emph{ideal principal à esquerda gerado por $x$}.
    Analogamente, o conjunto $Ax=\{ax:a \in A\}$ é um ideal à direita de $A$, e é chamado de \emph{ideal principal à direita gerado por $x$}.
    Se $A$ é comutativo, o ideal $xA=Ax$ é chamado de \emph{ideal principal gerado por $x$}.
\end{definition}

Observação: note que, comparando as definições, se $A$ é um anel comutativo com unidade, $xA=\langle x\rangle$.

Notemos que ideais triviais são principais à esquerda e à direita, pois $0A=\{0\}=A0$ e $A1=A=1A$.

\begin{definition}[Domínio de ideais principais]
    Um domínio de ideais principais (DIP), ou anel principal, é um domínio de integridade $A$ tal que todo ideal de $A$ é principal.
\end{definition}

\begin{prop}[Ideais de um corpo são triviais]
    Todo ideal de um corpo é trivial. Em particular, todo corpo é um DIP. Reciprocamente, se $A$ é um anel comutativo não trivial cujo todo ideal é trivial, então $A$ é um corpo.
\end{prop}
\begin{proof}
Seja $K$ um corpo e $I$ um ideal de $K$. Se $I=\{0\}$, então $I$ é trivial. Se $I\neq \{0\}$, então existe $a \in I$ tal que $a \neq 0$. Daí $1=a^{-1}a=\in I$. Logo, para todo $k \in K$, $k=1k\in I$.

Para a recíproca, seja $A$ um anel comutativo não trivial tal que todo ideal de $A$ é trivial, e fixe $x \in A\setminus \{0\}$. Como $Ax$ é um ideal trivial e $0\neq x \in Ax$, temos que $Ax=A$. Logo, existe $a \in A$ tal que $ax=1$. Assim, $x$ é invertível. Portanto, $A$ é um corpo.
\end{proof}

\begin{prop}[Um DIP que não é um corpo] O anel dos inteiros $\mathbb Z$ é um domínio de ideais principais que não é um corpo.
\end{prop}
\begin{proof}
    Seja $I$ um ideal de $\mathbb Z$.
    Veremos que $I$ é um ideal principal.
    Se $I=\{0\}$, então $I$ é principal.
    Caso contrário, $I$ contém ao menos um elemento positivo, já que, sendo $x\in I\setminus\{0\}$, temos que $-x \in I$ e um dos $x, -x$ é positivo.

    Seja $n$ o menor inteiro positivo de $I$.
    Afirmamos que $I=n\mathbb Z$.
    De fato, se $x \in I$, então escreva $x=qn+r$, onde $q,r \in \mathbb Z$ e $0\leq r<n$.
    Como $x \in I$, temos que $r=x-qn \in I$. Assim, $r=0$, ou violaríamos a minimalidade de $n$.
    Logo, $x=qn\in n\mathbb Z$.
    Portanto, $I\subseteq n\mathbb Z$.
    Como $n\mathbb Z=\langle n\rangle$ e $n \in I$, temos que $n\mathbb Z\subseteq I$, o que completa a prova.
\end{proof}

\section{Quocientes}
\begin{definition}
    Seja $A$ um anel. Uma relação de congruência em $A$ é uma relação de equivalência $\sim$ em $A$ que ``preserva operações''. Explcitamente, tal que para todos $a, b, c, d \in A$, se Se $a\sim b$ e $c\sim d$, então $a+c\sim b+d$ e $ac\sim bd$.
\end{definition}

Quais são todas as relações de congruência em $A$? A proposição abaixo classifica-as a partir dos ideais de $A$.
\begin{prop}[Relações de congruência vs ideais]
    Seja $A$ um anel, $\mathcal R(A)$ o conjunto de todas as relações de congruência em $A$ e $\mathcal I(A)$ o conjunto de todos os ideais de $A$. Então, existe uma bijeção entre $\mathcal R(A)$ e $\mathcal I(A)$ dada por
    $\sim \mapsto I_{\sim}=\{a \in A: a\sim 0\}$,
    cuja inversa se dá por $I\mapsto \sim_I=\{(a, b) \in A^2: a-b \in I\}$.
\end{prop}
\begin{proof}
Primeiro, vejamos que se $\sim$ é uma relação de congruência, então $I_\sim$ é um ideal de $A$.

\begin{itemize}
\item $0 \in I_\sim$, pois $0\sim 0$.
\item Se $a, b \in I_\sim$, então $a\sim 0$ e $b\sim 0$, logo $a+b\sim 0+0=0$, portanto, $a+b \in I_\sim$.
\item Se $x \in A$ e $a \in I_\sim$, então $a\sim 0$ e $x\sim 0$, logo $ax\sim a0=0$ e $xa=0a=0$, portanto, $ax, xa \in I_\sim$.
\end{itemize}

Agora, vejamos que se $I$ é um ideal, então $\sim_I$ é uma relação de congruência. De fato, temos que, para todos $a, b, c, d \in A$:
\begin{itemize}
    \item $a\sim_I a$ pois $a-a=0\in I$.
    \item Se $a\sim_I b$, então $a-b \in I$, logo $(-1)(a-b)=b-a\in I$, e, portanto, $b\sim_I a$.
    \item Se $a\sim_I b$ e $b\sim_I c$, então $a-b \in I$ e $b-c \in I$, logo, $(a-b)+(b-c)=a-c \in I$, portanto, $a\sim_I c$.
    \item Se $a\sim_I b$ e $c\sim_I d$, então $a-b \in I$ e $c-d \in I$, logo, $(a-b)+(c-d)=(a+c)-(b+d)\in I$, portanto, $a+c\sim_I b+d$.
    \item Se $a\sim_I b$ e $c\sim_I d$, então $a-b \in I$ e $c-d \in I$, logo, $(a-b)c=ac-bc\in I$ e $b(c-d)=bc-bd\in I$, logo $(ac-bc)+(bc-bd)=ac-bd\in I$, portanto, $ac\sim_I bd$.
    \end{itemize}

Se $I$ é ideal, $I_{\sim_I}=I$, pois, para todo $a\in A$:

$$a\in I_{\sim_I}\Leftrightarrow a\sim_I 0\Leftrightarrow a-0\in I\Leftrightarrow a\in I.$$

Finalmente, se $\sim$ é relação de congruência, $\sim_{I_\sim}=\sim$, pois, para todos $a, b \in A$:

$$a\sim_{I_\sim} b\Leftrightarrow a-b\in I_\sim \Leftrightarrow a-b\sim 0\Leftrightarrow a\sim b.$$

Justificando a última equivalência: se $a-b\sim 0$, como $b\sim b$, temos que $a-b+b\sim b$, ou seja, que $a\sim b$. Reciprocamente, se $a\sim b$, como $(-b)\sim (-b)$, segue que $a+(-b)\sim b+(-b)$, ou seja, que $a-b\sim 0$.
\end{proof}

Como feito nos inteiros, podemos, ao invés de trabalhar com relações de congruência, encontrar anéis em que a congruência corresponda exatamente à igualdade.

\begin{definition}
Seja $A$ um anel e $\sim$ uma relação de congruência. Define-se que $A/\sim$ é $A/\sim=\{[a]_\sim: a \in A\}$, onde $[a]_\sim=\{b\in A: b\sim a\}$ é a classe de equivalência de $a$ com relação a $\sim$.

Define-se que $[a]_\sim+[b]_\sim=[a+b]_\sim$ e que $[a]_\sim[b]_\sim=[ab]\sim$.

Se $I$ é um ideal, $A/I=A/\sim_I$, e o mapa quociente de $A$ em $A/I$ se dá por $q:A\longrightarrow A/I$ dada por $q(a)=[a]_{\sim_I}$.
\end{definition}

Pelas propriedades das relações de congruência, a soma e produto de $A/\sim$ (ou $A/I$) estão bem definidas. Além disso:

\begin{lemma}[Propriedades do quociente]
    Na notação acima:
    \begin{enumerate}[label=\alph*)]
        \item $q$ é epimorfismo de anéis. \label{lemma:propriedadesQuociente_a}
        \item $\ker q = I$. \label{lemma:propriedadesQuociente_b}
        \item $q(a)=a+I=\{a+x: x \in I\}$ para todo $a \in A$. \label{lemma:propriedadesQuociente_c}
        \item Se $A$ é anel comutativo, $A/I$ também é. \label{lemma:propriedadesQuociente_d}
    \end{enumerate}
\end{lemma}

\begin{proof}
    \ref{lemma:propriedadesQuociente_a} Seja $a, b, c, d \in A$. Temos que $q(a+b)=q(a)+q(b)$ e $q(ab)=q(a)q(b)$ por definição da soma em $A/I$, e $q$ é sobrejetora pela definição de $q$. Finalmente, $q(1_A)$ é identidade pois para todo $a \in A$, $q(1_A)q(a)=q(1_Aa)=q(a)$ e $q(a)q(1_A)=q(a1_A)=q(a)$, logo, $q(1_A)=1_{A/I}$.

    \ref{lemma:propriedadesQuociente_b} Temos que $\ker q=\{a \in A: q(a)=q(0)\}=\{a \in A: a\sim_I 0\}=\{a \in A: a\in I\}=I$.

    \ref{lemma:propriedadesQuociente_c} Temos que $q(a)=[a]_{\sim_I}=\{b \in A: b\sim_I a\}=\{b \in A: b-a\in I\}=\{a+x: x \in I\}$ pois se $b-a \in I$ se, e somente se $a-b=x$ para algum $x \in I$.

    \ref{lemma:propriedadesQuociente_d} Se $A$ é comutativo, então $A/I=\ran q$ também é, pois $q$ é homomorfismo de anéis.
\end{proof}

\section{Teoremas do isomorfismo}
\begin{theorem}[Teorema do homomorfismo]
    Seja $f:A\rightarrow R$ um homomorfismo de anéis e $J$ um ideal tal que $J\subseteq \ker f$. Então, existe um único homomorfismo de anéis $g:A/J\rightarrow R$ tal que $g\circ q=f$, onde $q:A\rightarrow A/J$ é o mapa quociente canônico dado por $q(a)=a+J$.
\end{theorem}
\begin{proof}
    Definimos $g:A/J\rightarrow R$ por $g(a+J)=f(a)$. Então, $g$ é bem definido, pois se $a+J=b+J$, então $a-b \in J\subseteq \ker f$, logo, $f(a-b)=0_R$, ou seja, $f(a)=f(b)$.

    Agora, vejamos que $g$ é um homomorfismo de anéis. De fato, para todo $a', b' \in A/J$, sendo $a'=a+J$ e $b'=b+J$, temos que:
    \begin{itemize}
        \item $g(a'+b')=g((a+J)+(b+J))=g((a+b)+J)=f(a+b)=f(a)+f(b)=g(a+J)+g(b+J)$.
        \item $g(a'b')=g((a+J)(b+J))=g(ab+J)=f(ab)=f(a)f(b)=g(a+J)g(b+J)$.
        \item $g(1_{A/J})=g(1_A+J)=f(1_A)=1_R$.
    \end{itemize}
\end{proof}

\begin{theorem}[Primeiro Teorema do Isomorfismo]
    Seja $f:A\rightarrow R$ um homomorfismo de anéis. Então, existe um único homomorfismo de anéis $g:A/\ker f\rightarrow R$ tal que $g\circ q=f$, onde $q:A\rightarrow A/J$ é o mapa quociente canônico dado por $q(a)=a+J$, e $g:A\ker f\rightarrow \ran f$ é isomorfismo.
\end{theorem}
\begin{proof}
    Pelo Teorema do homomorfismo com $J=\ker f$, existe um único homomorfismo de anéis $g:A/\ker f\rightarrow \ran f$ tal que $g\circ q=f$. Como $g\circ q=f$ e $q$ é sobrejetora, então $\ran g=\ran (g\circ q)=\ran f$, logo, $q$ é sobre $\ran f$.
    
    Resta ver que $g$ é injetora. De fato, seja $q(a)\in A/\ker f$ tal que $g(q(a))=0_R$. Então, $f(a)=0_R$, logo, $a\in \ker f=J$. ou seja, $a\sim_I 0$, logo $q(a)=q(0)=0_{A/\ker f}$. Assim, $g$ é injetora.
\end{proof}